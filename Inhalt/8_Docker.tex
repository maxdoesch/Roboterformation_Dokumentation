% !TEX root = ../Projektdokumentation.tex
\chapter{Docker}\label{ch:docker}
Zur Vereinfachung des Arbeitsflusses und Ordnung einzelner Softwarekomponenten wurde \texttt{Docker} und \texttt{Docker Compose} verwendet.\\
Im folgenden Kapitel wird nun die Funktion und Vorgehensweise während der Nutzung von \texttt{Docker} erklärt.
\vspace{12cm}
\clearpage
\section{Docker}




\subsection{Einführung und Funktion von Docker}
Docker ist eine offene Plattform für die Entwicklung, Bereitstellung und Ausführung von Anwendungen.
Dabei verpackt Docker die Software in standardisierte Einheiten, sogenannte Container.
Diese enthalten alles, was zum Ausführen der Software erforderlich ist, wie Bibliotheken, Systemtools, Code und Laufzeit.\footnote{weitere Infos unter \url{https://docs.docker.com/get-started/overview/}}\\
Docker führt also Container aus.
Dabei fungiert ein Container als eine Art von Virtueller Maschine.
Allerdings ist hier der Aufbau relativ leicht, da dieser nicht virtualisiert ist und auf dem selben Kernel läuft.
Gleichzeitig ist der Container trotzdem abgeschottet, was dem Prinzip des Sandboxings
\footnote{Sandboxing ist eine Softwareverwaltungsstrategie, die Anwendungen von wichtigen Systemressourcen und anderen Programmen isoliert
Sie bietet eine zusätzliche Sicherheitsebene, die verhindert, dass sich Malware oder schädliche Anwendungen negativ auf Ihr System auswirken.\\Nähere Infos unter: \url{https://techterms.com/definition/sandboxing}} ähnelt.\\[\baselineskip]
Dabei erstellt man Container von einem Abbild.
Diese kann man selbst erstellen und nennt man Container-Image.
Der große Vorteil ist, dass man Container letztlich wie Rezepte zum Beschreiben verwendet, die ein leichtes Installieren der Software garantieren.\\
Eine Erweiterung stellt Docker Compose dar.
Dies erlaubt uns einen kompletten Stack zu beschreiben, also ganze Ansammlungen von Containern (und nicht nur einzelne Container) zusammen zu starten oder mit denen zu kommunizieren.\\[\baselineskip]
Bei einem Start des Projekts muss nur das repository gecloned werden und ein Befehl gestartet werden.
Diese Art zu Arbeiten ermöglicht einen schnellen Workflow.
\section{Installation von Docker-Desktop}
\begin{enumerate}
  \item Stellen Sie zunächst sicher, dass Windows auf dem neuesten Stand ist
  \begin{itemize}
    \item Geben Sie in der Windows-Suche "Windows Update" ein und wählen Sie \texttt{Windows Update-Einstellung}.
    \item Sie sollten ein grünes Häkchen sehen und "Sie sind auf dem neuesten Stand" . Falls nicht, klicken Sie auf "Nach Updates suchen". Sie müssen diesen Vorgang so lange wiederholen, bis Sie keine Updates mehr zu installieren haben.
  \end{itemize}
  \item Installation von \href{https://docs.microsoft.com/en-us/windows/wsl/install}{WSL2}
  \begin{itemize}
    \item Geben Sie in der Windows-Suche "powershell" ein, klicken Sie mit der rechten 
    Maustaste auf \texttt{Windows PowerShell} und dann auf \texttt{Als Administrator ausführen}.
    \item Klicken Sie auf "Ja", um PowerShell zu erlauben, Änderungen an Ihrem Gerät vorzunehmen.
    \item Führen Sie im Windows PowerShell-Fenster den Befehl "wsl --install -d Ubuntu" aus.
    \item Aktivieren Sie anschließend die Plattform für virtuelle Maschinen. Im Windows PowerShell ausführen. 
    (kopieren und einfügen) "dism.exe /online /enable-feature /featurename:VirtualMachinePlatform /all /norestart".
    \item Laden Sie das \href{https://wslstorestorage.blob.core.windows.net/wslblob/wsl_update_x64.msi}{WSL2 Linux-Kernel-Update-Paket für x64-Maschinen} herunter und installieren Sie es.
    \item Windows neu starten. 
    \item Geben Sie nochmal in der Windows-Suche "powershell" ein, klicken Sie mit der rechten 
    Maustaste auf \texttt{Windows PowerShell} und dann auf \texttt{Als Administrator ausführen}.
    \item Führen Sie im Windows PowerShell-Fenster den Befehl "wsl --set-default-version 2" aus.
    \item Als nächstes installieren Sie eine Linux-Distribution aus dem \href{https://aka.ms/wslstore}{Microsoft Store}. Ich empfehle \href{https://www.microsoft.com/store/productId/9MTTCL66CPXJ}{Ubuntu 20.04.4 LTS}
    .(Das Herunterladen und Installieren wird einige Minuten dauern)
    \item Sie werden aufgefordert, einen Linux-Benutzer einzurichten. Am besten Verwenden Sie dafür denselben Benutzernamen, den Sie für Windows verwenden.
    \item Sie können nun Linux-Befehle im Ubuntu-Terminalfenster ausführen. Ich empfehle, das Ubuntu-Symbol an die Taskleiste zu heften.
  \end{itemize}
  \item Jetzt können Sie \href{https://desktop.docker.com/win/main/amd64/Docker%20Desktop%20Installer.exe}{Docker Desktop} für Windows installieren
  \begin{itemize}
  \item Führen Sie das Installationsprogramm aus und starten Sie anschließend Windows neu.
  \item Melden Sie sich bei Windows an und starten Sie Docker-Desktop.Lassen sie Docker die Einrichtung abschließen, dies kann je nach Rechner einige Minuten dauern.
\end{itemize}
\end{enumerate}
\newpage
\section{Unsere Verwendung von Docker}
\subsection{Docker-Compose}
\label{dockercompose}
Sämtliche Software, die in unserem Projekt ausgeführt wird, läuft in Docker-Containern. Im folgenden werden die \texttt{Docker Compose}-Befehle, welche wir in unserem 
\textit{dokcer-compose.yml} Datei verwenden auführlich erklärt. 
\lstdefinelanguage{docker-compose-2}{
  keywords={version, volumes,networks,services},
  keywordstyle=\color{blue}\bfseries,
  keywords=[2]{image, environment, ports,networks,command,depends_on,context, container_name, ports, links, build},
  keywordstyle=[2]\color{olive}\bfseries,
  identifierstyle=\color{black},
  sensitive=false,
  comment=[l]{\#},
  commentstyle=\color{purple}\ttfamily,
  stringstyle=\color{red}\ttfamily,
  morestring=[b]',
  morestring=[b]"
}
\begin{lstlisting}[language=docker-compose-2,caption={docker-compose.yml}\label{buildBefehle},breaklines=true,basicstyle=\footnotesize]
version: '3.9' #Compose file format wird definiert, hierfuer wird Docker Engine 19.03.0 und hoeher verwendet
networks: #Mit diesen Befehlen erzeugen wir ein Netzwerk names rosnet
  rosnet:

services: #Unter services werden die einzelnen Containern definiert

  master: #Hier wird ein Container mit dem Namen "master" angelegt
    image: ros:noetic-robot # Als image wird hier "ros:noetic-robot" aus Docker-Hub verwendet 
    command:
      - roscore #Mit command wird der Befehl "roscore" in der Kommandozeile ausgefuehrt
    networks:
      - rosnet #networks definiert das Netzwerk des masters 

#Da, die Einstellungen und Befehle von den letzten zwei Container sich aehneln, wird es nur fuer ein Container erklaert.  
  rosbridge: #Hier wird ein Container mit dem Namen "rosbridge" angelegt
    build: 
      context: ./rosbridge #Als image, bzw build wird hier zu einem Dockerfile navigiert, welche unter dem Verzeichnis "./rosbridge" befindet
    environment:
      - "ROS_HOSTNAME=rosbridge"
      - "ROS_MASTER_URI=http://master:11311" #Unter environment werden Environmentvariablen festgelegt, in dem Fall wie der HOSTNAME des Containers lautet und wo sich der master im Netzwerk befindet
    networks:
      - rosnet 
    depends_on: 
      - master #Der Container faehrt erst hoch, wenn der master bereits laeuft und schaltet sich vorm master aus 
    ports:
      - 9090:9090 #Hier werden Ports in diesem Form (HOST-PORT:CONTAINER-PORT) nach aussen freigegeben 

  rosrobotbridge:
    build:
      context: ./rosrobotbridge
    environment:
      - "ROS_HOSTNAME=rosrobotbridge"
      - "ROS_MASTER_URI=http://master:11311"
    networks:
      - rosnet
    depends_on:
      - master
    ports:
      - 2888:2888
\end{lstlisting}
\newpage
\subsection{Dockerfile}
Im Unterkapitel \hyperref[dockercompose]{Docker-Compose} werden in einem \textit{dokcer-compose.yml} Datei mehrere \hyperref[buildBefehle]{build-Befehle} aufgerufen, welche zu dem eigentlichen \textit{Dockerfile} nagivieren und diesen als 
\newline Hauptimage bauen. Im folgenden werden die Befehle aus dem \textit{Dockerfile} und deren \newline Funktionen abhand eines Beispiels aus unserem Projekt erklärt und geschildert. 
\lstdefinelanguage{Dockerfile}{
  keywords={FROM, RUN,COPY,CMD},
  keywordstyle=\color{blue},
  identifierstyle=\color{orange},
  sensitive=false,
  comment=[l]{\#},
  commentstyle=\color{purple}\ttfamily,
  stringstyle=\color{red}\ttfamily,
  morestring=[b]',
  morestring=[b]"
}
\begin{lstlisting}[language=Dockerfile,caption={Dockerfile (rosbridge)},breaklines=true,basicstyle=\footnotesize]
  FROM ros:noetic-robot #Hier wird ein bereits vorhandenes Image aus dem Dockerhub aufgerufen und als Basis verwendet
  RUN apt-get -y update #Mit dem RUN-Befehl werden Befehle in der Kommandozeile ausgefuehrt
  RUN apt-get -y upgrade # -y sorgt dafuer, dass wenn es nach dem ausfuehren von dem Befehl eine JA-NEIN-Frage erscheinen soll, diese automatisch mit JA bzw YES beantwortet wird
  RUN apt-get -y install ros-noetic-rosbridge-server 
  COPY ./start.sh start.sh #COPY-Befehl kopiert die Datei start.h auf unserem Image. COPY SOURCE-Verzeichnis DESTINATTION-Verzeichnis
  RUN chmod +x start.sh
  CMD ["./start.sh"] #Die Datei start.h wird ausgefuehrt
\end{lstlisting}