\chapter{Web-App}\label{ch:webapp}
Die Web-App ist eine Browserapplikation, die manuelle Echtzeit Steuerung der Roboter und deren LED-Farben, unabhängig von deren Anzahl ermöglicht. Die Steuerung erfolgt
über eine ROS-Schnittstelle "'rosbridge"' mit der sich die Web-App, über ein Javabibliothek "'\href{http://wiki.ros.org/roslibjs}{roslibjs}"', mit dem Rosserver verbinden kann, 
um Daten einzulesen oder zusenden.
\newpage
\section{React}

React ist ein Javascript Bibliothek zum Entwickeln und Erstellen von Benutzeroberflächen.
React ist ein Opensource Projekt, welche damals von Facebook jetzt Meta entwickelt wurde.
Um in React Web-Applikationen zuerstellen, braucht man Grundkenntnisse in CSS, HTML und Javascript.\\
Die wichtigste Eigenschaft von React ist, dass die Zustände der Applikation und der Benutzeroberfläche synchronisiert agieren, das heißt, wenn Änderungen am Sourcecode vorgenommen werden
, verändert sich auch die Benutzeroberfläche.
React ist Komponenten basiert, ein Reactapp besteht daher aus vielen kleinen React-Komponenten, welche das Programmieren und die Wiederverwendbarkeit von Objekten erleichtern.
\lstdefinelanguage{npm}{
  keywords={npm},
  keywordstyle=\color{blue}\bfseries,
  keywords=[2]{install, start},
  keywordstyle=[2]\color{olive}\bfseries,
  identifierstyle=\color{black},
  sensitive=false,
  comment=[l]{\#},
  commentstyle=\color{purple}\ttfamily,
  stringstyle=\color{red}\ttfamily,
  morestring=[b]',
  morestring=[b]"
}
\section{Einrichten der Web-App}
\begin{enumerate}
    \item Vorbereiten der Umgebung 
    \begin{itemize}
        \item Zuallererst muss ein geeigneter Editor auf dem Rechner installiert sein. Empfohlen wird \href{https://code.visualstudio.com}{Visual Studio Code}, welches ebenfalls zum Entwickeln der Web-App verwendet wurde.
        \item Installation von \href{https://nodejs.org/en/}{Node.js} und npm\footnote[1]{npm ist ein Packet Manager, welcher bei der Installation von Node.js mit geliefert wird. Durch npm wird das Installieren und 
        Aktualisieren von Drittbibliotheken für Entwickeler mit kurzen \href{https://docs.npmjs.com/cli/v6/commands}{Befehlen} erleichtert.}. Bitte installiert hierzu die empfohlene Version und nicht die neuste, da es sein kann, dass 
        manche Befehle nicht mehr verfügbar oder überarbeitet wurden. 
    \end{itemize}
    \item Die \href{https://git.efi.th-nuernberg.de/gitea/sammarimo78617/Webapp-th.git}{Git-Repository} von der Web-App in einem Projekt-Ordner klonen. 
    \item Den Ordner "'Webapp-th"' in Visual Studio Code öffnen.
    \item Ein \href{https://thomaskrause.github.io/nlp-mit-python/01-python-starten/index.html#:~:text=Um%20ein%20neues%20Terminal%20in,”%20oder%20“sh”).}{neues Terminal} in VS-Code öffnen und folgende Befehle eingeben.
 
    \begin{lstlisting}[language=npm,caption={Befehle zum Starten der Web-App},breaklines=true,basicstyle=\footnotesize]
        npm install #installiert alle Packet, die sich in package.json befinden, welche fuer die Web-App essential sind. Dies kann je nach Internetverbindung eine gewisse Zeit in Anspruch nehmen 
        npm start #Startet die Web-App. Im Anschluss erscheint die Web-App in dem eingestellten Standardbrowser. 
    \end{lstlisting}
    \item Wenn alle Schritte erfolgreich abgeschlossen sind, erscheint folgendes im Terminalfenster.
    \begin{figure}[H]
        \centering
        \includegraphics[width=0.5\textwidth]{figures/Web-App/Web-App-Erfolgreich.png}
        \caption{Erfolgreicher Web-App-Start}
        \label{fig:erfolgreich-Web-App-Start}
    \end{figure}
Die Web-App wird nun Lokal auf dem Rechner und in unserem Netzwerk, wie es in Abbildung \ref{fig:erfolgreich-Web-App-Start} zu sehen ist, gehostet. Man kann daher 
die Web-App mit anderen Endgeräten, welche sich im unserem Netzwerk befinden, steuern. 
\end{enumerate}
\section{Bedienung der Web-App-Oberfläche}
Die Web-App besteht aus eine Hauptschaltfälche, die je sich je nach Funktion ändert. Die Änderung der Hauptschaltfläche erfolgt mit einem Rechtsklick auf die Funktions-Buttons, welche in der Sidebar vertikal untereinander platziert sind. Die Sidebar kann mit dem Switch-Button
auf- und zugeklappt werden. Oberhalb der Hauptschaltfläche ist der Topbar, welche das Logo der Hochschule und den Switch-Button beinhaltet. 
\begin{figure}[H]
    \centering
    \includegraphics[width=0.8\textwidth]{figures/Web-App/Web-App-Fläche.jpeg}
    \caption{Web-App Oberfläche}
    \label{fig:Web-App-Oberflaeche}
\end{figure}