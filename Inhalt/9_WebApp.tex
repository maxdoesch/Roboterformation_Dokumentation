\chapter{Web-App}\label{ch:webapp}
Die Web-App ist eine Browserapplikation, die manuelle Echtzeit Steuerung der Roboter und deren LED-Farben, unabhängig von deren Anzahl ermöglicht. Die Steuerung erfolgt
über eine ROS-Schnittstelle "rosbridge" mit der sich die Web-App, über ein Javabibliothek \href{http://wiki.ros.org/roslibjs}{"roslibjs"}, mit dem Rosserver verbinden kann, 
um Daten einzulesen oder zusenden.
\newpage
\section{React}

React ist ein Javascript Bibliothek zum Entwickeln und Erstellen von Benutzeroberflächen.
React ist ein Opensource Projekt, welche damals von Facebook jetzt Meta entwickelt wurde.
Um in React Web-Applikationen zuerstellen, braucht man Grundkenntnisse in CSS, HTML und Javascript.\\
Die wichtigste Eigenschaft von React ist, dass die Zustände der Applikation und der Benutzeroberfläche synchronisiert agieren, das heißt, wenn Änderungen am Sourcecode vorgenommen werden
, verändert sich auch die Benutzeroberfläche.
React ist Komponenten basiert, ein Reactapp besteht daher aus vielen kleinen React-Komponenten, welche das Programmieren und die Wiederverwendbarkeit von Objekten erleichtern.

