\chapter{Web-App}\label{ch:webapp}
Die Web-App ist eine Browserapplikation, die manuelle Echtzeit Steuerung der Roboter und deren LED-Farben, unabhängig von deren Anzahl ermöglicht. Die Steuerung erfolgt
über eine ROS-Schnittstelle "'rosbridge"' mit der sich die Web-App, über ein Javabibliothek "'\href{http://wiki.ros.org/roslibjs}{roslibjs}"', mit dem Rosserver verbinden kann, 
um Daten einzulesen oder zusenden.
\\ \colorbox{yellow}{\textbf{\textcolor{red}{Wichtiger Hinweis! }}}\\
\textbf{\textcolor{red}{Die Web-App verbindet sich nur mit der rosbridge, wenn man die IP-Adresse des Hostsrechners manuel in "'rosserver.js"' ändert. Stell also sicher, dass da die richtige IP-Adresse des Hostsrechners steht. Die Portnummer bleibt unverändert.}}
\newpage
\section{React}

React ist ein Javascript Bibliothek zum Entwickeln und Erstellen von Benutzeroberflächen.
React ist ein Opensource Projekt, welche damals von Facebook jetzt Meta entwickelt wurde.
Um in React Web-Applikationen zuerstellen, braucht man Grundkenntnisse in CSS, HTML und Javascript.\\
Die wichtigste Eigenschaft von React ist, dass die Zustände der Applikation und der Benutzeroberfläche synchronisiert agieren, das heißt, wenn Änderungen am Sourcecode vorgenommen werden
, verändert sich auch die Benutzeroberfläche.
React ist Komponenten basiert, ein Reactapp besteht daher aus vielen kleinen React-Komponenten, welche das Programmieren und die Wiederverwendbarkeit von Objekten erleichtern.
\lstdefinelanguage{npm}{
  keywords={npm},
  keywordstyle=\color{blue}\bfseries,
  keywords=[2]{install, start},
  keywordstyle=[2]\color{olive}\bfseries,
  identifierstyle=\color{black},
  sensitive=false,
  comment=[l]{\#},
  commentstyle=\color{purple}\ttfamily,
  stringstyle=\color{red}\ttfamily,
  morestring=[b]',
  morestring=[b]"
}
\section{Einrichten der Web-App}
\begin{enumerate}
    \item Vorbereiten der Umgebung 
    \begin{itemize}
        \item Zuallererst muss ein geeigneter Editor auf dem Rechner installiert sein. Empfohlen wird \href{https://code.visualstudio.com}{Visual Studio Code}, welches ebenfalls zum Entwickeln der Web-App verwendet wurde.
        \item Installation von \href{https://nodejs.org/en/}{Node.js} und npm\footnote[1]{npm ist ein Packet Manager, welcher bei der Installation von Node.js mit geliefert wird. Durch npm wird das Installieren und 
        Aktualisieren von Drittbibliotheken für Entwickeler mit kurzen \href{https://docs.npmjs.com/cli/v6/commands}{Befehlen} erleichtert.}. Bitte installiert hierzu die empfohlene Version und nicht die neuste, da es sein kann, dass 
        manche Befehle nicht mehr verfügbar oder überarbeitet wurden. 
    \end{itemize}
    \item Die \href{https://git.efi.th-nuernberg.de/gitea/sammarimo78617/Webapp-th.git}{Git-Repository} von der Web-App in einem Projekt-Ordner klonen. 
    \item Den Ordner "'Webapp-th"' in Visual Studio Code öffnen.
    \item Ein \href{https://thomaskrause.github.io/nlp-mit-python/01-python-starten/index.html#:~:text=Um%20ein%20neues%20Terminal%20in,”%20oder%20“sh”).}{neues Terminal} in VS-Code öffnen und folgende Befehle eingeben.
 
    \begin{lstlisting}[language=npm,caption={Befehle zum Starten der Web-App},breaklines=true,basicstyle=\footnotesize]
        npm install #installiert alle Packet, die sich in package.json befinden, welche fuer die Web-App essential sind. Dies kann je nach Internetverbindung eine gewisse Zeit in Anspruch nehmen 
        npm start #Startet die Web-App. Im Anschluss erscheint die Web-App in dem eingestellten Standardbrowser. 
    \end{lstlisting}
    \item Wenn alle Schritte erfolgreich abgeschlossen sind, erscheint folgendes im Terminalfenster.
    \begin{figure}[H]
        \centering
        \includegraphics[width=0.5\textwidth]{figures/Web-App/Web-App-Erfolgreich.png}
        \caption{Erfolgreicher Web-App-Start}
        \label{fig:erfolgreich-Web-App-Start}
    \end{figure}
Die Web-App wird nun Lokal auf dem Rechner und in unserem Netzwerk, wie es in Abbildung \ref{fig:erfolgreich-Web-App-Start} zu sehen ist, gehostet. Man kann daher 
die Web-App mit anderen Endgeräten, welche sich im unserem Netzwerk befinden, steuern. 
\end{enumerate}
\section{Bedienung der Web-App-Oberfläche}
Die Web-App besteht aus eine Hauptschaltfälche, die je sich je nach Funktion ändert. Die Änderung der Hauptschaltfläche erfolgt mit einem Rechtsklick auf die Navigationsbuttons, welche in der Sidebar vertikal untereinander platziert sind. Die Sidebar kann mit dem Switch-Button
auf- und zugeklappt werden. Oberhalb der Hauptschaltfläche ist der Topbar, welche das Logo der Hochschule und den Switch-Button beinhaltet. 
\begin{figure}[H]
    \centering
    \includegraphics[width=0.8\textwidth]{figures/Web-App/Web-App-Fläche.jpeg}
    \caption{Web-App Oberfläche}
    \label{fig:Web-App-Oberflaeche}
\end{figure}
Die Funktionnamen der einzelnen Navigationsbuttons werden eingeblendet, wenn man mit der Maus über die Buttons hovert. 
\subsection{Home-Button}
Nach dem Bedienen von Home-Button ändert sich die Hauptschaltfläche und es wird folgendes wie in der Abbildung \ref{fig:Web-App-Home} angezeigt. Mit dem Button "'Update State"' werden die Daten erneuert und die Anzahl der 
verbundnen Roboter aktualisiert. 
\begin{figure}[H]
    \centering
    \includegraphics[width=0.5\textwidth]{figures/Web-App/Home.png}
    \caption{Web-App/Home}
    \label{fig:Web-App-Home}
\end{figure}
\subsection{LED-Button}
Die Hauptschaltfläche ändert sich fogendermaßen wie in der Abbildung \ref{fig:Web-App-LED}, wenn man den LED-Button auswählt.\label{Dropdown-Menu} Oben links ist ein Dropdown-Menu, welche den jeweiligen Verbindungszustand
oder die \\ Roboternummber erfasst und anzeigen lässt. Diese wurde so eingestellt, dass jedesmal nach dem Aufruf der Menuleiste, die Verbindungszustände und die Roboternummer erneut erfasst werden, damit der 
Benutzer keine Falschdaten senden kann und ebenfalls sofort erfassen kann, wo das Problem liegt. Falls Verbindungsprobleme angezeigt werden, werden mögliche Problemsbehebungsmaßnahmen dem Benutzer
in der Menuleiste vorgeschlagen. Im besten Fall funktioniert alles und die verbundenen Roboternummber erscheinen in der Menuleiste. Nun kann der Benutzer ein Roboter selektieren, welches nach dem Selektieren 
in der Dropdown-Menu angezeigt wird. Die ausgewählte Roboternummer wird zwischengespeichert und man kann entweder die LED-Farben mit einem RGB-Generator einstellen und setzten oder vorprogrammierte Abläufe abspielen lassen. 
\begin{figure}[H]
    \centering
    \includegraphics[width=0.5\textwidth]{figures/Web-App/LED.png}
    \caption{Web-App/LED}
    \label{fig:Web-App-LED}
\end{figure}
\newpage
\subsection{Controller-Button}
Die Hauptschaltfläche vom Controller sieht wie folgt in der Abbildung \ref{fig:Web-App-Controller} aus. Der Dropdown-Menu funktioniert ähnlich, wie bei der \hyperref[Dropdown-Menu]{LED-Hauptschaltfläche} außer, dass die selektierte 
Roboternummber rechts davon angezeigt wird. Unten ist ein Schieberegler mit dem man die Geschwindigkeit einstellen kann\footnote[1]{Die maximal Geschwindigkeit muss manuel im Code, in der Datei "'Controller.js"' geändert werden, da die gepushte Geschwindigkeit, ein prozentuler Anteil der Maximalgeschwindigkeit ist. Dieser Prozentualanteil kann mit dem Schieberegler eingestellt werden}.
Der Benutzer hat die Wahl, den Roboter mit einem Joystick oder mit Controller-Buttons zu bedienen, die Bedienoberfläche ändert sich, wenn man auf den jeweilen Button drückt.  
\begin{figure}[H]
    \centering
    \includegraphics[width=0.5\textwidth]{figures/Web-App/Controller.png}
    \caption{Web-App/Controller}
    \label{fig:Web-App-Controller}
\end{figure}