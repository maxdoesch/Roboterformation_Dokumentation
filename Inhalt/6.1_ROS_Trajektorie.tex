% !TEX root = ../Projektdokumentation.tex
\chapter{ROS-Trajektorie}\label{ch:ros-trajec}
Dieses Kapitel behandelt die Umsetzung der durch die Vorgänger-Mastergruppe entwickelte Trajektorienplanung.
Genauso wird beschrieben, wie diese in ROS durchgeführt wird.
Falls zukünftig Änderungen und Erweiterungen vorgenommen werden, sollte die Vorgehensweise dazu ersichtlich werden.

\section{Trajektorienplanung}
Bisher wurde jeder berechneter Trajektorienpunkt 
\begin{lstlisting}[language=C++,caption={Speichern einzelner errechneter Punkte in einem Vektor},breaklines=true,basicstyle=\footnotesize]
    void ToRobot::storeInTrajectory(pos_d trajectory_state)
    {
        trajecgenerator::c_trajec trajectory_state_vector;

        trajectory_state_vector.x = trajectory_state.x;
        trajectory_state_vector.y = trajectory_state.y;
        trajectory_state_vector.dx = trajectory_state.dx;
        trajectory_state_vector.dy = trajectory_state.dy;
        trajectory_state_vector.ddx = trajectory_state.ddx;
        trajectory_state_vector.ddy = trajectory_state.ddy;
        trajectory_state_vector.timestamp = ros::Time::now().toNSec();

        trajectory.points.push_back(trajectory_state_vector);
    }
    
\end{lstlisting}

\begin{lstlisting}[language=C++,caption={Publishen der gesamten Trajektorie},breaklines=true,basicstyle=\footnotesize]
    bool ToRobot::publish()
    {
        float dist = sqrt(pow(trajectory.points.front().x - trajectory.points.back().x, 2) + pow(trajectory.points.front().y - trajectory.points.back().y, 2));

        if(dist < 0.1 && trajectory.points.size() > 100)
        {
            c_trajecPub.publish(trajectory);
            trajectory.points.clear();

            return true;
        }

        return false;
    }

\end{lstlisting}

\section{Erstellen eines individuellen ROS Message-Types}

