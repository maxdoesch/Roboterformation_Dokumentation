% !TEX root = ../Projektdokumentation.tex
\chapter{ROS-Trajektorie}\label{ch:ros-trajec}
Dieses Kapitel behandelt die Umsetzung der durch die Vorgänger-Mastergruppe entwickelte Trajektorienplanung.
Genauso wird beschrieben, wie diese in ROS durchgeführt wird.
Falls zukünftig Änderungen und Erweiterungen vorgenommen werden, sollte die Vorgehensweise dazu ersichtlich werden.

\section{Trajektorienplanung}
\begin{lstlisting}[language=cs,caption={},captionpos=b]
    void ToRobot::storeInTrajectory(pos_d trajectory_state)
    {
    trajecgenerator::c_trajec trajectory_state_vector;

    trajectory_state_vector.x = trajectory_state.x;
    trajectory_state_vector.y = trajectory_state.y;
    trajectory_state_vector.dx = trajectory_state.dx;
    trajectory_state_vector.dy = trajectory_state.dy;
    trajectory_state_vector.ddx = trajectory_state.ddx;
    trajectory_state_vector.ddy = trajectory_state.ddy;
    trajectory_state_vector.timestamp = ros::Time::now().toNSec();

    trajectory.points.push_back(trajectory_state_vector);
    }
    
\end{lstlisting}

