% !TEX root = ../Projektdokumentation.tex
\chapter{ROS-Trajektorie}\label{ch:ros-trajec}
Dieses Kapitel behandelt die Umsetzung der durch die Vorgänger-Mastergruppe entwickelte Trajektorienplanung.
Genauso wird beschrieben, wie diese in ROS durchgeführt wird.
Falls zukünftig Änderungen und Erweiterungen vorgenommen werden, sollte die Vorgehensweise dazu ersichtlich werden.

\section{Trajektorienplanung}
Bisher wurde durch den Code der Mastergruppe jeder berechneter Trajektorienpunkt einzeln an den jeweiligen Roboter geschickt.
Da dies bei eventuell auftretenden Verbindungsschwierigkeiten dazu führen könnte, dass die Trajektorie nicht vollständig oder zeitversetzt übergeben wird, war es unsere Zielsetzung die komplett bestimmte Trajektorie gesammelt über das Verbindungsprotokoll zu verschicken.\\
In der nachfolgend abgebildeten Funktion werden die einzelnen für die Trajektorie relevanten Parameter nacheinander in einem Vektor abgespeichert.
Diesem wird jeweils ein Zeitstempel hinzugefügt, dass passend zu den errechneten Punkten und deren Ableitungen auch die entsprechenden Zeitinformationen vorhanden sind.
Nachdem ein Punkt mit der x- und y-Koordinate sowie den dazugehörigen ersten und zweiten Ableitungen mit Zeitinformation nun den eigens erstellten message Typ \texttt{c\_trajec} innehaben, werden diese Informationen im Vektor gespeichert.\\
In \ref{sec:msg-type} wird beschrieben, wie der message Typ individuell erstellt wird.
\begin{lstlisting}[language=C++,caption={Speichern einzelner errechneter Punkte in einem Vektor},breaklines=true,basicstyle=\footnotesize]
    void ToRobot::storeInTrajectory(pos_d trajectory_state)
    {
        trajecgenerator::c_trajec trajectory_state_vector;

        trajectory_state_vector.x = trajectory_state.x;
        trajectory_state_vector.y = trajectory_state.y;
        trajectory_state_vector.dx = trajectory_state.dx;
        trajectory_state_vector.dy = trajectory_state.dy;
        trajectory_state_vector.ddx = trajectory_state.ddx;
        trajectory_state_vector.ddy = trajectory_state.ddy;
        trajectory_state_vector.timestamp = ros::Time::now().toNSec();

        trajectory.points.push_back(trajectory_state_vector);
    }
    
\end{lstlisting}
Die Punkte der Trajektorie werden fortlaufend berechnet und in einem Vektor mit all ihren Informationen gespeichert.
Damit festgestellt werden kann, zu welchem Zeitpunkt alle einzelnen Punkte berechnet und somit im Vektor gespeichert wurden, wurde in der folgenden Funktion eine Abbruchbedingung festgelegt.
Ist die Distanz zwischen zwei nacheinander berrechneten Punkten geringer als 10 cm beziehungsweise wurden bereits 100 Punkte für die Trajektorie berechnet, bricht die Trajektoriengenerierung ab.
Anschlie{\ss}end wird die fertige Trajektorie als Vektor komplett gepublisht.
\begin{lstlisting}[language=C++,caption={Publishen der gesamten Trajektorie},breaklines=true,basicstyle=\footnotesize]
    bool ToRobot::publish()
    {
        float dist = sqrt(pow(trajectory.points.front().x - trajectory.points.back().x, 2) + pow(trajectory.points.front().y - trajectory.points.back().y, 2));

        if(dist < 0.1 && trajectory.points.size() > 100)
        {
            c_trajecPub.publish(trajectory);
            trajectory.points.clear();

            return true;
        }

        return false;
    }

\end{lstlisting}
Beide Funktionen werden hierbei vom \texttt{Trajechandler} aufgerufen.\\[\baselineskip]
Mit diesen Parametern dauerte es circa 20 Sekunden bis alle Punkte der Trajektorie berechnet, die gesamte Trajektorie gepublisht und vom Roboter letztlich ausgeführt wurde.

\section{Erstellen eines individuellen ROS Message-Types}
\label{sec:msg-type}
 Um die Trajektorie in einem Array speichern zu können, in dem alle für die auf dem Mikrocontroller implementierten Regler wichtigen Informationen enthalten sind, musste ein neuer, individueller "`message type"' in ROS implementiert.
Da die Beschreibungen und Anleitungen hierzu im Internet etwas spärlich ausfallen, wird im Folgenden anhand des in der Projektarbeit verwendeten custom message Types "`c\_trajec.msg"' erklärt, wie ein neuer message Typ erstellt wird.
 Dies kann im Falle einer Erweiterung oder Änderung der ROS-Software von Bedeutung sein.
 \begin{enumerate}
    \item \textbf{Navigieren zum entsprechenden ROS Paket} \newline
    \bverb|roscd trajecgenerator|
    \item \textbf{Erstellen eines \texttt{msg} Ordners (sofern nicht bereits vorhanden)} \newline
    \bverb|mkdir msg|
    \item \textbf{Navigieren zum \texttt{msg} Ordner} \newline
    \bverb|cd msg| 
    \item \textbf{Definitionen des neuen Message Typs erstellen (entweder per Kommandozeile oder direkt in der Datei)} \newline
    \bverb|echo "float32 x" > msg/c_trajec.msg| \newline
    \bverb|echo "float32 y" > msg/c_trajec.msg| \newline
    \bverb|echo "float32 dx" > msg/c_trajec.msg| \newline
    \bverb|echo "float32 dy" > msg/c_trajec.msg| \newline
    \bverb|echo "float32 ddx" > msg/c_trajec.msg| \newline
    \bverb|echo "float32 ddy" > msg/c_trajec.msg| \newline
    \bverb|echo "uint64 timestamp" > msg/c_trajec.msg|
    \item \textbf{Überprüfen der Inhalte in der \texttt{c\_trajec.msg} Datei} \newline
    \bverb|cat c_trajec.msg| 
    \item \textbf{Editieren der \texttt{package.xml} Datei} \newline
    In der Datei \texttt{package.xml} sicherstellen, dass die Inhalte \newline
    \bverb|<build_depend>message_generation</build_depend>| \newline
    \bverb|<run_depend>message_runtime</run_depend>| vorhanden und auskommentiert sind.
    \item \textbf{Editieren der \texttt{CmakeLists.txt} Datei} \newline
    In der Datei \texttt{CmakeLists.txt} bei \texttt{COMPONENTS} den Inhalt \texttt{message\_generation} hinzufügen, damit mindestens Folgendes zu sehen ist: \newline
    \bverb|find_package(catkin REQUIRED COMPONENTS| \newline
    \bverb|     roscpp| \newline
    \bverb|     rospy| \newline
    \bverb|     std_msgs| \newline
    \bverb|     message_generation| \newline
    \bverb|)|
    \item \textbf{Einstellen der Dependencies in der \texttt{CmakeLists.txt} Datei} \newline
    \bverb|catkin_package(| \newline
    \bverb|CATKIN_DEPENDS roscpp std_msgs| \newline
    \item \textbf{Hinzufügen der Message-Datei in der \texttt{CmakeLists.txt} Datei} \newline
    \bverb|add_message_files(| \newline
    \bverb|FILES| \newline 
    \bverb|c_trajec.msg| \newline
    \bverb|# c_trajec_vector.msg| \newline
    \bverb|)| \newline
    \item \textbf{Editieren der \texttt{CmakeLists.txt} Datei} \newline
    In der Datei \texttt{CmakeLists.txt} sicherstellen, dass die Inhalte \newline
    \bverb|generate_messages(| \newline
    \bverb|     DEPENDENCIES| \newline
    \bverb|     std_msgs| \newline
    \bverb|)| \newline vorhanden und auskommentiert sind.
    \item \textbf{Erneutes Bauen des Packages} \newline
    \bverb|roscd trajecgenerator| \newline
    \bverb|catkin_make|
\end{enumerate}




