% !TEX root = ../Projektdokumentation.tex
\chapter{3D-Roboterdesign}\label{ch:3D}

\section{Allgemeine Veränderung}
Wie auch bei der Bachelor Vorgängergruppe, soll der Roboter schnell reproduziert werden. Jedoch gefiel uns nicht, dass die notwendigen Bauteile einzeln auf der Grundplatte 
des Roboters befestigt wurden. Zudem wurden diese Bauteile mit einzelnen Leitungen verbunden, was 
das Innere vom Roboter allgemein unübersichtlich machte und auch das Schnelle austauschen eines defekten Bauteils nicht ermöglichte. Wir lösten das Problem, indem wir 
zwei selbstentworfene Platinen benutzten, worauf die einzelnen Komponenten mit Leiterbahnen verbunden sind.

Im Vergleich zur Vorgängergruppe gab es auch eine Verbesserung für die LEDs. Die komplizierte Konstruktion, um vier einzelne LEDs anzusteuern, wurde durch die obere Platine ersetzt, welche mit LEDs bestückt war.
So ist es nun auch möglich farbige Animationen abzuspielen. 

Aus den Erfahrungen der Vorgängergruppen entschieden wir uns auch für einen Differentialantrieb. Die Räder wurden 3D gedruckt und anschließend mit einem Gummi Ring überzogen. 
Zu Beginn hatten wir als Stützrad eine Kugel, jedoch hatte diese eine zu große Reibung, sodass sie nur am Boden rumgeschliffen ist. Es wurde dann eine Konstruktion aus 
zwei Kugellagern entworfen, wodurch das Rumfahren mit dem Roboter deutlich verbessert wurde. 

Eine weitere Veränderung zur Vorgängergruppe ist, dass der Roboter nun ein kreisförmiges Design hat. Dies hat den Vorteil, dass man nicht so schnell 
an Kanten oder auch an anderen Robotern hängen bleibt. Es wurde ein kreisförmiges Gehäuse aus zwei halbkreisförmigen Teilen mithilfe eines 3D Druckers gedruckt. 
Zudem wird der Roboter auf der Oberseite von einer auch selbstgedruckten diffusen Plastikscheibe abgedeckt. Das hat den Sinn, dass die einzelnen LEDs nicht blenden und die 
gesamte Fläche vom Roboter gleichmäßig beleuchtet wird. 

\section{Aufbauanleitung}

\subsubsection*{Schritt 1: Stützrad}
Um das Stützrad zusammenzubauen muss zuerst das erste Kugellager mithilfe einer Schraube mit der Verbindung, welche auf die Platine geschraubt wird, 
verschraubt werden. Anschließend kann das zweite Kugellagerrad mithilfe eines gedruckten Bolzens in die zuvor zusammengeschraubte Halterung befestigt werden.
\begin{figure}[H]
    \centering
    \begin{minipage}{0.4\textwidth}
        \centering
        \includegraphics[width=\textwidth]{figures/Roboterdesign/Stützrad_1.png}
        \caption{Einzelteile des Stützrads}
    \end{minipage}
    \hfill
    \begin{minipage}{0.4\textwidth}
        \centering
        \includegraphics[width=\textwidth]{figures/Roboterdesign/Stützrad_2.png}
        \caption{Befestigtes Kugellager} 
    \end{minipage} 
\end{figure}

\begin{figure}[H]
    \centering
    \begin{minipage}{0.4\textwidth}
        \centering
        \includegraphics[width=\textwidth]{figures/Roboterdesign/Stützrad_3.png}
        \caption{Bolzen mit Kugellager}
    \end{minipage}
    \hfill
    \begin{minipage}{0.4\textwidth}
        \centering
        \includegraphics[width=\textwidth]{figures/Roboterdesign/Stützrad_4.png}
        \caption{Fertiges Stützrad} 
    \end{minipage}
\end{figure}

\subsubsection*{Schritt 2: Befestigung von Motoren}

Als nächstes werden die Motoren mit jeweils 4 Schrauben von der Oberen Platine montiert. 
Zudem wird die Steckverbindung an der Platine angeschlossen.

\begin{figure}[H]
    \centering
    \begin{minipage}{0.45\textwidth}
        \centering
        \includegraphics[width=\textwidth]{figures/Roboterdesign/Motoren_1.png}
        \caption{Steckverbindung der Motoren}
    \end{minipage}
    \hfill
    \begin{minipage}{0.45\textwidth}
        \centering
        \includegraphics[width=\textwidth]{figures/Roboterdesign/Motoren_2.png}
        \caption{Befestigung der Motoren} 
    \end{minipage}
\end{figure}

\subsubsection*{Schritt 3: Montage der Räder}
Die Beiden Räder werden jeweils an die Motorwellen gesteckt.
\begin{figure}[H]
    \centering
    \begin{minipage}{0.45\textwidth}
        \centering
        \includegraphics[width=\textwidth]{figures/Roboterdesign/raeder.png}
        \caption{Einzelteile des Stützrads}
    \end{minipage}

\end{figure}

\subsubsection*{Schritt 4: Diffuses Glas}
Die diffuse Plastikscheibe hat 2 kleine Löcher für die externen Mikrofone vom Marvelmind-Sensor. 
Diese wurden mithilfe von Heißkleber an die untere Seite geklebt. 

\begin{figure}[H]
    \centering
    \begin{minipage}{0.45\textwidth}
        \centering
        \includegraphics[width=\textwidth]{figures/Roboterdesign/diffuses_glas_1.png}
        \caption{Befestigt mit Heißkleber}
    \end{minipage}
    \hfill
    \begin{minipage}{0.45\textwidth}
        \centering
        \includegraphics[width=\textwidth]{figures/Roboterdesign/diffuses_glas_2_1.png}
        \caption{Befestigte Mikrofone} 
    \end{minipage}
\end{figure}

\subsubsection*{Schritt 5: Verbinden der Oberen Platine mit der Unteren}
Beim Verschrauben der beiden Platinen sollte zuerst sichergestellt werden, dass beide Platinen über die 12 polige Steckverbindung verbunden sind. 
Anschließend wird die untere Platine mit der oberen mithilfe von 4 Schrauben und 8 Abstandshalterungen verschraubt.

\begin{figure}[H]
    \centering
    \begin{minipage}{0.45\textwidth}
        \centering
        \includegraphics[width=\textwidth]{figures/Roboterdesign/schritt_5_1.png}
        \caption{Einzelteile}
    \end{minipage}
    \hfill
    \begin{minipage}{0.45\textwidth}
        \centering
        \includegraphics[width=\textwidth]{figures/Roboterdesign/schritt_5_2.png}
        \caption{Verbundene Platinen} 
    \end{minipage}
\end{figure}

\subsubsection*{Schritt 6: Außenhülle}
Bevor die beiden Außenteile miteinander verschraubt werden, muss erst die Halterung vom Stützrad mit 2 
Schrauben an das Gehäuse und die Platine verschraubt werden. Danach kann man die beiden Außenteile ebenfalls mit 2 Schrauben verschrauben. 

\begin{figure}[H]
    \centering
    \begin{minipage}{0.45\textwidth}
        \centering
        \includegraphics[width=\textwidth]{figures/Roboterdesign/aussenhuelle_1.png}
        \caption{Einzelteile der Hülle}
    \end{minipage}
    \hfill
    \begin{minipage}{0.45\textwidth}
        \centering
        \includegraphics[width=\textwidth]{figures/Roboterdesign/aussenhuelle_2.png}
        \caption{Zusammengesetzte Außenhülle} 
    \end{minipage}
\end{figure}

\subsubsection*{Schritt 7: Fertigstellung}
Abschließend wird das diffuse Glas mit 4 Schrauben von oben an den Roboter befestigt

\begin{figure}[H]
    \centering
    \begin{minipage}{0.45\textwidth}
        \centering
        \includegraphics[width=\textwidth]{figures/Roboterdesign/schritt_7.png}
        \caption{Einzelteile des Stützrads}
    \end{minipage}

\end{figure}

