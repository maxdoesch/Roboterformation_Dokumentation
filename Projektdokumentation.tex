\documentclass[ngerman]{report}
%\pdfminorversion=5 % erlaubt das Einfügen von pdf-Dateien bis Version 1.7, ohne eine Fehlermeldung zu werfen (keine Garantie für fehlerfreies Einbetten!)
\usepackage[utf8]{inputenc} % muss als erstes eingebunden werden, da Meta/Packages ggfs. Sonderzeichen enthalten

% !TEX root = Projektdokumentation.tex

% Hinweis: der Titel muss zum Inhalt des Projekts passen und den zentralen Inhalt des Projekts deutlich herausstellen
\newcommand{\titel}{Roboterformation}
\newcommand{\untertitel}{Fortführung}
\newcommand{\kompletterTitel}{\titel{} -- \untertitel}

\newcommand{\autorName}{Cara Bettendorf, Maximilian Dösch, Kevin Heise und Moin Sammari}

\newcommand{\betriebLogo}{TH-Nuernberg-Logo.jpeg}

\newcommand{\ausbildungsberuf}{Elektro- \& Informationstechnik}
\newcommand{\betreff}{Projektarbeit}
\newcommand{\pruefungstermin}{Sommer 2022}
\newcommand{\abgabeOrt}{Nürnberg}
\newcommand{\abgabeTermin}{23.06.2022}
 % Metadaten zu diesem Dokument (Autor usw.)
\input{Allgemein/Packages} % verwendete Packages
\input{Allgemein/Seitenstil} % Definitionen zum Aussehen der Seiten
\input{Allgemein/Befehle} % eigene allgemeine Befehle, die z.B. die Arbeit mit LaTeX erleichtern

\setlength{\headheight}{43.6pt}

\begin{document}

% kann nach dem Lesen entfernt werden ---------------------------------------
\pagestyle{plain}
%\cleardoublepage
\pagestyle{scrheadings}
% ---------------------------------------------------------------------------
\phantomsection
\thispagestyle{plain}
\pdfbookmark[1]{Deckblatt}{deckblatt}
% !TEX root = Projektdokumentation.tex
\begin{titlepage}

\begin{center}
\includegraphics[scale=0.8]{figures/TH-Nuernberg-Logo.jpeg}\\[1ex]
\Large{Fakultät}\\[1ex]

\Large{\ausbildungsberuf}\\[4ex]
\huge{\betreff}\\[4ex]

\huge{\textbf{\titel}}\\[1.5ex]
\Large{\textbf{\untertitel}}\\[4ex]

\begin{figure}[H]
    \centering
    \includegraphics[width=0.8\textwidth]{figures/Roboterformation_Render}
\end{figure}

\normalsize
Abgabetermin: \abgabeOrt, den \abgabeTermin\\[3em]
\textbf{Projektteilnehmer:}\\
\autorName\\
[5ex]

\end{center}

\small
\noindent

Dieses Werk einschlie{\ss}lich seiner Teile ist \textbf{urheberrechtlich geschützt}.
Jede Verwertung au{\ss}erhalb der engen Grenzen des Urheberrechtgesetzes ist ohne
Zustimmung des Autors unzulässig und strafbar. Das gilt insbesondere für
Vervielfältigungen, Übersetzungen, Mikroverfilmungen sowie die Einspeicherung
und Verarbeitung in elektronischen Systemen.

\end{titlepage}
%\cleardoublepage

% Preface --------------------------------------------------------------------
\phantomsection
\pagenumbering{Roman}
\pdfbookmark[1]{Inhaltsverzeichnis}{inhalt}
\tableofcontents

%\cleardoublepage

\phantomsection
\listoffigures
%\cleardoublepage

\phantomsection
\listoftables
%\cleardoublepage

% \newcommand{\abkvz}{Abkürzungsverzeichnis}
% \renewcommand{\nomname}{\abkvz}
% \section*{\abkvz}
% \markboth{\abkvz}{\abkvz}
% \addcontentsline{toc}{section}{\abkvz}
% \input{Abkuerzungen}
% \clearpage

% Inhalt ---------------------------------------------------------------------
\pagenumbering{arabic}
% !TEX root = Projektdokumentation.tex
% !TEX root = ../Projektdokumentation.tex
\chapter{Einführung in die Projektarbeit}

Als ein ansprechendes Modell zum Vorführen bei Messen oder Tag-der-offenen-Tür der Hochschule und des Lehrstuhls wurde das Projekt „Roboterformation“ in Auftrag gegeben.
Von Professor Bernhard Wagner betreut, soll im Laufe mehrerer Projektgruppen eine Formation von bis zu zwanzig mobilen Robotern entwickelt werden, die einen gesteuerten Tanz aufführen.
\\Diese Projektgruppe ist nunmehr das vierte Team, das sich dieser Aufgabe widmet.\\[\baselineskip]
An die Ergebnisse unserer Vorgänger Bachelor- und Mastergruppen anknüpfend ist es unser Ziel den bestehenden Roboter zu verbessern und sowohl hardware- als auch softwaretechnisch so auszulegen, damit dieser für die Formation in massentauglicher Stückzahl produziert werden kann.
Gleichzeitig ist es das Ziel, dass der Roboter genau einer vorgegebenen Trajektorie folgen kann. Hierbei liegt unser Fokus darin, die Grundlage dazu an einem bis zu maximal zwei Robotern zu schaffen.
Der Aufbau von mehreren Robotern als Formation, die synchronisierte Bewegungsabläufe vollziehen, wird erst von nachfolgenden Gruppen bearbeitet werden.\\[\baselineskip]
Die Hardware ist durch die vorangegangene Gruppe bereitgestellt worden.
Der Aufbau besteht aus zwei Rädern, die jeweils durch einen Motor angesteuert werden.
Zusätzlich ist ein Stützrad angebracht, welches aus zwei Kugellagern besteht, sodass eine dreieckförmige Anordnung entsteht.
Dadurch kann ein 360 Grad Fahren ermöglicht werden. Der hierzu benötigte Strom wird durch einen Akku mit 5V bereitgestellt.\\[\baselineskip]
Die Software wird zweigeteilt ausgeführt. Zum einen wird in einer ROS (Robot Operating System) Umgebung auf einem externen Computer die Trajektorienplanung des Roboters berechnet.
Hierbei stellt ROS ein Softwarepaket dar, das viele Bibliotheken mit sich bringt, die das Programmieren eines Roboters deutlich vereinfachen und in verschiedene Nodes aufgeteilt wird.
Der Code wird in der Programmiersprache C++ verfasst und sendet dann über sogenannte Publisher Daten an Topics.
Der ROS-Server speichert diese Nachrichten und stellt sie anderen Nodes, die dieses Topic abonnieren, den sogenannten Subscribern, zur Verfügung.
Zum Bestimmen der Trajektorie werden einzelne Wegpunkte, die der Roboter auf dem Weg zum gewünschten Endpunkt abfahren soll, mit den geeigneten mathematischen Splines ermittelt.
Nach der Berechnung wird die Trajektorie als Array mit allen errechneten Punkten und korrespondierenden Zeitwerten an den Mikrocontroller übergeben.
Auf diesem findet der zweite Teil der Software-Entwicklung statt.\\[\baselineskip]
Als Mikrocontroller verwendet unsere Projektgruppe den 2-Kern Prozessor ESP32.
Dieser verfügt über WLAN, Bluetooth, effizientes Powermanagement und verschiedene Peripherien als Funktionen.
Hierüber werden die Motoren angesteuert.
Um die Geschwindigkeit einzustellen, wird eine Drehzahlregelung verwendet.
Dazu wird an der Außenseite des Motors gemessen, wie oft sich das Rad innerhalb einer Sekunde vollständig gedreht hat.
Über die ermittelte Drehzahl kann mit der Differenzbildung zum Sollwert die entsprechend benötigte Geschwindigkeit eingestellt werden.
Gleichzeitig läuft auf dem ESP32 übergeordnet eine Top Level State machine.
Diese bestimmt, ob zu einem gewissen Zeitpunkt Positionen eingelesen oder Trajektorien berechnet wurden und gibt letztlich den Befehl, dass eine Trajektorie ausgeführt werden soll.
Da der Roboter und die zukünftige Formation besonders zu Werbezwecken aufgeführt werden soll, wird auch auf eine besonders ansprechende Visualisierung geachtet.
Dazu befinden sich auf der Platine über 40 LEDs, die je nach Wunsch in verschiedenen Farben erleuchten. Zudem wurde mittels des 3D-Druckers ein transparentes Gehäuse gedruckt, durch das das LED-Licht gestreut wird.
Auch die Ansteuerung des Lichts wird über den Mikrocontroller abgewickelt. 

Unsere Webseite stellt die Oberfläche für den Endnutzer zur Verfügung.
In den Programmiersprachen CSS und react.js verfasst, bietet sie die Schnittstelle zwischen der Eingabe der gewünschten Zielpunkte, Lichteffekte, manuellen Steuerung des Roboters sowie einer Anzeige der aktuellen Positionen der jeweiligen Roboter.
Grundsätzlich kann man sich den Prozessablauf wie folgt vorstellen: Auf einem zentralgesteuerten PC wird die Trajektorie im ROS-System berechnet und über das selbst entwickelte Verbindungsprotokoll an den Mikrocontroller geschickt.
Dieser sitzt auf der Platine, die auf dem Roboter aufgebracht ist.
Dann setzt sich der Roboter zum angegebenen Ort oder in der vorgegeben Formation in Bewegung.
Dem geht zuvor die Eingabe des gewünschten Punktes auf unserer Webseite einher.
Zusätzlich können dort Einstellungen zur Geschwindigkeit und Licht vorgenommen werden.



\section{Einordnung dieser Projektarbeit in das Gesamtprojekt}
Verwendung der durch die Mastergruppe vorgegebenen Regelungen und Trajektoriengenerierung $\rightarrow$ Integration/Implementierung auf echte Hardware
\section{Zielsetzung der Projektarbeit}


% !TEX root = ../Projektdokumentation.tex
\chapter{Platinenentwurf}\label{ch:platinenentwurf}
Ein Ziel dieser Projektarbeit war die Entwicklung einer Platform, welche alle notwendigen Komponenten kompakt zusammenfasst.
Hierfür müssen elektronische Bauteile und mechanische wie Motoren oder 3D-Druckteile gut aufeinander abgestimmt werden.
Während auf der einen Seite die Funktionalität des Gesamtsystems wichtig ist, soll andererseits jeder Roboter auch visuell durch ein Array von RGB-Leds auffallen.
Mithilfe eines doppel Platinenstack wurde erreicht, dass die Motoren direkt auf die untere Platine (Main\_PCB) aufgeschraubt werden können und 
die Leds auf der oberen Platine (Led\_PCB) den richtigen Abstand zum Led-Diffusor haben. 


\section{Blockschaltbild}
\begin{figure}[h]
    \centering
    \includegraphics[width=\linewidth]{figures/Platinenentwurf/PCB_Blockdiagramm.png}
    \caption{Blockdiagramm PCB}
\end{figure}
Das Blockdiagramm zeigt den Leistungs- und Signalfluss zwischen den Bauteilen auf beiden Platinen.
Die primäre Stromversorgung übernehmen drei 18650 LiIon Zellen. Diese haben im geladenen Zustand eine Gesamtspannung von $3*4.2 = 12.6V$ 
und können über einen Schalter an- und ausgeschalten werde. Der Roboter ist gegenüber Verpolung aller Zellen geschützt.
Für die Versorgung des Mikrocontrollers müssen die 12V mit einem Buck-Converter auf 3.3V gewandelt werden.
Außerdem soll der Roboter für den Test-/Programmierbetrieb auch über die 5V der USB-C oder der JTAG/Programmierschnittstelle stationär betrieben werden können.
Hierfür müssen die Spannungen elektronisch getrennt werden, um einen Kurzschluss zwischen den Potentialen zu verhindern.
Der ESP32 kann über die USB-C Schnittstelle und eine separate serielle Schnittstelle programmiert werden.
Für den 2 Quadranten betrieb der Motoren ist eine doppel H-Brücke vorgesehen. 
Die Ist-Geschwindigkeit wird über Hallsensoren auf den Bürstenmotoren ermittelt.
Auf die zweite Platine wird der Marvelmind Mini-RX Empfänger aufgesteckt. Dieser kommuniziert über UART mit dem Mikrocontroller. 
Die beiden externen Mikrofone können einfach auf die LED\_PCB Platine gelötet werden. 
Da das Marvelmind Modul 5V benötigt werden diese über einen zweiten Buck-Converter erzeugt.
Die 43 RGB Led-ICs werden über die 12V Spannung versorgt und mittels einer Datenleitung vom Mikrocontroller angesteuert.
Da die Led Bausteine 5V Logikpegel benötigen, wird das 3.3V Signal vom ESP32 mittels 1-Bit Level Shifter verstärkt.



\section{Bauteile}
\subsection{ESP32}
Das Herzstück des Roboters bildet ein ESP32-WROVER-E in der 8MB Flash Variante. 
Dieses SOM (System On Module) von Espressife beinhaltet einen Xtensa LX6 Dualcore Mikrocontroller mit 520kB internen
und 8MB externen SRAM, sowie 2MB internen und 8MB externen Flash Speicher. 
Die externen Speicher sind über SPI verbunden und werden über eine MMU (Memory Management Unit) in den Speicherbereich gemappt.
Des Weiteren bietet der ESP32 verschiedene Peripherie wie Timer, GPIO, UART, I2C, PWM und Wifi.
Espressife stellt mit dem ESP-IDF Framework unter anderem ein Hardwareabstraktions Layer bereit über welches Peripherie leicht initialisiert und gesteuert werden kann.
Aus diesem Grund bleibt es dem Programmierer erspart sich mit der Hardware auf Registerebene zu beschäftigen.
Dies macht den ESP32 zu einem beliebten IOT-Mikrocontroller im Hobbybereich.
Diagramm \ref{fig:esp32pinlayout} und Tabelle \ref{tab:esp32pinbelegung} beschreibt die Pinbelegung am ESP32.

\begin{figure}[H]
    \centering
    \begin{minipage}{0.35\textwidth}
        \centering
        \includegraphics[width=\textwidth]{figures/Platinenentwurf/ESP32_Pin_Layout.png}
        \caption{ESP32 Pin Layout}
        \label{fig:esp32pinlayout}
    \end{minipage}
    \hfill
    \begin{minipage}{0.55\textwidth}
        \centering
        \resizebox*{\textwidth}{!}{
        \begin{tabular}{|l|l|l|}
            \hline
            Name       & Pin Nr. & Funktion                                                 \\ \hline
            GND        & 1       & Ground                                                   \\ \hline
            3V3        & 2       & Power                                                    \\ \hline
            EN         & 3       & Enable Signal                                            \\ \hline
            Sensor\_VP & 4       & Hall Sensor 1 Motor B                                    \\ \hline
            Sensor\_VN & 5       & Hall Sensor 2 Motor B                                    \\ \hline
            IO34       & 6       & Hall Sensor 1 Motor A                                    \\ \hline
            IO34       & 7       & Hall Sensor 2 Motor A                                    \\ \hline
            IO32       & 8       & SCL I2C Bus                                              \\ \hline
            IO33       & 9       & SDA I2C Bus                                              \\ \hline
            IO25       & 10      & serielles LED Signal                                     \\ \hline
            IO26       & 11      & Marvelmind UART TX                                       \\ \hline
            IO27       & 12      & Marvelmind UART RX                                       \\ \hline
            IO14       & 13      & JTAG\_TMS                                                \\ \hline
            IO12       & 14      & JTAG\_TDI                                                \\ \hline
            GND        & 15      & Ground                                                   \\ \hline
            IO13       & 16      & JTAG\_TCK                                                \\ \hline
            IO15       & 23      & JTAG\_TDO                                                \\ \hline
            IO2        & 24      & GPIO2 (sollte nicht verwendet werden, da Strapping Pin)  \\ \hline
            IO0        & 25      & GPIO0 Boot Mode Selektor                                 \\ \hline
            IO4        & 26      & NC                                                       \\ \hline
            IO5        & 29      & GPIO5 (sollte nicht verwendet werden, da Strapping Pin)  \\ \hline
            IO18       & 30      & DRV8841 Motor B IN2                                      \\ \hline
            IO19       & 31      & DRV8841 Motor B IN1                                      \\ \hline
            IO21       & 33      & DRV8841 Motor A IN1                                      \\ \hline
            RXD0       & 34      & UART für Programmierung RX                               \\ \hline
            TXD0       & 35      & UART für Programmierung TX                               \\ \hline
            IO22       & 36      & DRV8841 Motor A IN2                                      \\ \hline
            IO23       & 37      & DRV8841 Enable                                           \\ \hline
            GND        & 38      & Ground                                                   \\ \hline
        \end{tabular}  
        }   
    \caption{ESP32 Pin Belegung}
    \label{tab:esp32pinbelegung}        
    \end{minipage}
\end{figure}

\subsubsection*{Power-Up}
\begin{figure}[H]
    \centering
    \includegraphics[width=0.5\textwidth]{figures/Platinenentwurf/ESP32 Power-Up.png}
    \caption{ESP32 Power-Up}
    \label{fig:esp32powerup}
\end{figure}
Wie die Illustation \ref{fig:esp32powerup} zeigt muss der Enable Pin (im Datenblatt CHIP\_PU genannt) für eine bestimmte Zeit $t_0 = 50\mu s$ nach dem Power-Up auf Low gehalten werden.
Das gleiche gilt für den Reset des Mikrocontrollers $t_1 = 50\mu s$. Die Power-Up/Reset Schaltung könnte zum Beispiel als Parallelschaltung eines RC Tiefpasses mit einem Taster umgesetzt werden.

\subsubsection*{Boot-Mode}
Der ESP32 verfügt über zwei verschiedene Boot Modis. Beim SPI Boot wird die Firmware aus dem Flash in den Arbeitsspeicher geladen.
Der Download Boot dient zum flashen einer neuen Firmware. Für den Wechsel zwischen den Modis wertet der Mikrocontroller beim Power-Up die Strapping Pins GPIO0 und GPIO2 aus.
\begin{table}[H]
    \centering
    \begin{tabular}{|l|l|l|l|}
    \hline
    Pin   & Default   & SPI Boot   & Download Boot \\ \hline
    GPIO0 & Pull-Up   & 1          & 0             \\ \hline
    GPIO2 & Pull-Down & Don't care & 0             \\ \hline
    \end{tabular}
    \caption{ESP32 Boot Mode}
    \label{tab:esp32bootmode}
\end{table}
Da GPIO2 intern auf LOW gezogen wird, muss dieser Pin nicht verbunden werden. Die Boot Modus Wahl geschieht dann alleine über GPIO0.

\subsubsection*{UART0}
Über die Pins UART0-RX/TX wird eine serielle Verbindung erstellt. 
Diese ermöglicht das Schreiben in den Flash-Speicher im Dowload-Boot und einen seriellen Monitor für z.B. printf Debugging im SPI-Boot.

\subsubsection*{JTAG}
TMS, TDI, TCK, TDO können für JTAG-Debugging verwendet werden.

\subsubsection*{I2C}
Auf beiden Platinen sind Anschlüsse an den I2C-Bus des ESP32 vorgesehen. Über diesen soll aber primär eine 9DOF-IMU (Intertial Measurement Unit) betrieben werden.

\subsection{CP2102N}
\label{sub:cp2102n}
Der CP2102N wirkt als "Übersetzer" zwischen der USB2.0 Fullspeed Schnittstelle und dem UART Interface.
Somit kann durch einen Virtual COM Port Treiber am Host-PC über USB- auf die UART-Schnittstelle zugegriffen werden.
Die UART Seite wird über die Pins RXD/TXD und das differenzielle USB Signal über die D+/D- Pins mit dem CP2102N verbunden.
Dies ermöglicht das Programmieren des ESP32 einfach über USB. Außerdem kann auch auf den seriellen Monitor zugegriffen werden.
Des Weiteren kann das RTS (Ready to Send; LOW aktiv) und DTR (Data Terminal Ready; LOW aktiv) Signal zur Auswahl des Boot Modus und Neustarten des Mikrocontrollers verwendet werden.
Die folgende Tabelle \ref{tab:cp2102n_dtr_rts} zeigt die notwendige logische Verknüfung der Signale.

\begin{table}[H]
    \centering
    \begin{tabular}{ll|ll}
    DTR & RTS & EN & GPIO0 \\ \hline
    1   & 1   & 1  & 1     \\
    0   & 0   & 1  & 1     \\
    1   & 0   & 0  & 1     \\
    0   & 1   & 1  & 0    
    \end{tabular}
    \caption{ESP32 DTR RTS}
    \label{tab:cp2102n_dtr_rts}
\end{table}

\subsection{Pololu 20D 63:1 Getriebemotoren}
\begin{figure}[H]
    \centering
    \begin{minipage}{0.40\textwidth}
        \centering
        \includegraphics[width=\textwidth]{figures/Platinenentwurf/Pololu_20D_Motor.jpg}
        \caption{Pololu 20D Motor}
        \label{fig:pololu_motor}
    \end{minipage}
    \hfill
    \begin{minipage}{0.40\textwidth}
        \centering
        \includegraphics[width=\textwidth]{figures/Platinenentwurf/Pololu_20D_Rueckplatte.jpg}
        \caption{Pololu 20D Rückplatte}
        \label{fig:pololu_rueckplatte}      
    \end{minipage}
\end{figure}
Die Pololu 20D Getriebemotoren sind für 12V Spannung ausgelegt. Dabei haben sie einen Leerlaufstrom von $80mA$ und im Stillstand $1.6A$.
Die Leerlaufdrehzahl wird durch das eingebaute Getriebe um den Faktor 63 auf $220RPM$ reduziert.
Für Drehzahlmessungen kann der Motor durch eine Platine und eine Magnetscheibe mit 10 Polen an der Motorwelle erweitert werden.
Auf der Platine befinden sich jeweils zwei Hall-Magnetfeldsensoren. Diese geben je nach Magnetfeldrichtung einen HIGH oder LOW Spannungspegel aus.
Durch Zählen der Pegeländerungen pro Zeiteinheit kann die Geschwindigkeit des Motors berechnet werden.
Die Drehrichtung wird über das Vorzeichen der Phasendifferenz der Signale von den beiden Hallsensoren ermittelt.

\subsection{DRV8841}
\begin{figure}[H]
    \centering
    \begin{minipage}{0.4\textwidth}
        \centering
        \includegraphics[width=\textwidth]{figures/Platinenentwurf/DRV8841_Pinout.png}
        \caption{DRV8841 Pinout}
        \label{fig:drv8841pinout}
    \end{minipage}
    \hfill
    \begin{minipage}{0.45\textwidth}
        \centering
        \resizebox*{\textwidth}{!}{
            \begin{tabular}{|l|l|l|l|}
                \hline
                xIN1 & xIN2 & xOUT1 & xOUT2 \\ \hline
                0    & 0    & L     & L     \\ \hline
                0    & 1    & L     & H     \\ \hline
                1    & 0    & H     & L     \\ \hline
                1    & 1    & H     & H     \\ \hline
            \end{tabular}
        }   
    \caption{DRV8841 H-Brücken Logik}
    \label{tab:drv8841_hbridgelogic}        
    \end{minipage}
\end{figure}
Mit der doppel H-Brücke DRV8841 von Texas Instruments können zwei DC-Motoren im 2 Quadranten-Betrieb betrieben werden.
Die beiden Motorausgänge (AOUT/BOUT) dürfen jeweils mit bis zu 2.5A belastet werden. 
Die Pololu 20D Motoren haben einen Stillstandsstrom von maximal 1.6A. 
Somit sollte der Treiber auch ohne die interne Strombegrenzung sicher betrieben werden können.
Dies bedeutet die Pins AI0, AI1, BI0, BI1 müssen nicht verbunden werden, ISENA und ISENB können direkt mit GND verbunden werden
und AVREF, BVREF werden auf 3.3V gezogen.
Geschwindigkeit und Drehrichtung der Motoren wird über die Pegel und Pulseweite der (AIN1/AIN2; BIN1/BIN2) 
Eingangssignale gesteuert. \ref{tab:drv8841_hbridgelogic}
Um den Motortreiber zu aktivieren müssen nRESET und nSLEEP auf 3.3V gezogen werden.
Des Weiteren ist auch noch ein Bootstrap-Kondensator zwischen den Pins CP1/CP2 als Ladepumpe notwendig.

\subsection{Marvelmind Mini-RX}
\begin{figure}[H]
    \centering
    \begin{minipage}{0.45\textwidth}
        \centering
        \includegraphics[width=\textwidth]{figures/Platinenentwurf/Marvelmind MiniRX Molex.png}
        \caption{Marvelmind Molex Pinout}
        \label{fig:marvelmind_molexpinout}
    \end{minipage}
    \hfill
    \begin{minipage}{0.45\textwidth}
        \centering
        \includegraphics[width=\textwidth]{figures/Platinenentwurf/Marvelmind MiniRX mic.png}
        \caption{Marvelmind Mic Pinout}
        \label{fig:marvelmind_micpinout}      
    \end{minipage}
\end{figure}
Der Marvelmind Mini-Rx Beacon empfängt die Ultraschallsignale der HW4.9 Beacons für die Berechnung der Pose des Roboters.
Über die integrierte UART Schnittstelle kann dann die Pose an den Mikrocontroller gesendet werden.
Da durch das Modul effektiv eine dritte Platine entsteht, wurde darauf geachtet diese möglichst platzschonend in den Platinenstack zu integrieren.
Hierfür wird das Gehäuse und der eingebaute Akku entfernt. Die Stromversorgung wird über auf der LED Platine erzeugte 5V gewährleistet.
Die 5V Spannung sowie die seriellen Signale RX/TX sind über die interne Molex PicoBlade Steckverbindung \ref{fig:marvelmind_molexpinout} mit dem Modul verbunden.
Da die UART Schnittstelle mit 3.3V Logikpegeln arbeitet kann diese ohne Probleme direkt mit dem ESP32 verbunden werden.
Des Weiteren soll das eingebaute Mikrofon durch zwei externe ersetzt werden. Diese sind über das 2.0mm 2x4 Pinout mit dem Mini-RX verbunden \ref{fig:marvelmind_micpinout}.

\subsection{WS2815B}
\begin{figure}[H]
    \centering
    \includegraphics[width=0.4\textwidth]{figures/Platinenentwurf/WS2815B_Pinout.png}
    \caption{WS2815B Pinout}
    \label{fig:ws2815B_pinout}
\end{figure}
Der WS2815B ist eine RGB-Led mit eingebautem LED-Treiber. Dieser Baustein kann über die Pins DO, DIN in Reihe zu einem Led-Array verschalten werden. 
Der RGB Farbwert wird in einem 24Bit (8Bit grün, 8 Bit rot, 8 Bit blau) Feld codiert. Dieser Wert wird für alle LEDs in einem Datenstrom kaskadiert und an den Eingang DIN der erste LED gesendet.
Diese schneidet sich den ersten Farbwert ab und gibt den restlichen Datenstrom über DO an die nächste LED weiter. Somit sind alle Bausteine einzeln adressierbar.
Des Weiteren wird 0 und 1 nicht über den Wert der Spannungspegel codiert. Um ein Bit zu übertragen benötigt es zu Beginn eine positive Flanke, dann eine negative und zum Schluss wieder ein positive Flanke.
Der Wert wird dann in die Dauer der High und Low-Phasen codiert.
Die LED wird mit 12V gespeist. Die Logikpegel der Datensignale arbeiten allerdings mit 5V Spannung. An den VCC Pin kann ein Entkopplungskondensator angeschlossen werden.
Außerdem wird der Eingang BIN mit DO der vorletzten LED verbunden und bringt somit Redundanz beim Aufall eines Treibers.

\subsection{PI4ULS5V201}
Da die LED-Treiber 5V Signalpegel benötigen, können diese nicht direkt mit dem ESP32 angesteuert werden.
Der PI4ULS5V201 ist ein Level-Shifter IC und kann somit den 3.3V Pegel des ESP32 auf die benötigten 5V verstärken. 

\subsection{TPS54331}
\label{sub:tps54331}
\begin{figure}[H]
    \centering
    \includegraphics[width=0.4\textwidth]{figures/Platinenentwurf/TPS54331.png}
    \caption{TPS54331 Pinout}
    \label{fig:tps54331_pinout}
\end{figure}
Auf den beiden Platinen gibt es insgesamt drei Spannungen (12V, 5V, 3.3V). 
Mit dem Buck Converter TPS54331 werden jeweils 5V und 3.3V aus den 12V erzeugt.
Da mit 5V und 3.3V keine Leistung gespeist wird, ist der Baustein mit maximal 3A Dauerstrom ausreichend dimensioniert.
Abbildung \ref{fig:tps54331_pinout} zeigt eine typische Verschaltung des Buck Converters.
Die Ausgangsspannung kann über die beiden Widerstände $R_{D1}$ und $R_{D2}$ eingestellt werden.
Das Datenblatt liefert eine Tabelle für die Dimensionierung aller passiven Bauelemente je nach Eingangs- und Ausgangsspannung.
Die beiden Widerstände $R_{en1}$ und $R_{en2}$ am EN Eingang dienen zur Einstellung des erlaubten Spannungsbereiches für VIN und werden mit folgender Formel berechnet:
$$R_{en1} = \frac{V_{start} - V_{stop}}{3\mu A}$$
$$R_{en2} = \frac{1.25V}{\frac{V_{start} - 1.25V}{R_{en1}} + 1\mu A}$$

\section{Schaltplan}
Die Schaltpläne für dieses Projekt wurden mit KiCad erstellt. KiCad ist ein freies ECAD Programm unter der GNU GPL Lizenz. 
Es integriert unter anderem Tools wie einen Schaltplan Editor "Eeschema" und einen Layout Editor "PCBNew".
Die beiden Schaltpläne MainPCB und LedPCB für dieses Projekt liegen als KiCad und PDF Dateien bei.
Die Bauteile wurden nach den Vorgaben und Applikationsbeispielen in den Datenblättern verschaltet.
Aus diesem Grund wird im Folgenden nur auf Besonderheiten eingegangen.

\subsection*{Hauptschalter und Verpolungsschutz}
\begin{figure}[H]
    \centering
    \includegraphics[width=0.4\textwidth]{figures/Platinenentwurf/KiCad_Powerswitch.png}
    \caption{KiCad Hauptschalter und Verpolungsschutz}
    \label{fig:kicad_hauptschalter_verpolungsschutz}
\end{figure}
Dieser Teil ermöglicht das Anschalten über einen mechanischen Schalter. 
Außerdem wird die Platine vor Verpolung der LiIon Zellen und Überstrom/Kurzschluss geschützt.
Der IRF7404 ist ein PMOS Transistor mit niedrigem Drain-Source Widerstand.
Befindet sich der Schalter SW1 in der dargestellten Stellung ist $V_{GS} = 0V$ und der Mosfet sperrt.
In der zweiten Schalterstellung wird $V_{GS} = 0V - V_{+BATT} \approx -12V$ und der Mosfet wird leitend.
Q2 wirkt als Verpolungsschutz für die gesamte Schaltung. Bei richtiger Polung von $V_{+BATT}$ ist die Body-Diode leitend und
$V_{GS} \approx -12V$. Bei Verpolung hat das Transistor Gate das Potential 12V. $V_{GS} < 0V$ ist somit unmöglich und der Mosfet sperrt.
Die Sicherung F1 schützt den Rest der Schaltung vor Überlast und Kurzschluss.

\subsection*{Eingangsquellenschutz}
\begin{figure}[H]
    \centering
    \includegraphics[width=0.4\textwidth]{figures/Platinenentwurf/KiCad_Eingansquellen_Schutz.png}
    \caption{KiCad Eingangsquellenschutz}
    \label{fig:kicad_eingangsquellenschutz}
\end{figure}
Die obere Schaltung befindet sich am Eingang des 3.3V Spannungswandlers.
Da dieser den Mikrocontroller versorgt, muss der Buck Converter über die LiIon Akkus (12V), USB (5V) und JTAG-/Programmierinterface (5V) betrieben werden können.
Es reicht nicht aus die verschiedenen Spannungschienen direkt zu verbinden, 
da sonst die Gefahr eines Kurzschlusses zwischen den Potentialen besteht, wenn mehrere Spannungsquellen gleichzeitig aktiv sind.
VBUS und 5VD werden durch Dioden vor Potentialausgleich geschützt. Diese Spannungen werden nur zum Debuggen verwendet. 
Es fließt nur wenig Strom und der Spannungsabfall über den Dioden ist somit hinnehmbar. 
Da die 12V den Roboter im Normalbetrieb mit teilweise großer Stromaufnahme versorgen, darf es keinen großen Spannungsabfall über der schützenden Komponente geben.
Der PMOS Transistor AO3401 hat einen geringen Drain-Source Widerstand und somit wenig Leistungsverlust bei großem Strom.
Sind 12V verbunden schaltet Q3 immer durch da $V_{GS} < 0V$. Falls nur VBUS oder 5VD aktiv ist, wird $V_{GS} = 0V$. Die 12V Schiene bleibt somit Spannungsfrei.

\subsection*{Reset- und Bootmodeschaltung}
\begin{figure}[H]
    \centering
    \includegraphics[width=0.4\textwidth]{figures/Platinenentwurf/KiCad_Boot_Reset.png}
    \caption{KiCad Reset- und Bootmodeschaltung}
    \label{fig:kicad_resetboot}
\end{figure}
In \ref{sub:cp2102n} wurde gezeigt, dass es möglich ist den ESP32 über das DTR und RTS Signal neuzustarten und den Bootmodus zu wechseln.
Die obere Schaltung setzt die hierfür notwendige logische Verknüfung zwischen DTR/RTS und EN/GPIO0 \ref{tab:cp2102n_dtr_rts} um.
Die beiden SS8050 sind npn-Bipolartransistoren. Diese Schaltung ist parallel geschaltet zu der Möglichkeit ESP\_EN und ESP\_IO0 über einen Taster auf GND zu ziehen.

\section{Layout}
Das Layout wurde mit dem Tool PCBNew aus KiCad erstellt. 
Die MainPCB Platine \ref{fig:kicad_mainpcb_layout} wurde als 4 Layer PCB umgesetzt (1. Schicht: Signal, 2.: 3.3V, 3.: 12V, 4.: GND) und LedPCB \ref{fig:kicad_ledpcb_layout} als 2 Layer (1.: 12V/Signal, 2.: GND).
Bei der Erstellung wurde versucht Angaben in den Datenblättern der Bauteile sowie "Good Practices" des Leiterplattenentwurf zu befolgen.
Im Folgenden wird somit nicht weiter darauf eingegangen.
\begin{figure}[H]
    \centering
    \includegraphics[width=0.8\textwidth]{figures/Platinenentwurf/KiCad_MainPCB_Layout.png}
    \caption{KiCad MainPCB Layout}
    \label{fig:kicad_mainpcb_layout}
\end{figure}
\begin{figure}[H]
    \centering
    \includegraphics[width=0.8\textwidth]{figures/Platinenentwurf/KiCad_LedPCB_Layout.png}
    \caption{KiCad LedPCB Layout}
    \label{fig:kicad_ledpcb_layout}
\end{figure}

\section{Designfehler}
Beim Schaltplan Design wurden leider 3 Fehler gemacht. Diese können allerdings ausgebessert werden oder beinträchtigen die Nutzung der Platine nur gering.
Der Dokumentation liegen einmal das fehlerhafte KiCad Projekt und eines mit ausgebesserten Fehlern bei.
\subsection*{3.3V Spannungswandler}
\begin{description}
    \item[Fehlerbeschreibung:] Der 3.3V Spannungwandler schaltet bei der Wandlung von 5V (USB, JTAG-/Programmierinterface) zu 3.3V ab. 
    Die Wandlung von 12V zu 3.3V funktioniert. 
    \item[Grund:] Über den Eingang EN des TPS54331 wird ein Unterspannungsschutz (UVLO) realisiert. Fällt die Spannung an EN unter $V_{EN} = 1.25V$ schaltet der Spannungswandler ab.
    Wie im Kapitel \ref{sub:tps54331} beschrieben lässt sich die minimal erlaubt Eingangsspannung über den Spannungsteiler $R_{en1}$ und $R_{en2}$ einstellen. 
    Diese ist für 5V falsch gewählt worden.
    \item[Fehlerbehebung:] Um den Unterspannungsschutz zu deaktivieren, können die Widerstände $R_{en1} = R1$ und $R_{en2} = R5 + R7$ entfernt werden.
\end{description}

\subsection*{CP2102N Stromversorgung}
\begin{description}
    \item[Fehlerbeschreibung:] Der Mikrocontroller startet nicht richtig, wenn die Platine nur über 12V verorgt wird und USB nicht verbunden ist.
    \item[Grund:] Da der CP2102N direkt über VBUS der USB Schnittstelle versorgt wird, ist der Chip nicht aktiv wenn nur die 12V Versorgung vorhanden ist.
    Die Ausgangssignale RTS und DTR sind somit "floatend"/undefiniert und können beim Einschalten der 12V einen Start in den Download Boot-Modus triggern.
    \item[Fehlerbehebung:] Damit die Ausgänge RTS und DTR immer auf definiertem Pegel sind, muss der CP2102N dauerhaft über die 3.3V versorgt sein. 
    Dazu muss REGIN mit 3.3V verbunden werden. Da der interne 5V zu 3.3V Regulator nicht mehr genutzt wird, kann das Label "CP2102N\_VDD" \ref{fig:cp2102nfehlerkorrektur} auch direkt mit 3.3V verbunden werden.
    Auf den bestehenden Platinen kann der Fehler behoben werden indem der CP210N Chip abgelötet wird und die Leiterspur zwischen VBUS und REGIN aufgetrennt wird. (Abbildung \ref{fig:cp2102nfehlerkorrekturlayout} grün)
    Anschließend muss REGIN mit VDD verbunden werden (Lötbrücke) und mithilfe von z.B. einem Kupferdraht mit 3.3V versorgt werden. (Abbildung \ref{fig:cp2102nfehlerkorrekturlayout} blau)
\end{description}

\begin{figure}[H]
    \centering
    \begin{minipage}{0.45\textwidth}
        \centering
        \includegraphics[width=\textwidth]{figures/Platinenentwurf/KiCad_CP2102N_Fehlerkorrektur.png}
        \caption{CP2102N Fehlerkorrektur}
        \label{fig:cp2102nfehlerkorrektur}
    \end{minipage}
    \hfill
    \begin{minipage}{0.45\textwidth}
        \centering
        \includegraphics[width=\textwidth]{figures/Platinenentwurf/KiCad_CP2102N_Fehlerkorrektur_Layout.png}
        \caption{CP2102N Korrektur Layout}
        \label{fig:cp2102nfehlerkorrekturlayout}      
    \end{minipage}
\end{figure}

\subsection*{USB-C Anschluss}
\begin{description}
    \item[Fehlerbeschreibung:] Die Signalübertragung über USB-C funktioniert nur in einer Steckrichtung. 
    \item[Grund:] Die Pins B6 und B7 wurden nicht mit D+/D- auf der Platine verbunden.
    \item[Fehlerbehebung:] Da die Lötstellen an der USB Buchse sehr klein sind, ist es nur schwer möglich eine Brücke zu löten.
\end{description}

\section{Verbesserungen} \label{ch:verbesserungen_hardware}
In zukünftigen Projektarbeiten wird die Platine wahrscheinlich weiter entwickelt.
Im Folgenden sollen deshalb mögliche Verbesserungsvorschläge aufgelistet werden.
\begin{itemize}
    \item Da das Ziel des Gesamtprojektes eine Roboterformation mit möglichst vielen einzelnen Robotern ist, muss zukünftig die Ladeinfrastruktur möglichst einfach gehalten werden.
    Aus diesem Grund ist es sinnvoll die Ladeelektronik für die LiIon Zellen direkt in die Platine zu integrieren. Über USB-C Power Delivery könnten die Roboter dann sehr einfach geladen werden.
    \item Momentan ist das Main\_PCB als 4 Layer Platine ausgelegt. Um die kosten der einzelnen Platine zu senken, sollte das Layout auf 2 Layer reduziert werden.
    \item Die Pinbelegung \ref{fig:esp32pinlayout} des ESP32 zeigt, dass es kaum mehr freie Pins am Mikrocontroller gibt. Die Regelung der Motoren belegt alleine neun dieser Pins.
    Diese könnte auf einen zweiten/sekundären Mikrocontroller ausgelagert werden. Somit würden wieder mehr Pins am ESP32 nutzbar sein. 
    Außerdem wird die ESP32 Software-Architektur vereinfacht und Rechenleistung für zukünftige Software Erweiterungen frei.
\end{itemize}

% !TEX root = ../Projektdokumentation.tex
\chapter{3D-Roboterdesign}\label{ch:3D}

In this chapter, we're actually using some code!

\begin{lstlisting}[language=Python,caption={This is an example of inline listing},captionpos=b]
x = 1
if x == 1:
    # indented four spaces
    print("x is 1.")

\end{lstlisting}

You can also include listings from a file directly:



% !TEX root = ../Projektdokumentation.tex
\chapter{Systemarchitektur}\label{ch:systemueberblick}

\section{Überblick}

\begin{figure}[h]
    \centering
    \includegraphics[width=0.9\linewidth]{figures/Systemarchitektur/Systemarchitektur.png}
    \caption{Systemarchitektur}
    \label{fig:systemarchitektur}
\end{figure}

Dieses Kapitel soll einen Überblick über den Aufbau und die Kommunikation zwischen den Software Modulen in diesem Projekt geben.
Für genauere Informationen wird auf das Kapitel \ref{ch:esp32} und Kapitel \ref{ch:ros} verwiesen.
Die Abbildung \ref{fig:systemarchitektur} zeigt, dass sich das Gesamtsystem in drei Teile aufteilen lässt.
Diese unterscheiden sich durch den Ausführungsort der jeweiligen Software.

Jeder Roboter wird durch einen ESP32 Mikrocontroller gesteuert. 
Die Firmware auf den Robotern ist identisch bis auf eine Stringkonstante im Flash des Mikrocontrollers.
Dieser Name des Roboters dient als eindeutiges Identifikationsmerkmal für die oberen Softwareschichten.
Über die Wifi Schnittstelle verbindet sich der Mikrocontroller mit einem WLAN Access Point und kann dann eine TCP Verbindung 
mit einem Rechner im Netzwerk aufbauen.

Dieser TCP Server ist eine ROS Node und stellt über ein eigenes Applikationsprotokoll eine Schnittstelle zwischen ROS Topics und ESP32 bereit.
Durch den entsprechenden TCP Client auf dem ESP32, wirkt es so, als wäre der Mikrocontroller direkt in das ROS Netzwerk eingebunden.
Es werden zum Beispiel Roboterposition "'gepublisht"' und Topics wie Robotergeschwindigkeit, Zielpunkt, etc. durch den Roboter "'subscribed"'.
Der Hintergedanke zur Verwendung des ROS Frameworks ist die Auslagerung der rechenleistungsintensiven Trajektoriengenerierung 
vom Mikrocontroller auf einen leistungsstarken Rechner. Der Roboterschwarm wird somit zentral durch ROS koordiniert.
Während in diesem Projekt nur einfache Trajektorien unabhängig von den einzelnen Roboterpositionen generiert werden können, 
sollte zukünftig eine komplexe Multiroboterpfadplanung die Trajektorien generieren. 
Die ROS Nodes tauschen sich über Topics aus. 
Damit zwischen den einzelnen Roboter unterschieden werden kann, besteht der Topicname aus einem Namespace und der Topicbezeichung: 
\begin{center}
    \verb|/namespace/topicbezeichnung|
\end{center}
\verb|namespace| ist die Identifikationskonstante im Flash des ESP32 und wurde in diesem Projekt auf \verb|robot_| plus Nummerierung gesetzt.
Damit der ESP32 mit den Nachrichten auf den Topics umgehen kann, wurde für alle verwendeten ROS-Nachrichtentypen ein entsprechender Datentyp in der ESP32 Software implementiert.
Tabelle \ref{tab:ros_topics} zeigt eine Auflistung aller vom Roboter genutzten Topics.

\begin{table}[H]
    \centering
    \begin{tabular}{c|c|c|}
        Topic-Name   & ROS-Nachrichtentyp                 & ESP32 Datentyp                      \\ \hline
        /goal\_point & geometry\_msgs::Point              & ros\_msgs::Point2D                  \\
        /vel         & geometry\_msgs::Twist              & ros\_msgs::Twist2D                  \\
        /trajectory  & trajecgenerator::c\_trajec\_vector & ros\_msgs::Trajectory               \\
        /pose2D      & geometry\_msgs::Pose2D             & ros\_msgs::Pose2D                   \\
        /cmd\_vel    & geometry\_msgs::Twist              & ros\_msgs::Twist2D                  \\
        /pose        & geometry\_msgs::Pose2D             & ros\_msgs::Pose2D                   \\
        /start\_log  & std\_msgs::String                  & ros\_msgs::String                   \\
        /data\_log   & std\_msgs::String                  & ros\_msgs::String                   \\
        /led         & std\_msgs::String                  & std\_msgs::String                  
    \end{tabular}
    \vspace{0.5cm}

    \begin{tabular}{c|c|c|c}
        Topic-Name   & sub & pub & Beschreibung                            \\ \hline
        /goal\_point & X   &     & Anfahren eines bestimmten Punktes       \\
        /vel         & X   &     & Steuerung der Robotergeschwindigkeit    \\
        /trajectory  & X   &     & Abfahren der übertragenen Trajektorie   \\
        /pose2D      &     & X   & Istposition des Roboters                \\
        /cmd\_vel    &     & X   & Stellgeschwindigkeit (Simulationsmodus) \\
        /pose        & X   &     & Istposition Roboter (Simulationsmodus)  \\
        /start\_log  & X   &     & Starten des Datenloggings               \\
        /data\_log   &     & X   & Logging Daten                           \\
        /led         & X   &     & LED-Farben/Muster                      
    \end{tabular}

    \caption{Roboter ROS Topics}
    \label{tab:ros_topics}
\end{table}

Damit das Gesamtsystem über eine benutzerfreundliche Schnittstelle einfach bedient werden kann, wurde eine Web-App entwickelt.
Die Web-App läuft über den Browser auf verschiedenen Endgeräten und kommuniziert über das Websocket Protokoll mit einer weiteren ROS Node.
Ursprünglich war geplant die Trajektoriengenerierung über ROS Service Calls aus der Web-App zu steuern. Dies konnte allerdings nicht mehr umgesetzt werden.
Das Gleiche gilt für das Anzeigen der Roboterpose und Setzen von Zielpunkten.
Zum derzeitigen Stand der Projektarbeit ist es möglich die Robotergeschwindigkeit und LED Muster/Farben über das Web Interface zu steuern.

\section{Applikationsprotokoll}
Für die Kommunikation zwischen Mikrocontroller und Host Rechner wurde ein eigenes Applikationsprotokoll entwickelt. 
Dieses soll als virtuelles Interface zwischen ROS und dem ESP32 wirken, damit diese mithilfe von ROS Topics miteinander kommunizieren können.
Für den Entwickler soll sich somit das "`Subscriben"' und "`Advertisen"' von Topics auf dem Mikrocontroller anfühlen wie in einer normalen ROS Applikation.
Eine wichtige Hauptaufgabe des Protokolls im Projekt ist allerdings die Übertragung der Trajektorien über die Topic "`/robot\_*/trajectory"'.

Eine wichtige Überlegung für den Aufbau des Protokolls war die Entscheidung zwischen UDP oder TCP als Transportunterschicht.
Das User Datagram Protokoll ist ein verbindungsloses Protokoll. Es garantiert somit nicht die sichere, verlustfreie Übertragung der Nutzdaten.
Da es zu keinen Neu-Übertragungen kommt, hat es geringe Latenzzeiten und eignet sich somit super zur Übertragung von Echtezeitdaten wie die Position des Roboters.
Die Hauptaufgabe des Protokolls in diesem Projekt ist die Übertragung von Kilobyte gro{\ss}en Trajektorien. 
Diese erfordert eine zuverlässige Verbindung ohne Datenverluste. 
Des Weiteren ist eine Übertragung der Daten in Echtzeit gut, aber nicht unbedingt erforderlich.
Das Transmission Control Protocol baut eine virtuelle Verbindung zwischen Server und Client auf und 
garantiert somit eine verlustfreie Übertragung der Daten. 
Aus diesem Grund wurde TCP als Transportunterschicht für das eigene Protokoll verwendet.
In diesem Kapitel wird der Aufbau des Protokolls beschrieben. 
Für die genaue Implementierung der Server und Client Seite wird auf die Kapitel \ref{ch:rosbridgeclient} und \ref{ch:rosbridgeserver} verwiesen. 

\subsection{Datenpakete}
Die Kommunikation mit dem Applikationsprotokoll ist paketbasierend. 
Diese Pakete müssen von der Software im Server oder Client aus dem TCP Datenstrom gefiltert werden.
Das Protokoll definiert insgesamt 5 verschiedene Pakete:

\begin{table}[H]
    \centering
    \begin{tabular}{l|l}
    ID & Paketname              \\ \hline
    0x01  & Initialisierungspaket \\
    0x02  & Advertise Paket        \\
    0x03  & Subscribe Paket        \\
    0x04  & Keep-Alive Paket       \\
    0x05  & Publish Paket         
    \end{tabular}
    \caption{Applikationsprotokoll Paketarten}
\end{table}

Jedes Paket beginnt mit der jeweiligen Identifikationsnummer. Der restliche Aufbau ist paketabhängig.
Im Folgenden werden der Aufbau und der Nutzen der unterschiedlichen Pakete erklärt.

\subsubsection*{Initialisierungspaket}
Das Initialisierungspaket ist das erste Paket, das nach Aufbau der TCP Verbindung vom Client an den Server geschickt wird.
Dabei überträgt es den Roboternamen, der von der ROS Server Node als Topic Namespace genutzt wird.
Das Initialisierungspaket wird nur vom Client an den Server geschickt werden.

\begin{table}[H]
    \centering
    \begin{tabular}{|c|c|c|}
    \hline
    0x01   & Robotername & '\textbackslash{}0' \\ \hline
    1 Byte & x Bytes     & 1 Byte              \\ \hline
    \end{tabular}
    \caption{Initialisierungspaket}
\end{table}

\subsubsection*{Advertise Paket}
Das Advertise Paket entspricht dem Aufruf der ROSCPP Methode advertise().
Dieses Paket kann nur vom Client an den Server geschickt werden.
Dabei fordert dieser den Server dazu auf die übermittelte Topic mit dem übermittelten Nachrichtentyp in ROS zu "`advertisen"'.
Erst nach dem Advertise Paket dürfen Nachrichten von der entsprechenden Topic mit dem Publish Paket verschickt werden.

\begin{table}[H]
    \centering
    \begin{tabular}{|c|c|c|c|c|}
    \hline
    0x02   & Topic Name    & '\textbackslash{}0' & Nachrichtentyp & '\textbackslash{}0' \\ \hline
    1 Byte & max. 32 Bytes & 1 Byte              & max. 32 Bytes  & 1 Byte              \\ \hline
    \end{tabular}
    \caption{Advertise Paket}
\end{table}

\subsubsection*{Subscribe Paket}
Ähnlich wie das Advertise Paket, teilt das Subscribe Paket dem Server mit, dass der Client eine Topic mit dem entsprechenden Nachrichtentyp "`subscriben"' möchte.
Auch das Subscribe Paket wird nur vom Client an den Server geschickt.
Der Server darf erst, nachdem er ein Subscribe Paket zu einer Topic empfangen hat, Publish Pakete an den Client weiterleiten.

\begin{table}[H]
    \centering
    \begin{tabular}{|c|c|c|c|c|}
    \hline
    0x03   & Topic Name    & '\textbackslash{}0' & Nachrichtentyp & '\textbackslash{}0' \\ \hline
    1 Byte & max. 32 Bytes & 1 Byte              & max. 32 Bytes  & 1 Byte              \\ \hline
    \end{tabular}
    \caption{Subscribe Paket}
\end{table}

\subsubsection*{Keep-Alive Paket}
Das Keep-Alive Paket wird alle $500ms$ vom Client an den Server und vom Server an den Client geschickt. 
Empfängt eine der beiden Seiten kein Keep-Alive für $3000ms$, wird ein Verbindungsabbruch festgestellt.
Dies ist vor allem notwendig, wenn keine Nutzdaten übertragen werden und zum Beispiel der Client abstürzt.
Der Server wird nach drei Sekunden feststellen, dass keine Keep Alive Pakete vom Client kommen und die TCP Verbindung schlie{\ss}en.
Zusätzlich wird ein Zeitstempel des jeweiligen Senders an den Empfänger übertragen. 
Dieser kann z.B. zur Zeitsynchronisation zwischen ROS und Roboter verwendet werden.

\begin{table}[H]
    \centering
    \begin{tabular}{|c|c|}
    \hline
    0x04   & Zeitstempel in $\mu s$                 \\ \hline
    1 Byte & 8 Bytes (uint64\_t) \\ \hline
    \end{tabular}
    \caption{Keep-Alive Paket}
\end{table}

\subsubsection*{Publish Paket}
Das Publish Paket dient der eigentlichen Übertragung von Nutzdaten vom Client zum Server und vom Server zum Client.
Durch das Feld "`Topic Name"' können die Daten einer durch das Subscribe- und Advertise-Paket initialisierten Topic zugeordnet werden.
Der Datenteil des Pakets unterscheidet zwischen Array- und Strukturdaten.

Strukturdaten serialisieren den ROS Nachrichtentyp der jeweiligen Topic.
Auf das Serialisieren und Deserialisieren der Nachrichten wird genauer in den Kapiteln \ref{ch:esp32} und \ref{ch:ros} eingegangen.

\begin{table}[H]
    \centering
    \begin{tabular}{|c|c|c|}
    \hline
    0x05   & Topic Name    & Strukturdaten \\ \hline
    1 Byte & max. 32 Bytes & x Bytes       \\ \hline
    \end{tabular}
    \caption{Publish Paket Strukturdaten}
\end{table}

Arraydaten werden in diesem Projekt für die Übertragung der Trajektorien und zur Übertragung des ROS Nachrichtentyp std\_msgs::String verwendet.
Nach dem Topic Namen folgt ein Feld für die jeweilige Arraylänge. Dieses beinhaltet die Grö{\ss}e des Arrays in Bytes.

\begin{table}[H]
    \centering
    \begin{tabular}{|c|c|c|c|}
    \hline
    0x05   & Topic Name    & Arraylänge in Bytes & Arraydaten \\ \hline
    1 Byte & max. 32 Bytes & 4 Bytes (int32\_t)  & x Bytes       \\ \hline
    \end{tabular}
    \caption{Publish Paket Arraydaten}
\end{table}

\subsection{Anwendungsbeispiel}

\begin{figure}[H]
    \centering
    \includegraphics[width=0.7\textwidth]{figures/Systemarchitektur/Applikationsprotokoll_Verbindungsaufbau.png}
    \caption{Applikationsprotokoll Anwendungsbeispiel}
    \label{fig:applikationsprotokoll_anwendungsbeispiel}
\end{figure}

Die Abbildung \ref{fig:applikationsprotokoll_anwendungsbeispiel} zeigt eine typische Kommunikation zwischen Client und Server.
Der Client startet die Verbindung mit dem Initialisierungspaket und teilt dem Server mit welche Topics er "`subscriben"' und "`advertisen"' will.
Alle $500ms$ tauschen beide Seiten ihr Keep-Alive Paket mit dem jeweiligen Zeitstempel aus. 
Daten zu den Topics werden mithilfe des Publish Pakets vom Client zum Server oder vom Server zum Client geschickt.
Ein Verbindungsabbruch würde durch ein fehlendes Keep-Alive bemerkt werden. 
Das Verhalten beider Seiten in diesem Fall wird in den Kapiteln \ref{ch:rosbridgeclient} und \ref{ch:ch:rosbridgeserver} beschrieben.




% !TEX root = ../Projektdokumentation.tex
\chapter{Softwarearchitektur auf dem ESP32}\label{ch:esp32}

\section{ESP-IDF}
Das Espressife IOT Development Framework kurz ESP-IDF ist das offizielle Tool zur Einrichtung und Programmierung des ESP32.
Das Open Source Framework beinhaltet unter anderem Bibliotheken zur Hardwareabstraktion, ein Konfigurationssystem, es integriert das Echtzeitbetriebssystem FreeRTOS und 
ermöglicht über Automatisierungsskripte Compilieren und Flashen von Binaries.
Der ESP32 lässt sich mit C oder C++ programmieren. 
Compilieren und Linken der Software wird von dem Build System CMake übernommen.
Die Installation der ESP-IDF wird im "Programming Guide" \cite{ESP-IDFProgrammingGuide} von Espressife beschrieben.
Das Projekt wurde mit der Version v4.4 und v5.0 getestet.

Damit die Peripherie des ESP32 nicht über Register angesteuert werden muss, bietet Espressife eine Hardwareabstraktionsschicht.
Diese erleichtert die Initialisierung und Steuerung von Hardwaremodulen wie zum Beispiel GPIO, UART, I2C, etc. über Funktionsaufrufe.

Das Konfigurationssystem von Espressife basiert auf KConfig. 
Mit dem Befehl \emph{idf.py menuconfig} wird ein Konfigurationsmenü gestartet.
Hier können z.b. hardwarespezifische Einstellungen für den ESP32 gemacht werden. 
Es ist allerdings auch möglich eigene Einstellungspunkte über die KConfig Syntax zu erstellen.
Im Projekt sind zum Beispiel Einstellungspunkte für Wifi SSID, Password und Server IP Addresse definiert.
Die Konfigurationen werden in der Datei sdkconfig gespeichert. Beim Bauen wird diese in eine Header Datei umgewandelt.
Durch Inkludieren der Datei sdkconfig.h werden die Einstellungen über "defines" in die C/C++ Programme eingebunden.

Die ESP-IDF ermöglicht die Unterteilung der Software in Komponenten. Dabei werden die Sourcen in einzelne Ordner unterteilt.
Dies erhöht die Projektstruktur und Wiederverwendbarkeit des Codes. 
Jedes neu erstellte ESP-IDF Projekt enthält eine standard "main" Komponente diese beinhaltet die main() Funktion des Programms.
Neue Komponenten können mit dem Befehl \emph{idf.py -C components create-component <Komponentenname>} erstellt werden.

Mit Hilfe der von Espressife bereitgestellten Automatisierungsskripte können zum Beispiel Binaries gebaut und geflasht werden.
Die Skripte werden mit Befehlen für die Kommandozeile gestartet.
Die folgende Auflistung enthält die für dieses Projekt wichtigsten Befehle:
\begin{itemize}
    \item \emph{idf.py build} Compiliert und Linkt alle Binaries
    \item \emph{idf.py flash} Flasht die Binaries auf den Mikrocontroller
    \item \emph{idf.py monitor} Zeigt die Ausgabe der seriellen Schnittstelle
    \item \emph{idf.py menuconfig} Öffnet das Konfigurationsmenü
    \item \emph{idf.py -C components create-component <Komponentenname>} Erstellt neue Komponente
\end{itemize}

FreeRTOS ist ein Echtzeitbetriebssystem für Mikrocontroller. 
Ähnlich wie ein normales Betriebssystem arbeitet es mit Tasks/Prozessen, so dass mehrere Programmzweige parallel ablaufen können.
Die ESP-IDF Software portiert FreeRTOS auf den ESP32.
Neben den Tasks implementiert FreeRTOS auch andere Konzepte wie "Queues", "Message Buffer", "Semaphors" und "Mutexes".
Diese dienen unter anderem der Synchronisierung von Prozessen und verhindern "Race Conditions" beim Zugriff auf geteilte Ressourcen.

Mit der Funktion \verb|static inline BaseType_t xTaskCreate(TaskFunction_t pvTaskCode, const char *const pcName, const uint32_t usStackDepth, void *const pvParameters, UBaseType_t uxPriority, TaskHandle_t *const pxCreatedTask)|
kann ein neuer Prozess erzeugt werden.
\begin{itemize}
    \item Der Pointer \verb|pvTaskCode| zeigt auf die Funktion die den Code des jeweiligen Prozesses beinhaltet. Diese Funktion muss eine Endlosschleife implementieren und darf nicht enden.
    \item \verb|pcName| ist der Name der FreeRTOS Task
    \item \verb|usStackDepth| Jeder Prozess hat einen eigenen Stack. Die Größe wird beim erstellen der Task statisch festgelegt und muss somit vom Programmierer geeignet dimensioniert werden.
    Große lokale Arrays oder viele verschachtelte Funktionsaufrufe im Prozess benötigen einen größeren Stack.
    \item Über \verb|pvParameters| können der Task Daten übergeben werden.
    \item \verb|uxPriority| Jeder Prozess hat eine Priorität. Eine höhere Zahl bedeutet eine höhere Priorität. Prozesse mit höhere Priorität werden vom FreeRTOS Scheduler bevorzugt.
    \item Das Objekt \verb|pxCreatedTask| repräsentiert den Prozess und kann z.B. zum löschen oder stoppen des Prozesses genutzt werden.
\end{itemize}

Die Tabelle \ref{tab:freertos_task} gibt einen Überblick über alle im ESP32 Code existierenden Tasks:

\begin{table}[H]
    \begin{tabular}{c|c|c|c|c}
    Name                           & Komponente         & Klasse           & Priorität & \begin{tabular}[c]{@{}c@{}}Stackgröße\\ (Byte)\end{tabular} \\ \hline
    app\_main()                    & main               &                  & 0         &                                                             \\
    \_communication\_handler()     & RosBridgeClient    & NodeHandle       & 5         & 8192                                                        \\
    \_kalman\_filter\_loop\_task() & SensorPose         & KalmanFilter     & 9         & 8192                                                        \\
    \_motor\_control\_loop\_task() & MotorController    & MotorController  & 10        & 2048                                                        \\
    \_uart\_read\_data\_task()     & SensorPose         & Marvelmind       & 5         & 2048                                                        \\
    \_control\_loop\_task()        & PositionController & ControllerMaster & 8         & 8192                                                        \\
    \_data\_logger\_task()         & DataLogger         & DataLogger       & 3         & 2048                                                        \\
    led\_handler()                 & LedStrip           & LedStrip         & 2         & 4096                                                       
    \end{tabular}
    \caption{FreeRTOS Tasks}
    \label{tab:freertos_task}
\end{table}


\section{Überblick}

\begin{figure}[H]
    \centering
    \includegraphics[width=0.75\textwidth]{figures/Softwarearchitektur_ESP32/ESP32_Architektur.png}
    \caption{Überblick ESP32 Architektur}
    \label{fig:esp32architektur_ueberblick}
\end{figure}

Abbildung \ref{fig:esp32architektur_ueberblick} zeigt alle Systemkomponenten des Projekts. Diese Kommunizieren größtenteils über
Funktionsaufrufe. Die Pfeilrichtung der dargestellten Aufrufe zeigt immer von der Aufrufenden zur aufgerufenen Komponente.
Der Programmeinstieg befindet sich in der main Komponente. Hier werden alle anderen Komponenten initialisiert.
In RosBridgeClient befindet sich die Ablaufsteuerung des Applikationsprotokolls zum Topicaustausch mit dem ROS-System.
Die Daten hierfür werden über Socket empfangen und gesendet. 
Empfängt der RosBridgeClient Daten auf der Geschwindigkeits, Trajektorien oder Zielpunkt Topic, wird über einen Callback die Komponente 
StateMachine informiert. Die Zustandsmaschine versetzt den Roboter basierend auf dem aktuellen Zustand und dem jeweiligen Aufruf in einen neuen Zustand und 
aktiviert z.B. für einen empfangenen Zielpunkt oder einer Trajektorie den PositionController. Diese Komponente regelt die Position des Roboters und kann somit 
einer Trajektorie folgen oder einen Zielpunkt anfahren. Hierfür besorgt sie sich die Roboterpose aus der Komponente SensorPose und gibt zyklisch einen neuen Sollgeschwindigkeitsvektor
an OutputVelocity weiter. OutputVelocity transformiert den Geschwindigkeitsvektor in Solldrehzahlen für die beiden Motoren. Diese werden dann in MotorController ausgeregelt.
Optional kann der Roboter auch in einen Simulationsmodus versetzt werden. Hierbei wird der Sollgeschwindigkeitsvektor an ROS gesendet. Außerdem erhält die Komponente SensorPose
eine Pose von ROS und nicht der Sensorik auf dem Roboter.
Die Komponente Wifi übernimmt den Verbindungsauf- und abbau mit dem verwendeten Wlan Access Point.
LedStrip steurt das Led-Array auf dem Roboter. Mithilfe von DataLogger können Programmlogs erstellt und in das ROS-System gesendet werden.
RosMsgs definiert die Datentypen der Topics für die Kommunikation mit ROS. 
Die Komponente esp-dsp ist eine von Espressife bereitgestellte Bibliothek zur digitalen Signalverarbeitung und stellt für dieses Projekt vor allem 
Funktionen zur Matrizenrechnung bereit.

\section{Wifi}
Die Komponente Wifi konfiguriert den ESP32 als WLAN Station und steuert Verbindungsaufbau sowie das Verhalten bei Verbindungsabbruch.
Dies wird in den Methoden der gleichnamigen Singleton Klasse umgesetzt. Zunächst wird die Wifi und ESP-NETIF Bibliothek von Espressife initialisiert und 
anschließend für bestimmte Wifi Events die Event Loop Library konfiguriert. Dabei handelt es sich um eine Zustandsmaschine, die bei Events wie z.B.
"Verbindung hergestellt" oder "Verbindung abgebrochen" eine Callback-Funktion im User Code ausführt. Hier kann das entsprechende Verhalten für die Events vom Programmierer gesteuert werden.
Nach dem Initialisieren und Starten der Wifi Klasse über \verb|init()| und \verb|begin()| interagiert die Komponente nicht mehr mit anderen Komponenten.
Das steuern der Event Zustandsmaschine übernimmt ein interner Prozess der Espressife Bibliotheken. 
Die Komponente Wifi beinhaltet drei Einstellungspunkte im KConfig Menü:

\begin{itemize}
    \item \verb|Wifi SSID| SSID des Wlan Routers
    \item \verb|WIFI Password| Passwort des Wlan Routers
    \item \verb|WIFI_MAX_RETRY| Anzahl der Verbindungsversuche bei fehlgeschlagenem Verbindungsaufbau
\end{itemize}


\section{Socket}
Die Komponente Socket dient als Abstraktionsschicht der POSIX Socket API, die von Espressife zum Aufbau einer TCP Verbindung genutzt wird.
Der ESP32 agiert dabei als Client und Verbindet sich mit einer ROS-Node die den Server darstellt. 
Eine Verbindung wird über ein Objekt der Klasse Socket gekapselt. Über die folgenden Methoden kann eine Verbindung auf- und abgebaut werden, sowie Daten empfangen und versendet werden:

\begin{itemize}
    \item \verb|Socket(int port, std::string ip_addr)| Dem Konstruktor müssen der Port sowie die IP-Adresse des Servers übergeben werden.
    \item Über \verb|void connect_socket()| wird eine Verbindung mit dem Server aufgebaut. Die Methode blockiert solange bis erfolgreich eine Verbindung aufgebaut werden konnte.
    Ist dies nicht der Fall wird versucht durch Abbau und erneutem Aufbau eine Verbindungs herzustellen. Des Weiteren konfiguriert die Methode die POSIX Socket API als "O\_NONBLOCK".
    Dies verhindert ein Blockieren beim Aufruf der POSIX Funktionen \verb|recv()| und \verb|send()|, da das Verhalten beim Senden und Empfangen durch die eigene Socket Klasse geregelt wird.
    \item \verb|void disconnect_socket()| schließt die TCP-Verbindung.
    \item \verb|int socket_receive(uint8_t* rx_buffer, int recv_bytes)| empfängt die mit \verb|recv_bytes| übergebene Anzahl an Bytes aus dem internen TCP-Puffer. Die Methode blockiert solange bis
    alle Bytes vollständig empfangen wurden oder ein Fehler beim Emfangen auftritt. Es wird die Anzahl empfangener Bytes oder bei einem Fehler das Makro \verb|SOCKET_FAIL| zurück gegeben.
    \item \verb|int socket_receive_string(std::string& rx_string, int max_bytes)| dient zum Empfangen von mit \verb|'\0'| beendeten Zeichenketten. Die Methode blockiert solange bis eine vollständige Zeichenkette
    empfangen wurde, \verb|max_bytes| aus dem Puffer entnommen wurden oder ein Fehler beim Empfangen auftritt. Es wird die Anzahl empfangener Bytes oder bei einem Fehler das Makro \verb|SOCKET_FAIL| zurück gegeben.
    \item \verb|int socket_receive_nonblock(uint8_t* rx_buffer, int recv_bytes)| empfängt die mit \verb|recv_bytes| übergebene Anzahl an Bytes aus dem internen TCP-Puffer. 
    Falls nicht genügend Bytes vorhanden sind oder ein Fehler beim Empfangen auftritt, endet die Methode sofort. Es wird die Anzahl empfangener Bytes oder bei einem Fehler das Makro \verb|SOCKET_FAIL| zurück gegeben.
    \item Mit \verb|int socket_send(uint8_t const* tx_buffer, int buffer_len)| werden \verb|buffer_len| Bytes an den Server gesendet. 
    Paralleles senden durch mehrere Prozesse könnte zu Überlagerung der Daten führen. Aus diesem Grund verhindert ein Mutex paralleles senden. 
    Rufen mehrere Prozesse gleichzeitig \verb|socket_send()| auf, darf der Erste senden, für die Anderen endet die Methode. Darf ein Prozess senden, wird solange blockiert bis alle Bytes gesendet wurden oder ein Fehler auftritt.
    Es wird die Anzahl gesendeter Bytes oder bei einem Fehler das Makro \verb|SOCKET_FAIL| zurück gegeben.
\end{itemize}


\section{RosBridgeClient}

\begin{figure}[H]
    \centering
    \includegraphics[width=0.75\textwidth]{figures/Softwarearchitektur_ESP32/RosBridgeClientClass.png}
    \caption{Klassendiagramm RosBridgeClient}
    \label{fig:ros_bridge_client_class}
\end{figure}
Die Komponente RosBridgeClient steuert das Applikationsprotokolls zur Kommunikation mit ROS. 
Ein Ziel des Applikationsprotokolls ist es die TCP Kommunikation zu abstrahieren. Damit wirkt es als wäre der ESP32 wie eine ROS-Node in das System eingebunden. 
Aus diesem Grund sind Klassen- und Funktionsnamen, sowie die Funktionsweise der Programmierschnittstelle stark an ROS orientiert.
Das Klassendiagramm in Abbildung \ref{fig:ros_bridge_client_class} zeigt den Aufbau von RosBridgeClient. 

\subsection{Publisher und SubscriberImpl}
Für jede im Protokoll verwendete Topic wird ein Objekt der Klasse \verb|Publisher| oder \verb|SubscriberImpl| erstellt.
Über Objekte der Klasse \verb|Publisher| kann der Programmierer Nachrichten auf der jeweiligen Topic verschicken. 
Objekte vom Typ \verb|SubscriberImpl| werden nur intern von NodeHandle zum empfangen von Topics verwendet.
Empfängt SubscribeImpl eine Nachricht wird das restliche Programm über eine Callback Funktion informiert.
Beide Klassen sind parametrierbar mit dem Nachrichtentyp der jeweiligen Topic.
Die Schnittstellen \verb|PublisherInterface| und \verb|Subscriber| dienen zum speichern der verschieden parametrierten Objekte in NodeHandle.
\verb|Publisher| oder \verb|SubscriberImpl| haben beide einen privaten Konstruktor. Es ist somit nur möglich Objekte über die Methoden \verb|advertise()| und \verb|subscribe()| der Freundesklasse NodeHandle zu erstellen.

Im Folgenden wird auf die Methoden der Klasse \verb|Publisher| genauer eingegangen:
\begin{itemize}
    \item \verb|Publisher(std::string const& topic, Socket& sock)| ist der Konstruktor der Klasse. 
    \item \verb|void advertise()| sendet das Advertise Paket über Socket an den ROS-Server.
    Die Methode wird von der namensgleichen Funktion aus NodeHandle oder beim Neustart des Protokolls aufgerufen.
    \item Mit \verb|void publish(T const& msg)| kann der Programmierer Nachrichten auf der jeweiligen Topic an den ROS-Server versenden.
    Der Parameter \verb|T| ist der Nachrichtentyp des Publish Pakets.
    \item Die Methoden \verb|void block_publishing()| und \verb|void unblock_publishing()| werden beim Neustart des Protokolls genutzt, um Publish Pakete aus anderen Teilen/Prozessen des Programms zu verhinden.
\end{itemize}

Im Folgenden wird auf die Methoden der Klasse \verb|SubscriberImpl| eingegangen: 
\begin{itemize}
    \item \verb|SubscriberImpl(std::string const& topic, Socket& sock, std::function<void(std::shared_ptr<T> ros_msg)> callback_function)| ist der Konstruktur der Klasse.
    Dem Konstruktor wird ein Zeiger auf eine Callback-Funktion übergeben. Diese wird nach dem Empfangen einer Nachricht auf der jeweiligen Topic ausgeführt.
    \item \verb|void subscribe()| sendet das Subscribe Paket an den ROS-Server. Die Methode wird von der namensgleichen Methode aus NodeHandle oder beim Neustart des Protokolls ausgeführt.
    \item \verb|void recvMessage()| interpretiert den Datenteil eines empfangenen Publish-Pakets. 
    Die Methode unterscheidet, ob es sich um Arraydaten oder reine Strukturdaten handelt.
    Anschließend werden die Daten deserialisiert und die enstprechende Callback-Funktion aufgerufen.
    \item \verb|bool compareTopic(std::string const& topic)| vergleicht den übergebenen Topicnamen mit dem Internen.
    Die Methode wird von NodeHandle verwendet, um für ein empfangenes Publish-Paket den richtigen Subscriber zu finden.
\end{itemize}


\subsection{NodeHandle}
NodeHandle ist eine Singleton Klasse. Es existiert somit nur ein NodeHandle Objekt im Programm. Dieses wird beim Aufruf von init() einmalig erstellt.
Parallel dazu wird das Initialisierungspaket an den ROS-Server über die Methode \verb|_send_init()| versendet. 
Anschließend startet der Hauptprozess \verb|_communication_handler|. Dieser läuft parallel zum restlichen Programm.
Die in der Abbildung \ref{fig:node_handle_ablauf} gezeigte Schleife wird vom Hauptprozess alle $50ms$ ausgeführt.
Dabei wird zyklisch alle $500ms$ ein Keep-Alive Paket versendet, Daten aus dem Socket Eingangspuffer ausgewertet 
und überprüft ob das letzte empfangene Keep-Alive Paket älter als $3000ms$ ist. 
Falls das letzte Keep-Alive älter ist oder beim empfangen und interpretieren der Daten etwas schief gelaufen ist, wird das Protokoll neu gestartet.

\begin{figure}[H]
    \centering
    \includegraphics[width=0.5\textwidth]{figures/Softwarearchitektur_ESP32/NodeHandleAblauf.png}
    \caption{Ablaufdiagramm NodeHandle}
    \label{fig:node_handle_ablauf}
\end{figure}

Im Folgenden wird genauer auf die Methoden aus NodeHandle eingegangen:
\begin{itemize}
    \item \verb|static NodeHandle& init(std::string const& ros_namespace, Socket& sock)| erzeugt beim ersten Aufruf einmalig ein NodeHandle Objekt.
    Der String \verb|ros_namespace| ist der Robotername, der über das Initialisierungspaket an den ROS-Server gesendet wird.
    \item Mit \verb|Publisher<T>& advertise(std::string const& topic)| wird ein neues Publisher Objekt erzeugt. 
    Dieses wird als Zeiger in NodeHandle abgespeichert. Außerdem wird ein Advertise Paket an den ROS Server gesendet. 
    \item Mit \verb|void subscribe(std::string const& topic, std::function<void(std::shared_ptr<T> ros_msg)> callback_function)|
    wird ein neues SubscriberImpl Objekt erzeugt. Dieses wird als Zeiger in NodeHandle abgespeichert. Außerdem wird ein Subscribe Paket an den ROS Server gesendet.
    \item Der Methode \verb|void registerConnectionLostCallback(std::function<void()> connection_lost_callback)| wird ein Funktionszeiger übergeben. 
    Diese Funktion wird ausgeführt wenn es zu einem Fehler kommt und das Protokoll neu gestartet wird. 
    Dies kann z.B. dazu verwendet werden den Roboter bei Verbindungsverlust zu stoppen.
    \item Im Konstruktur \verb|NodeHandle(std::string const& ros_namespace, Socket& sock)| wird über \verb|_send_init()| das Initialisierungspaket versendet und
    es wird der Hauptprozess \verb|_communication_handler()| erstellt.
    \item \verb|static void _communication_handler(void* arg)| ist der Hauptprozess von NodeHandle.
    \item \verb|int _send_init()| sendet das Initialisierungspaket an den ROS-Server.
    \item \verb|void _send_keep_alive()| sendet das Keep-Alive Paket an den ROS-Server.
    \item \verb|int _interpret_receive()| interpretiert die empfangenen Daten aus Socket. Hierfür wird zunächst basierend auf dem ersten Message ID Byte entschieden,
    ob ein KeepAlive oder ein Publish Paket empfangen wurde. Falls ein Keep-Alive Paket empfangen wurde, wird der übertragene Zeitstempel aus dem Socket Puffer entnommen und abgespeichert.
    Falls ein PublishPaket empfangen wurde, wird basierend auf dem übertragenen Topicnamen mithilfe der Methode \verb|_getSubscriber()| das richtige Subscriber Objekt gesucht und mit \verb|recvMessage()| der Datenteil interpretiert.
    Läuft beim empfangen oder interpretieren etwas schief gibt die Methode das Makro \verb|SOCKET_FAIL| zurück.
    \item \verb|Subscriber* _getSubscriber(std::string const& topic)| gibt das jeweilige Subscriber Objekt zu der Topic zurück.
    \item \verb|void _restart_protocol()| wird dazu verwendet das Protokoll neu zu starten. Bei einem Neustart wird zunächst das Publishen für alle Publisherobjekte blockiert,
    alle registrierten "ConnectionLost" Callbackfunktionen ausgeführt, die TCP Verbindung über disconnect und connect neu aufgebaut, ein Initialisierungspaket versendet und für alle Publisher und Subscriber
    ein Advertise bzw. Subscribe Paket an den ROS Server versendet. Zuletzt wird die Publishblockierung für alle Publisher aufgehoben.
\end{itemize}

\section{RosMsgs}
Die Komponente RosMsgs definiert die Nachrichtentypen mit denen das Applikationsprotokoll kommuniziert. Diese Nachrichtentypen orientieren sich stark an den entsprechenden Typen aus ROS.
Die Nachrichtenklassen werden dem C++ Namensraum \verb|ros_msgs| untergeordnet. Ein zweiter Namensraum \verb|ros_msgs_lw| beinhaltet namensgleiche Klassen.
Wobei hier ausschließlich mit float im Gegensatz zu double bei \verb|ros_msgs| gearbeitet wird. Da die Floating Point Unit des ESP32 nur den float Datentyp unterstützt, 
dient \verb|ros_msgs_lw| für Berechnungen und \verb|ros_msgs| ausschließlich zur Kommunikation mit dem ROS Server. 
Im Applikationsprotokoll gibt es Stuktur- und Arraynachrichtentypen. Der Aufbau dieser Nachrichten ist durch die Klassen in \verb|ros_msgs| vorgegeben.  Identische Klassen extistieren auch in den Sourcen des ROS-Servers.
Um einem Publisher oder Subscriber aus der Komponente RosBridgeClient einen Nachrichtentyp zu zuweisen, werden diese mit den Klassen aus \verb|ros_msgs| parametriet. 
Aus dem Grund müssen alle Nachrichtenklassen die folgenden Methoden implementieren:

\begin{itemize}
    \item \verb|size_t getSize()| gibt die Größe der Daten der entsprechenden Klasse in Bytes zurück.
    Für Arraynachrichtentypen gibt die Methode null zurück, wenn dem Array noch keine Daten zugewiesen wurden.
    \item Mit \verb|void allocateMemory(int32_t msg_len)| wird Speicherplatz für die Daten bei Arraynachrichtentypen reserviert.
    Für Stukturdatentypen muss nur ein leerer Methodenrumpf implementiert sein.
    \item \verb|std::string getMsgType()| gibt den Namen des entsprechenden ROS Nachrichtentyps zurück.
    \item \verb|void serialize(uint8_t* buffer)| wird von der Klasse Publisher aus RosBridgeClient zur Serialisierung der Daten verwendet.
    \item \verb|void deserialize(uint8_t* buffer)| wird von der Klasse Subscriber aus RosBridgeClient zur Deserialisierung der Daten verwendet.
\end{itemize}


\section{MotorController}

\begin{figure}[H]
    \centering
    \includegraphics[width=0.6\textwidth]{figures/Softwarearchitektur_ESP32/MotorController_Class.png}
    \caption{Klassendiagramm MotorController}
    \label{fig:motor_controller_class}
\end{figure}

Die Komponente MotorController steuert die beiden Motoren des Roboters. Die Drehzahl der Motoren wird über einen PI-Regler geregelt.
Das Stellsginal ist hierbei die Pulsbreite der Ausgangssignale für den Motortreiber. In der Software haben diese einen Wertebereich von -100.0 bis 100.0 Prozent.
Ein Überschreiten des Wertebereichs wird durch eine Stellbegrenzung mit Begrenzungsbeobachter verhindert.
Die beiden Regelkreise laufen mit einer Widerholungsrate von $1000Hz$. Des Weiteren besitzt die Komponente einige Einstellungspunkte im KConfig Menü. 
Es können die Pinbelegungen für die Signale xIN1, xIN2, Enable des Motortreibers DRV8841, die Hallsensoreingängen und die PWM Frequenz des Stellsignals verändert werden.
Die Standardeinstellung passt allerdings schon zur Pinbelegung der Platine.
Die Komponente MotorController ist in die Klassen Motor und MotorController aufgeteilt.

\subsection{Motor}
Die Klasse Motor interagiert sehr viel mit der Hardware. Aus diesem Grund können nur zwei Objekte (Motor\_a, Motor\_b) für jeweils einen Motor erzeugt werden.
Der ESP32 besitzt ein Hardwaremodul "MCPWM\_UNIT" zur Steuerung von Motoren. Dieses Hardwaremodul hat mehrere Timer Kanäle über die ein PWM Signal zur Steuerung 
der IN1 oder IN2 Eingänge des Motortreibers erzeugt werden kann. Zur Erfassung der Motordrehzahl besitzt die "MCPWM\_UNIT" mehrere "Capture" Eingänge. 
Mit diesen werden die beiden Hallsensoren eines Motors verbunden. Eine positive Flanke erzeugt ein Interrupt, in dem die Zeitdifferenz zwischen zwei Hallsensorflanken berechnet wird.
Basierend darauf kann dann die Motordrehzahl ermittelt werden.
Die Klasse Motor übernimmt somit die Einrichtung der "MCPWM\_UNIT" über die Hardwareabstraktionsschicht von Espressife und stellt die Interruptfunktion zur Verfügung.
Außerdem befindet sich auch eine Methode zur Aktualisierung des PI-Reglers in der Motor Klasse. Diese Methode wird periodisch für beide Motoren von MotorController aufgerufen.

Im Folgenden wird auf die Methoden der Klasse Motor eingegangen:
\begin{itemize}
    \item \verb|void init_a()| und \verb|void init_b()| erzeugt das Motor\_a und Motor\_b Objekt.
    \item \verb|void setSetpointVelocity(float setpoint_velocity)| setzt die Solldrehzahl des Motors.
    \item \verb|float getSetpointVelocity()| gibt die Solldrehzahl des Motors zurück.
    \item \verb|float getActualVelocity()| gibt die aktuelle Motordrehzahl zurück. Diese wird aus der im Interrupt berechneten Zeitdifferenz ermittelt.
    \item \verb|float updatePIControl(float actual_velocity)| aktualisiert den PI-Regler. Die Methode gibt die Pulsbreite des PWM Stellsignals zurück. 
    \item \verb|void setDuty(float duty_cycle)| setzt die Pulsbreite für den Motortreiber.
    \item \verb|bool _encoder_callback(...)| wird von der "MCPWM\_UNIT" bei einem Interrupt durch eine positive Flanke auf einem der Hallsensoreingängen ausgeführt.
    Es wird die Zeitdifferenz zum letzten Aufruf ermittelt und über die Art der letzten Flanke (positiv/negativ) des zweiten Hallsignals die Drehrichtung bestimmt.
\end{itemize}

\subsection{MotorController}
Die Klasse MotorController steuert den PI-Regler in Motor. Ein Timer Interrupt des ESP32 führt jede Millisekunde eine Interruptfunktion aus MotorController aus.
Die \verb|updatePIControl()| Methode aus Motor wird allerdings nicht aus dem Interrupt ausgeführt. Der Aufruf befindet sich in einem FreeRTOS Prozess \verb|_motor_control_loop_task()|.
Dieser blockiert solange bis er aus der Interruptfunktion aufgeweckt wird. Dann holt sich der Prozess die aktuelle Drehzahl der Motoren \verb|getActualVelocity()|, aktualisiert die PI-Regler \verb|updatePIControl()|, setzt die neue Pulsbreite \verb|setDuty()| und wartet
wieder darauf aus dem Interrupt aufgeweckt zu werden. Das Steuern eines Prozesses aus einem Interrupt wird "Deferred Interrupt Handling" genannt und hat den Vorteil, 
dass das Interrupt möglichst schnell durchläuft und niemals blockiert wird. Außerdem wird aus MotorController der Enable Pin für den Motortreiber aktiviert.

Im Folgenden wird auf die Methoden der Klasse MotorController eingegangen:
\begin{itemize}
    \item \verb|MotorController& init(Motor& motor_a, Motor& motor_b)| erstellt das MotorController Objekt.
    \item \verb|void setVelocity(float setpoint_velocity_a, float setpoint_velocity_b)| setzt die Solldrehzahlen beider Motoren.
    \item \verb|void enablePIcontrol()| aktiviert den PI-Regler
    \item \verb|void disablePIcontrol()| deaktiviert den PI-Regler. Wenn der PI-Regler deaktiviert ist, werden die Solldrehzahlen zur Steuerung der Pulsbreiten des PWM-Signals genommen.
    \item \verb|void _motor_control_loop_task(void* pvParameters)| ist der FreeRTOS Prozess zur Steuerung des PI-Regelkreises.
    \item \verb|bool _motor_control_interrupt(void* args)| ist die Interruptfunktion, die den Prozess zyklisch aktiviert.
\end{itemize}


\section{OutputVelocity}

\begin{figure}[H]
    \centering
    \includegraphics[width=0.9\textwidth]{figures/Softwarearchitektur_ESP32/OutputVelocity_Class.png}
    \caption{Klassendiagramm OutputVelocity}
    \label{fig:outputvelocity_class}
\end{figure}

Die Komponente OutputVelocity besteht aus zwei Unterklassen die von einer Oberklasse OutputVelocity erben.
Es kann nur ein Objekt von OutputVelocityImpl oder OutputVelocitySim existieren.
Die Unterklasse OutputVelocityImpl ist für den echten Roboterbetrieb zuständig und OutputVelocitySim für den Simulationsbetrieb.
Durch die abstrakte Oberklasse ist ein einfacher Wechsel zwischen Simulation und Echtbetrieb möglich.
OuputVelocity ist nicht komplett abstrakt, da die Methode \verb|ros_msgs_lw::Twist2D getVelocity() const| implementiert wird.
Über diese Methode kann der aktuelle Geschwindigkeitsvektor des Roboters abgefragt werden.

\subsection{OuputVelocityImpl}
Die Unterklasse OutputVelocityImpl transformiert den Sollgeschwindigkeitsvektor des Roboters in Drehzahlen ($\omega_L \ und \ \omega_R$) für die Motoren und gibt diese an den MotorController weiter.
Der Geschwindigkeitsvektor eines Roboters mit Differentialantrieb besteht aus einer lineare Geschwindigkeit $v$ und einer Rotationsgeschwindigkeit $\omega$.
Für die Transformation muss außerdem der Radabstand $d$ und der Raddurchmesser $r$ bekannt sein.

\[
  \begin{pmatrix}
    \omega_R\\
    \omega_L\\
  \end{pmatrix} = 
  \frac{1}{2\pi\cdot r}
  \begin{pmatrix}
      1 & -0.5\cdot d\\
      -1 & -0.5\cdot d\\
  \end{pmatrix} \cdot
  \begin{pmatrix}
      v\\
      \omega\\
  \end{pmatrix}
\]

Die maximale Drehzahl der im Projekt verwendeten Polulu 20D Motoren beträgt 220 Umdrehungen pro Minute. 
Ergibt die Transformation eine größere Drehzahl werden beide Drehzahlen um den gleichen Faktor runterskaliert.
\begin{itemize}
    \item \verb|OutputVelocity& init(MotorController& motor_controller)| erzeugt das OuputVelocityImpl Objekt.
    \item \verb|void setVelocity(ros_msgs_lw::Twist2D const& velocity)| setzt einen neuen Geschwindigkeitsvektor, transformiert diesen in Drehzahlen und gibt diese an MotorController weiter.
\end{itemize}

\subsection{OutputVelocitySim}
Die Unterklasse OutputVelocitySim ermöglicht den Simulationsbetrieb. 
Geschwindigkeitvektoren werden nicht in Drehzahlen umgewandelt, sondern über einen Publisher an ROS gesendet.
In ROS können diese dann als Eingangssignal für z.B. den "Turtlesim" Simulator verwendet werden.
\begin{itemize}
    \item \verb|OutputVelocity& init(MotorController& motor_controller)| erzeugt das OuputVelocitySim Objekt.
    \item \verb|void setVelocity(ros_msgs_lw::Twist2D const& velocity)| setzt einen neuen Geschwindigkeitsvektor und sendet diesen an ROS.
\end{itemize}


\section{SensorPose}

\begin{figure}[H]
    \centering
    \includegraphics[width=\textwidth]{figures/Softwarearchitektur_ESP32/SensorPose_Class.png}
    \caption{Klassendiagramm SensorPose}
    \label{fig:sensorpose_class}
\end{figure}

Die Komponente SensorPose dient zur Erfassung der Pose des Roboters. Durch die Schnittstelle SensorPose ist es möglich zwischen Unterschiedlichen 
"Sensoren" zu wechseln. Diese Sensoren sind als Unterklasse implementiert. Im Projekt gibt es die Möglichkeit zwischen Marvelmind, SensorPoseSim und KalmanFilter zu wechseln.
Der Sensor Marvelmind liefert die Pose direkt aus den Messungen des Marvelmind Mini-RX Beacons. SensorPoseSim bekommt Positionsdaten aus ROS und ermöglicht dadurch die Simulation des Roboters in ROS.
KalmanFilter besitzt eine Liste von sogenannten KalmanSensoren. Marvelmind und SensorPoseSim sind auch Unterklassen der Schnittstelle KalmanSensor.
Die Positionsdaten der KalmanSensoren werden aufgwertet, so dass Positionsdaten mit einer höheren Aktualisierungsrate von $100Hz$ verfügbar sind.
Ursprünglich war auch eine Sensorfusion mit z.B. Motordrehzahl und Messungen einer IMU geplant. Dies konnte allerdings nicht mehr aus Zeitgründen umgesetzt werden.

\subsection{SensorPose}
Die Klasse SensorPose definiert die gemeinsame Schnittstelle aller Sensoren und ermöglicht somit eine einfaches Wechseln zwischen unterschiedlichen Sensoren.
SensorPose definiert zwei Möglichkeiten die Pose des Roboters zu lesen. 

\begin{itemize}
    \item \verb|bool getPose(ros_msgs_lw::Pose2D& current_pose)| liest nur eine Position, falls seit dem letzten Lesen eine neue Messung verfügbar ist.
    \item \verb|bool peekAtPose(ros_msgs_lw::Pose2D& current_pose)| liefert immer eine Position. Unabhängig von dem Alter der letzten Messung.
    \item Über \verb|void reInit()| können interne Zustände des Sensors neu initialisiert werden.
\end{itemize}


\subsection{KalmanFilter}
Die Positionsmessungen der Klassen Marvelmind und SensorPoseSim haben eine relativ geringe Aktualisierungsrate. Marvelmind gibt z.B. nur $7Hz\ - \ 15Hz$ an.
Da für die Positionsregelung allerdings eine Aktualisierungsrate von $100Hz$ vorgesehen ist, müssen über einen Kalman Filter öfter Positionsdaten bereitgestellt werden.
Der Kalman Filter schätzt über die Systemgleichung des Roboters zyklisch eine neue Position. Wenn eine Messung verfügbar wird, wird die Plausibilität der Positionsmessung und der geschätzten Position anhand
der jeweiligen Kovarianzmatrix bewertet. Basierend darauf werden beide Position dann fusioniert.
Der Zustandsvektor $z_t$ des Roboters ist:

\begin{equation}
    z_t =
    \begin{pmatrix}
        x_{t}\\
        y_{t}\\
        \theta_{t}
    \end{pmatrix}
\end{equation}

Der Geschwindigkeitsvektor $u_t$ dient als Steuersignal des Roboters:

\begin{equation}
    u_t =
    \begin{pmatrix}
        v\\
        \omega 
    \end{pmatrix}
\end{equation}

Der Differentialantrieb des Roboters wird mit dem Unicycle Modell \cite{ReSiSchwa} angenähert. Die Zustandsgleichung beträgt somit:

\begin{equation}
    z_t = f(z_{t-1}, u_t) =
    \begin{pmatrix}
        x_{t-1}\\
        y_{t-1}\\
        \theta_{t-1}
    \end{pmatrix}
    + \Delta t
    \begin{pmatrix}
        \cos\theta_{t-1} & 0\\
        \sin\theta_{t-1} & 0\\
        0          & 1
    \end{pmatrix}
    \begin{pmatrix}
        v\\
        \omega 
    \end{pmatrix}
    \label{eq:zustandsgleichung}
\end{equation}

Die Messungen der Sensoren $y_t$ werden über die folgende Beobachtungsgleichung in den Zustandsraum des Roboters $z_t$ transformiert:

\begin{equation}
    y_t = H \cdot z_t = 
    \begin{pmatrix}
        1 & 0 & 0 \\
        0 & 1 & 0 \\
        0 & 0 & 1
    \end{pmatrix}
    \begin{pmatrix}
        x_{t}\\
        y_{t}\\
        \theta_{t}
    \end{pmatrix}
\end{equation}

Der Kalman Filter geht von einem linearen Zustandsraum aus. Da die Zustandsgleichung des Roboters nicht linear ist, muss diese mithilfe der Jacobi-Matrix linearisiert werden.
Im Allgemeinen gilt dies auch für die Beobachtungsgleichung, wenn diese nicht linear ist.

\begin{equation}
    \begin{gathered}
      F_t = \left.\frac{\partial f}{\partial z}\right|_{z_{t-1}, u_t} =
      \begin{pmatrix}
          1 & 0 & -v\cdot \sin(\omega_t) \\
          0 & 1 & v\cdot \cos(\omega_t) \\
          0 & 0 & 1
      \end{pmatrix}
    \end{gathered}
\end{equation}

Bei dieser Erweiterung spricht man vom Extended Kalman Filter. Die Funktionsweise lässt sich in zwei Schritte unterteilen. Im ersten Vorhersage Schritt wird ein neuer Zustand $\hat{z}_t^-$ mithilfe der Zustandsgleichung \ref{eq:zustandsgleichung} geschätzt.
Außerdem wird die Unsicherheit $P_t^-$ dieser Schätzung mithilfe des Prozessrauschens $Q$ und mit der vorherigen Kovarianz $P_{t-1}^+$ berechnet.

\begin{equation}
    \begin{gathered}
        \hat{z}_t^- = f(\hat{z}_{t-1}^+, u_t) \\
        P_t^- = F_t P_{t-1}^+ F_t^T + Q \\
        \\
    \end{gathered}
\end{equation}

Im zweiten Update Schritt wird die Schätzung mithilfe der Positionsmessung $\tilde{y}_t$ von z.B. dem Marvelmindsensor korrigiert. Da auch jede Messung etwas Messrauschen $R$ besitzt, muss die Plausibilität der Messung und der Schätzung bestimmt werden.
Hierfür wird der Kalman Gain $K_t$ aus $P_t^-$ und $R$ berechnet. Der Kalman Gain bestimmt die Gewichtung der Messung in der abschließenden Positionsschätzung $\hat{z}_t^+$.
Außerdem wird noch die Kovarianz $P_t^+$ dieser Schätzung für den nächsten Durchlauf berechnet.

\begin{equation}
    \begin{gathered}
        K_t = P_t^- H^T (H P_t^- H^T + R)^{-1} \\
        \hat{z}_t^+ = \hat{z}_t^- + K_t (\tilde{y}_t - H\hat{z}_t^-)) \\
        P_t^+ = (I - K_t H)P_t^-
    \end{gathered}
\end{equation}

Falls neue Messungen noch nicht verfügbar sind, kann der zweite Schritt auch übersprungen werden und für den nächsten Zyklus $\hat{z}_t^+ = \hat{z}_t^-$ und  $P_t^+ = P_t^-$ gesetzt werden. 
Des Weiteren ist es durch Wiederholung des zweiten Schrittes für unterschiedliche Sensoren möglich, mehrere Messungen zu Fusionieren.
Dies wird im Projekt über eine Liste von KalmanSensoren umgesetzt. Diese wird in jedem Zyklus durchlaufen und die Zustandsvariablen aktualisiert, falls neue Messungen verfügbar sind.
Die Berechnung geschieht in einem FreeRTOS Prozess \verb|_kalman_filter_loop_task()|, der über einen Softwaretimer auf $100Hz$ getaktet wird.

\begin{itemize}
    \item \verb|SensorPose& init(std::initializer_list<KalmanSensor const*> const& sensor_list, OutputVelocity const& output_velocity)| erzeugt ein KalmanFilter Objekt. 
    Dem KalmanFilter muss eine Liste von KalmanSensoren übergeben werden sowie das OuputVelocity Objekt, aus dem der Sollgeschwindigkeitsvektor bezogen wird.
    \item Über die Methode \verb|void reInit()| wird der interne Zustandsvektor im KalmanFilter neu initialisiert.
    Hierfür stellt der erste Sensor aus der internen Liste Positionsdaten bereit. Dieser Mechanismus ist notwendig, wenn der Roboter z.B. händisch versetzt wird.
    \item Der Prozess \verb|void _kalman_filter_loop_task(void* pvParameters)| berechnet beide KalmanFilter Schritte. Nach einem Zyklus blockiert der Prozess bis er durch den Software Timer entblockt wurde.
    \item \verb|void _kalman_filter_loop_timer(TimerHandle_t timer)| ist die Callbackfunktion des Softwaretimers und entblockt den \verb|_kalman_filter_loop_task()| Prozess.
\end{itemize}

\subsection{KalmanSensor}
Die Klasse KalmanSensor definiert eine Schnittstelle für alle Sensoren, die von KalmanFilter fusioniert werden sollen.
Ein KalmanSensor muss die folgenden Methoden implementieren:

\begin{itemize}
    \item \verb|bool calculateKalman(ros_msgs_lw::Pose2D const& a_priori_estimate, dspm::Mat const& a_priori_cov, ros_msgs_lw::Pose2D& a_posterior_estimate, dspm::Mat& a_posterior_cov)| berechnet den Update Schritt für den jeweiligen KalmanSensor.
    \item \verb|bool getAbsolutePose(ros_msgs_lw::Pose2D& initial_pose)| gibt die aktuelle Pose des Roboters zurück. Diese Methode muss nur für Kalman Sensoren, die die absolute Position erfassen implementiert werden.
    \item \verb|void getMeasurementNoiseCov(dspm::Mat& measurement_cov)| gibt die Kovarianzmatrix des Messrauschens des jeweiligen Sensors zurück.
    \item \verb|void calculateMeasurementNoiseCov()| ermittelt das Messrauschen des jeweiligen Sensors.
\end{itemize}

\subsection{Marvelmind}
\begin{figure}[H]
    \centering
    \includegraphics[width=\textwidth]{figures/Softwarearchitektur_ESP32/Marvelmind_Positions_Paket.png}
    \caption{Marvelmind Positions Paket}
    \label{fig:marvelmind_pose_paket}
\end{figure}
Die Klasse Marvelmind kann als KalmanSensor und auch als alleinstehender Sensor über SensorPose genutzt werden. Marvelmind setzt die Hardwarekonfiguration für die UART Schnittstelle des ESP32 um und kommuniziert über UART mit dem Marvelmind Mini RX Beacon.
In dem Marvelmind Dashboard lassen sich unterschiedliche Pakete aktivieren, die über die UART Schnittstelle versendet werden. Diese sind in dem Dokument "Hardware interfaces and protocols of data exchange with Marvelminddevices" \cite{MarvelHaIn} definiert.
Abbildung \ref{fig:marvelmind_pose_paket} zeigt das in diesem Projekt genutzte Paket. Dieses übermittelt Position in mm und Orientierung in Dezigrad. 
Die Klasse Marvelmind liest die seriellen Pakete über einen FreeRTOS Prozess \verb|_uart_read_data_task()|, deserialisiert die Daten und speichert sie in einer FreeRTOS "Queue".
Die Pose kann dann über die Methoden der SensorsPose Schnittstelle ausgelesen werden oder als KalmanSensor von der Klasse KalmanFilter vewendet werden. 
Marvelmind definiert auch einige KConfig Menüeinstellungen für die UART Pins am ESP32, Geschwindigkeit der UART Kommunikation und die Größe eines Empfangspuffers.

\subsection{SensorPoseSim}
SensorPoseSim ist auch eine Unterklasse von SensorPose und KalmanSensor. Die Klasse dient zur Simulation des Roboters in ROS über z.B. den Turtlesim.
Die Roboterpose wird auf der Topic "/robotername/pose" empfangen. Diese wird im Konstruktor des SensorPoseSim Objektes abonniert. Die Methode \verb|void _setPose()| dient dabei als Callbackfunktion für den Subscriber. 


\section{PositionController}

\begin{figure}[H]
    \centering
    \includegraphics[width=\textwidth]{figures/Softwarearchitektur_ESP32/PositionController_Class.png}
    \caption{PositionController Klassendiagramm}
    \label{fig:position_controller_class}
\end{figure}

Die Komponente PositionController setzt den äußeren Regelkreise des Roboters um. Da der Roboter mit unterschiedlichen Regelaufgaben umgehen muss, ist kein Regler fest implementiert. Es wird dem Programmmierer erlaubt unterschiedliche Regler zu implementieren, 
die über eine gemeinsame Schnittstelle angesteuert werden. Im Projekt wurde der p2pController, approxLinController, statInOutLinController und dynInOutLinController Regler implementiert. Mit dem ersten können Punkte angefahren werden. Die letzen drei dienen zum Folgen von Trajektorien.
Da Funktionsweise und mathematische Herleitung dieser Regler in der vorhergehenden Projektarbeit "Regelung und Simulation von Schwarmrobotern" \cite{ReSiSchwa} behandelt wurde, wird in diesem Kapitel nur auf Besonderheiten der Implementierung eingegangen.

\subsection{ControllerMaster}
Die Klasse ControllerMaster schließt den Regelkreis der Positionsregelung. Es wird zunächst die aktuelle Position aus dem SensorPose Objekt bezogen, anschließend wird ein neuer Geschwindigkeitsvektor als Stellgröße über das PositionController Objekt berechnet und dieser Vektor an das OutputVelocity Objekt weitergereicht.
Zuletzt überprüft der ControllerMaster, ob das Ziel der Regelung erreicht wurde und beendet den Regelungszyklus. Durch eine Callbackfunktion werden andere Komponenten über das Ende der Regelung informiert. 
Der FreeRTOS Prozess \verb|_control_loop_task()| beinhaltet den Regelkreislauf und ein Softwaretimer taktet diesen auf $100Hz$.

\begin{itemize}
    \item \verb|ControllerMaster& init(OutputVelocity& output_velocity, SensorPose& sensor_pose)| erzeugt das ControllerMaster Objekt. 
    ControllerMaster ist eine Singleton Klasse. Es kann somit nur ein Objekt existieren.
    \item \verb|void start_controller(PositionController* pos_controller, std::function<void()>destination_reached_callback)| startet die Regelung. Falls eine Regelung bereits im Gange ist wird diese beendet und mit der neuen ersetzt.
    Der Methode wird neben dem PositionController Objekt auch die Callbackfunktion übergeben.
    \item Mit \verb|void stop_controller()| kann die Regelung von außerhalb der PositionController Komponente beendet werden.
    \item \verb|void _control_loop_task(void* pvParameters)| ist ein FreeRTOS Prozess und tätigt die Berechnung des Regelzyklus.
    \item \verb|void _control_loop_timer(TimerHandle_t timer)| ist die Callbackfunktion des Softwaretimers über den der Prozess getaktet wird.
\end{itemize}

\subsection{PositionController}
Die Klasse PositionController definiert die gemeinsame Schnittstelle über die die Regler vom ControllerMaster gesteuert werden.
Die Regelung ist komplett innerhalb der Unterklassen dieser Schnittstelle gekapselt. ControllerMaster besitzt somit keine Information über die Art der Regelung.

\begin{itemize}
    \item Mit \verb|ros_msgs_lw::Twist2D update(ros_msgs_lw::Pose2D const& actual_pose)| wird für eine neue Istposition ein neuer Geschwindigkeitsvektor als Stellgröße berechnet.
    Der ControllerMaster muss diese Methode alle $10ms$ aufrufen. 
    \item Mit \verb|bool destination_reached()| kann der ControllerMaster überpüfen, ob das Ziel der Regelung (Zielpunkt oder Trajektorienende) erreicht wurde.
\end{itemize}

\subsection{p2pController}
Über den Punkt zu Punkt Regler können einzelne Punkte angefahren werden. Dieser Zielpunkt wird dem Konstruktor des Reglers übergeben.

\subsection{approxLin-, statInOutLin-, dynInOutLinController}
Mit den Reglern approximierte Linearisierung, statische Ein- und Ausgangslinearisierung und dynamische Ein- und Ausgangslinearisierung können Trajektorien abgefahren werden.
Eine Trajektorie besteht aus einem Array von Trajektorienpunkten. Jeder Trajektorienpunkte beinhaltet Positions-, Geschwindigkeits, Beschleunigungsvektoren und einen Zeitstempel.
Über den Zeitstempel wird bei jedem Aufruf der \verb|update()| Funktion der aktuellste Trajektorienpunkt als Sollwert für die Trajektorienregelung genommen.
Da die Trajektorien als Nachrichten aus dem ROS-System entspringen, stimmt der Wert des Zeitstempels nicht mit der ESP32 internen Zeit überein. Diese Zeitdifferenz wird zum Beginn der Regelung berechnet und folgende Punkte korrigiert.


\section{StateMachine}

\begin{figure}[H]
    \centering
    \includegraphics[width=\textwidth]{figures/Softwarearchitektur_ESP32/State_Machine_Class.png}
    \caption{StateMachine Klassendiagramm}
    \label{fig:state_machine_class}
\end{figure}

Die Komponente StateMachine steuert das Verhalten des Roboters auf der höchsten Ebene. Eine Zustandsmaschine kontrolliert das Verhalten bei bestimmten Aktionen.
Die Zustandsmaschine ist nach dem "'State Pattern"' Entwurfsmuster implementiert. Für jede Methode des Objektes StateMachine definiert der Zustand eine Methode mit dem selben Namen.
Ein Aufruf von StateMachine wird an die Methode des derzeitigen Zustandes weitergereicht. Somit ändert sich das Verhalten des StateMachine Objektes basierend auf dem internen Zustand.
Abbildung \ref{fig:state_machine_class} zeigt das Klassendiagramm dieses Entwurfmusters. Die Übergänge der Zustandsmaschine sind von der Implementierung der Unterklassen von State abhängig.
Abbildung \ref{fig:zustandsdiagramm} zeigt die in diesem Projekt umgesetzte Zustandsmaschine. Die Methoden aus StateMachine werden als Callbackfunktion des RosBridgeClient Subscribers verwendet. 
Somit würde z.B. eine empfangene Geschwindigkeitsvektor Nachricht einen Zustandsübergang von Idle nach DriveWithVelocity veranlassen.

\begin{figure}[H]
    \centering
    \includegraphics[width=0.8\textwidth]{figures/Softwarearchitektur_ESP32/Zustandsdiagramm.png}
    \caption{Zustandsdiamm Roboter}
    \label{fig:zustandsdiagramm}
\end{figure}

\section{DataLogger}

Über die Komponente DataLogger können Systemdaten zum Debugging aufgezeichnet werden. Das Logging wird mit einer Nachricht auf der Topic "robotername/start\_log" gestartet.
Die Logging Dauer wird in der Nachricht als \verb|ros_msgs::String| codierte Gleitkommazahl übermittelt.
Über die Makrofunktion \verb|LOG_DATA()| können Logging Einträge erstellt werden.
Allerdings muss das Makro \verb|DATA_LOGGING| vor dem Inkludieren der Header Datei "DataLogger.h" definiert sein. Dies ermöglicht das einfache Ein- und Ausschalten des Loggings in einer Sourcedatei.
Die einzelnen Logs werden dann in einem Puffer zwischengespeichert, bis dieser eine bestimmte Größe erreicht hat. Anschließend wird der gesamte Puffer über den Nachrichtentyp \verb|ros_msgs::String| 
auf der Topic "robotername/data\_log" an das ROS-System gesendet. Das Logging wird automatisch nach der zu Beginn übermittelten Zeit beendet.
DataLogger definiert auch einige KConfig Menüeinstellungen. Die Komponente Data Logger besteht aus einer Klasse. Im Folgenden wird auf die Methoden dieser Klasse eingegangen:
\begin{itemize}
    \item \verb|DataLogger& init(ros::Publisher<ros_msgs::String>& publisher)| erstellt das DataLogger Objekt. DataLogger ist eine Singleton Klasse somit gibt es nur ein DataLogger Objekt.
    \item Über die Methode \verb|void startLogging(std::shared_ptr<ros_msgs::String> log_time)| wird das Logging gestartet. Dies Methode dient als Callbackfunktion für den Subscriber der Topic "robotername/start\_log".
    \item Über das Makro \verb|LOG_DATA(format, ...)| kann ein Logeintrag erstellt werden. Die Parameter der Makrofunktion entsprechen den Parameter der \verb|printf()| Funktion. 
    Es wird ein Formatstring und die in dem Formatstring definierten Variablen übergeben.
    \item \verb|void logData(const char* format, ...)| wird von dem Makro \verb|LOG_DATA()| aufgerufen. Es speichert die Logeinträge in einem Puffer zwischen.
    \item \verb|void _data_logger_task(void *pvParameters)| sendet den Puffer an das ROS-System, falls dieser eine maximale Größe erreicht hat.
\end{itemize}

\section{Main}

Die Main Komponente beinhaltet die \verb|main()| Funktion der ESP32 Software. Diese ist der erste von FreeRTOS gestartete Prozess. Zunächst werden alle Komponenten konfiguriert. Dann geht der main Prozess in eine unendlich Schleife über. 
Diese sendet alle $100ms$ die aktuelle Postion des Roboters auf der Topic "'robotername/pose2D"' an ROS. Die main Komponente besitzt zwei Einstellungspunkte im KConfig Menü zum einstellen der ROS Server IP- und Port-Adresse.
Über defines am Beginn der main.cpp Datei kann die Konfigurationen des Roboters verändert werden.

\begin{itemize}
    \item Mit \verb|#define Data_Logging| wird die DataLogging Komponente aktiviert und konfiguriert.
    \item \verb|#define KALMAN| aktiviert den Kalman Filter. Das SensorPose Objekt im Roboter ist somit vom Typ KalmanFilter.
    \item \verb|#define USE_SIM| konfiguriert den Roboter im Simulationsmodus. SensorPoseSim dient somit als Quelle der Postionsdaten und die Geschwindigkeitdaten werden an OutputVelocitySim gegeben.
    \item \verb|#define STEP_RESPONSE| deaktiviert den PI Regler und kann z.B. zum Aufzeichnen der Sprungantwort des Roboters genutzt werden.
\end{itemize}


% !TEX root = ../Projektdokumentation.tex
\chapter{ROS}\label{ch:ros}
Damit sich die einzelnen Roboter als Roboterschwarm bewegen können, muss es einen zentralen Computer geben, der die Roboter koordiniert.
Diese Projekt stützt die Entscheidung vorherige Roboterformations Projekte, ROS als Framework für die Programme auf diesem zentralen Computer zu nutzen.
Das Framework bietet eine Robotik orientierte Schnittstelle zum Nachrichtenaustausch zwischen verschiedenen Programmen.
Au{\ss}erdem bietet es auch viele Analysetools, die das Roboterdebugging und die Robotervisualiserung vereinfachen.

\section{Ros Bridge Server}
Da der ESP32 als externer Mikrocontroller nicht direkt in das ROS System eingebunden werden kann, wurde das Programm ROS Bridge Server entwickelt. 
Dieser Server verbindet sich mit der Komponente RosBridgeClient auf dem ESP32 und kommuniziert über das Applikationsprotokoll ROS-Topics mit dem Roboter.
Somit erscheint es innerhalb des ROS Systems, als wäre der Mikrocontroller des Roboters in das ROS System eingebunden.
Das ROS Programm setzt sich aus den Klassen Socket, CommunicationHandler, PublisherImpl, Subscriber und den RosMsgs Nachrichten zusammen.

\subsection*{Socket}
Die Klasse ist ähnlich wie die namensgleichen Klasse der ESP32 Sourcen aufgebaut. Allerdings abstrahiert Socket nun die POSIX API für einen TCP Server.
Dieser Server wartet bis sich ein Client über den entsprechenden Serverport verbinden will und baut dann eine TCP Verbindung mit diesem Client auf.
Diese TCP Verbindung wird über das Socket Objekt gekapselt. Allerdings wird auch der Verbindungsaufbau bereits über ein Socket Objekt getätigt.

\begin{itemize}
    \item Die Methode \bverb|void _create_socket()| konfiguriert einen TCP Server über die POSIX API. Dieser wird über einen "`File Descriptor"' repräsentiert und ist statisch in jedem Socket Objekt hinterlegt.
    Die Methode wird beendet, sobald der Server erstellt ist. Wenn es nach vier Versuchen nicht möglich war einen Server zu erstellen, wird das Programm beendet.
    \item Der Konstruktor \bverb|Socket()| ruft die Methode \bverb|_create_socket()| auf, wenn noch kein Server "`File Descriptor"' existiert.
    \item Über \bverb|int accept_connection()| werden Verbindungen von einem Client akzeptiert. Die Verbindung wird von der POSIX API ebenfalls durch einen "`File Descriptor"' repräsentiert.
    Dieser "`File Descriptor"' wird in dem jeweiligen Socket Objekt abgespeichert. Das Socket Objekt ist anschlie{\ss}end für diese Verbindung verwantwortlich.
    \item \bverb|void close_connection()| schlie{\ss}t die Verbindung mit dem jeweiligen Client.
    \item \bverb|int socket_receive(uint8_t* rx_buffer, int recv_bytes)| empfängt die mit \bverb|recv_bytes| übergebene Anzahl an Bytes aus dem internen TCP-Puffer. Die Methode blockiert solange bis
    alle Bytes vollständig empfangen wurden oder ein Fehler beim Emfangen auftritt. Es wird die Anzahl empfangener Bytes oder bei einem Fehler das Makro \bverb|SOCKET_FAIL| zurück gegeben.
    \item \bverb|int socket_receive_string(std::string& rx_string, int max_bytes)| dient zum Empfangen von mit \bverb|'\0'| beendeten Zeichenketten. Die Methode blockiert solange bis eine vollständige Zeichenkette
    empfangen wurde, \bverb|max_bytes| aus dem Puffer entnommen wurden oder ein Fehler beim Empfangen auftritt. Es wird die Anzahl empfangener Bytes oder bei einem Fehler das Makro \bverb|SOCKET_FAIL| zurück gegeben.
    \item \bverb|int socket_receive_nonblock(uint8_t* rx_buffer, int recv_bytes)| empfängt die mit \bverb|recv_bytes| übergebene Anzahl an Bytes aus dem internen TCP-Puffer. 
    Falls nicht genügend Bytes vorhanden sind oder ein Fehler beim Empfangen auftritt, endet die Methode sofort. Es wird die Anzahl empfangener Bytes oder bei einem Fehler das Makro \bverb|SOCKET_FAIL| zurück gegeben.
    \item Mit \bverb|int socket_send(uint8_t const* tx_buffer, int buffer_len)| werden \bverb|buffer_len| Bytes an den Server gesendet. 
    Es wird die Anzahl gesendeter Bytes oder bei einem Fehler das Makro \bverb|SOCKET_FAIL| zurück gegeben. 
\end{itemize}

\subsection*{RosMsgs}
Unter dem C++ Namespace \bverb|ros_msgs| befinden sich die Nachrichtentypen, mit denen über das Applikationsprotokoll zwischen ROS-Server und dem ESP32 Client kommuniziert wird.
Eine ähnliche Datei existiert auch in den Sourcen der ESP32 Komponente RosBridgeClient. Im Gegensatz dazu erben die einzelnen Nachrichtentypen aus RosBridgeServer von den entsprechenden ROS Datentypen.

\subsection*{Subscriber und PublisherImpl}

Subscriber und PublisherImpl kapseln jeweils ein Publisher oder Subscriber Objekt aus ROS. Subscriber abboniert über das interne ros::Subscriber Objekt eine ROS-Topic. 
Au{\ss}erdem stellt die Klasse eine Callbackfunktion \bverb|_subscribtion_callback()| bereit, die beim Empfangen einer Topic aus ROS ausgeführt wird. 
Diese Callbackfunktion serialisiert die ROS-Nachricht und sendet diese anschlie{\ss}end über das Applikationsprotokoll an den Mikrocontroller.
Der Publisher inseriert eine ROS-Topic über das ros::Publisher Objekt. Wenn eine Nachricht vom Mikrocontroller empfangen wurde, wird diese durch den Publisher in das ROS-System weitergeleitet.

Im Folgenden wird auf die Methoden der Klasse Subscriber eingegangen:
\begin{itemize}
    \item \bverb|void _subscribtion_callback(T const& msg)| wird als Callbackfunktion vom ros::Subscriber ausgeführt. Die Methode wird mit den beiden Parametern T und S parametriert.
    T ist ein Nachrichtentyp aus dem ROS-System. S ist der entsprechende selbstdefinierte Datentyp aus ros\_msgs. Zunächst wird die Nachricht von dem Typ T in den Typ S umgewandelt.
    Anschlie{\ss}end wird ein Publish-Paket aus Publish-ID, Topic-Name und den serialisierten Daten über Socket an den Client gesendet.
    \item Über \bverb|void subscribe(std::string const& topic, ros::NodeHandle* node_handle)| wird mit dem internen ros::Subscriber eine Topic abboniert.
    Diese Methode wird von CommunicationHandler nach dem Empfangen eines Subscribe-Pakets ausgeführt.
\end{itemize}

Die Klasse PublisherImpl muss im Gegensatz zu Subscriber mit T und S parametriert werden. T ist ein Nachrichtentyp aus dem ROS-System. S ist der entsprechende selbstdefinierte Datentyp aus ros\_msgs.
Im Folgenden wird auf die Methoden von PublisherImpl eingegangen:

\begin{itemize}
    \item \bverb|bool recvMessage()| wird von CommunicationHandler nach dem Empfangen eines Publish-Pakets ausgeführt. Die Methode deserialisiert den Datenteil des Pakets, wobei zwischen Array- und Strukturdaten unterschieden wird.
    Nach dem Deserialisieren werden die Daten von dem Datentyp S nach T umgewandelt und dann nach ROS gepublisht.
    \item \bverb|bool compareTopic(std::string const& topic)| vergleicht den übergebenen Topic-Namen mit dem Internen. Die Methode wird von CommunicationHandler verwendet, um für ein empfangenes Publish-Paket den richtigen Publisher zu finden.
\end{itemize}

\begin{figure}[H]
    \centering
    \includegraphics[width=0.75\textwidth]{figures/ROS/RosBridgeServerClass.png}
    \caption{Klassendiagramm RosBridgeServer}
    \label{fig:ros_bridge_server_class}
\end{figure}

\subsection*{CommunicationHandler}

\begin{figure}[H]
    \centering
    \includegraphics[width=0.5\textwidth]{figures/ROS/CommunicationHandlerAblauf.png}
    \caption{Ablaufdiagramm CommunicationHandler}
    \label{fig:communication_handler_ablauf}
\end{figure}

Die Klasse CommunicationHandler steuert die Verbindung über das Applikationsprotokoll mit dem ESP32. Somit gibt es für jeden Client auch ein CommunicationHandler Objekt im RosBridgeServer Programm.
Jedes CommunicationHandler Objekt kapselt einen eigenen Prozess. Dieser Prozess durchläuft ähnlich wie NodeHandle alle $50ms$ eine Schleife, die das Applikationsprotokoll steuert.
Nach dem Empfangen des Init-Pakets ist die Verbindung über das Applikationsprotokoll aufgebaut. Dann wird zyklisch ein Advertise-Paket versendet und die empfangenen Pakete ausgewertet. 
Wenn für mehr als $3000ms$ kein Advertise-Paket empfangen wurde oder beim Empfangen ein Fehler auftritt, wird die Verbindung mit dem Client geschlossen.
Das entsprechende Socket und CommunicationHandler Objekt werden gelöscht.

\begin{itemize}
    \item \bverb|void _communication_handler(CommunicationHandler *conn_handle)| ist der Prozess von CommunicationHandler.
    \item Der Konstruktor \bverb|CommunicationHandler(Socket& sock)| erstellt den Prozess \bverb|_communication_handler()|.
    \item \bverb|int _interpret_receive()| interpretiert die empfangenen Daten aus Socket. Hierfür wird zunächst basierend auf dem ersten Message ID Byte entschieden,
    ob ein Init, KeepAlive, Advertise, Subscribe oder ein Publish Paket empfangen wurde. Bei einem Init-Paket wird der Robotername entnommen und ein ros::NodeHandle Objekt mit dem Roboternamen erstellt. 
    Falls ein Keep-Alive Paket empfangen wurde, wird der übertragene Zeitstempel aus dem Socket Puffer entnommen und abgespeichert.
    Bei einem Subscribe oder Advertise Paket werden die CommunicationHandler Methoden \bverb|_subscribe()| oder \bverb|_advertise()| ausgeführt.
    Falls ein PublishPaket empfangen wurde, wird basierend auf dem übertragenen Topicnamen mithilfe der Methode \bverb|_getPublisher()| das richtige Publisher Objekt ermittelt und mit \bverb|recvMessage()| der Datenteil interpretiert.
    Läuft beim Empfangen oder Interpretieren etwas schief, gibt die Methode das Makro \bverb|SOCKET_FAIL| zurück.
    \item \bverb|void _send_keep_alive()| sendet das Keep-Alive Paket an den ROS-Server.
    \item \bverb|Publisher* _getPublisher(std::string const& topic)| gibt das jeweilige Publisher Objekt zu der Topic zurück.
    \item Mit \bverb|void _advertise(std::string const& topic, std::string const& message_type)| wird ein Publisher Objekt von der jeweiligen Topic erstellt. Dieses Objekt ist mit den entsprechenden Datentypen parametriert.
    \item \bverb|void _subscribe(std::string const& topic, std::string const& message_type)| erstellt ein Subscriber Objekt und ruft die \bverb|subscribe()| Methode mit den entsprechenden ROS-Datentyp Parametern auf.
    
\end{itemize}

\section{DataLogger}

Das ROS Programm DataLogger ist das Back-End zu der DataLogger Komponente im ESP32 Code. Nach Start des Programms wird eine Nachricht auf der Topic "'robotername/start\_log"' veröffentlicht. 
Die durch den ESP32 nach ROS gesendete LOG-Pakete werden auf der Topic "'robotername/data\_log"' empfangen und in einer Datei abgespeichert. 
Das Programm kann über ROS Parameter konfiguriert werden. Es können der Robotername, Logging Dauer, Dateiname und ein Logging Modus konfiguriert werden.

\section{RvizMsgTransformer}

Rviz ist ein von ROS bereitgestelltes Programm zur Roboter Visualisierung. Es können aber auch z.B. Zielpunkte in einem Koordinatensystem gesetzt werden.
In diesem Projekt wird Rviz zur Darstellung der Trajektorie und des Roboters in einem Koordinatensystem genutzt. Au{\ss}erdem soll ein Zielpunkt für den Punkt zu Punkt Regler gesetzt werden können.
Da Rviz mit zu diesem Projekt verschiedenen ROS Nachrichtentypen arbeitet, dient RvizMsgTransformer zur Übersetzung der Rviz Nachrichten.

\section{Trajektorienplanung}

\section{Starthilfe}
Die Sourcen für den ROS Code befinden sich im Projektordner oder können von dem GitHub Repository \url{https://github.com/maxdoesch/Roboterformation_ROS.git} geklont werden.
Im Projekt wird die Noetic Distribution von ROS genutzt. Diese muss zunächst installiert werden. Der Installtionsprozess wird im ROS Wiki \cite{ROS-Wiki} beschrieben.
Anschlie{\ss}end müssen die folgenden Schritte befolgt werden, um die ROS Programme zu bauen und zu starten:

\begin{enumerate}
    \item GitHub Repository klonen \newline
    \bverb|git clone https://github.com/maxdoesch/Roboterformation_ROS.git|
    \item Bauen des ROS Workspace \newline
    \bverb|catkin_make| im Oberverzeichnis des Workspace
    \item Den Workspace initialisieren \newline
    \bverb|source ./devel/setup.bash| 
    \item Rosmaster starten \newline
    \bverb|roscore|
    \item Zuletzt können die verschiedenen Nodes gestartet werden
    \begin{itemize}
        \item \bverb|rosrun ros_bridge_server ros_bridge_server|
        \item \bverb|rosrun trajecgenerator trajecgenerator_node|
        \item \bverb|rosrun rviz_msg_tranformer rviz_msg_transformer|
        \item \bverb|rosrun data_logger data_logger|
    \end{itemize}
\end{enumerate}

Es muss mindestens die Node ROSBridgeServer gestartet werden. Die Roboter verbinden sich automatisch nach dem Einschalten mit dem Server und sind dann in ROS unter ihrem Namen als Namespace sichtbar.
\footnote{Wenn ROS in einer virtuellen Maschine genutzt wird, muss eventuell der Serverport 2888 weitergeleitet werden. Au{\ss}erdem könnte der Port von der Host Firewall blockiert werden.}
Jetzt kann mit dem Roboter auf ihren Topics kommuniziert werden. Eine Liste aller verfügbarbaren Topics bekommt man mit \bverb|rostopic list|.

Mit \bverb|rostopic pub <topic_name> <nachrichtentyp> <daten>| kann eine Nachricht auf einer Topic versendet werden und mit \bverb|rostopic echo <topic_name>| können Nachrichten empfangen werden. 
Zum Beispiel kann mit dem Befehl \bverb|rostopic pub --once /robotername/goal_point geometry_msgs/Point| \bverb|"x: -2.0 y: 7.0 z: 0.0"| der Punkt (-2.0; 7.0) angefahren werden oder mit \bverb|rostopic echo /robot_1/pose2D| die Istposition ausgegeben werden.

Über das ROS Tool "'Teleop Keyboard"' kann die Robotergeschwindigkeit über die Tasten "'WASD"' gesteuert werden. Das Tool kann z.B. mit dem Apt Package Manager installiert werden \bverb|sudo apt-get install ros-noetic-teleop-twist-keyboard| 
und wird über den Befehl \bverb|rosrun teleop_twist_keyboard teleop_twist_keyboard.py cmd_vel:=/<roboter_name>/vel| gestartet. Am Ende des Befehls wird die Standardtopic \bverb|cmd_vel| auf die im Projekt verwendete Topic geändert.

Die Trajektoriengenerierung wird über die folgenden ROS Service gesteuert:
\begin{itemize}
    \item \bverb|rosservice call /m1/addCircleTrajecHandler|
    \item \bverb|rosservice call /m1/addCrSplineTrajecHandler|
    \item \bverb|rosservice call /m1/addCSplineTrajecHandler|
    \item \bverb|rosservice call /m1/removeTrajecHandler|
\end{itemize}
Die Bedeutung der Parameter für die Service kann der Dokumentation \cite{ReSiSchwa} der vorherigen Projektgruppe entnommen werden.

Zuletzt soll noch auf den Simulationsmodus des Roboters eingegangen werden. Zunächst muss der Roboter im ESP32 Code in den Simulationsmodus versetzt werden. 
Da der Simulator mit dem ROS Namespace \bverb|/turtle1| arbeitet, wird der Roboter auf diesen Namen umbenannt.
Im Simulationsmodus sind zwei zusätzliche Topics (/turtle1/cmd\_vel und /turtle1/pose) in ROS sichtbar. Diese werden als Stellgrö{\ss}e und Istgrö{\ss}e des Roboters vom Simulator "'Turtlesim"' genutzt.
Der Turtlesim Simulator wird mit dem Befehl \bverb|rosrun turtlesim turtlesim_node| gestartet.



% !TEX root = ../Projektdokumentation.tex
\chapter{ROS-Trajektorie}\label{ch:ros-trajec}
Dieses Kapitel behandelt die Umsetzung der durch die Vorgänger-Mastergruppe entwickelte Trajektorienplanung.
Genauso wird beschrieben, wie diese in ROS durchgeführt wird.
Falls zukünftig Änderungen und Erweiterungen vorgenommen werden, sollte die Vorgehensweise dazu ersichtlich werden.

\section{Trajektorienplanung}
Bisher wurde durch den Code der Mastergruppe jeder berechneter Trajektorienpunkt einzeln an den jeweiligen Roboter geschickt.
Da dies bei eventuell auftretenden Verbindungsschwierigkeiten dazu führen könnte, dass die Trajektorie nicht vollständig oder zeitversetzt übergeben wird, war es unsere Zielsetzung die komplett bestimmte Trajektorie gesammelt über das Verbindungsprotokoll zu verschicken.\\
In der nachfolgend abgebildeten Funktion werden die einzelnen für die Trajektorie relevanten Parameter nacheinander in einem Vektor abgespeichert.
Diesem wird jeweils ein Zeitstempel hinzugefügt, dass passend zu den errechneten Punkten und deren Ableitungen auch die entsprechenden Zeitinformationen vorhanden sind.
Nachdem ein Punkt mit der x- und y-Koordinate sowie den dazugehörigen ersten und zweiten Ableitungen mit Zeitinformation nun den eigens erstellten message Typ \texttt{c\_trajec} innehaben, werden diese Informationen im Vektor gespeichert.\\
In \ref{sec:msg-type} wird beschrieben, wie der message Typ individuell erstellt wird.
\begin{lstlisting}[language=C++,caption={Speichern einzelner errechneter Punkte in einem Vektor},breaklines=true,basicstyle=\footnotesize]
    void ToRobot::storeInTrajectory(pos_d trajectory_state)
    {
        trajecgenerator::c_trajec trajectory_state_vector;

        trajectory_state_vector.x = trajectory_state.x;
        trajectory_state_vector.y = trajectory_state.y;
        trajectory_state_vector.dx = trajectory_state.dx;
        trajectory_state_vector.dy = trajectory_state.dy;
        trajectory_state_vector.ddx = trajectory_state.ddx;
        trajectory_state_vector.ddy = trajectory_state.ddy;
        trajectory_state_vector.timestamp = ros::Time::now().toNSec();

        trajectory.points.push_back(trajectory_state_vector);
    }
    
\end{lstlisting}
Die Punkte der Trajektorie werden fortlaufend berechnet und in einem Vektor mit all ihren Informationen gespeichert.
Damit festgestellt werden kann, zu welchem Zeitpunkt alle einzelnen Punkte berechnet und somit im Vektor gespeichert wurden, wurde in der folgenden Funktion eine Abbruchbedingung festgelegt.
Ist die Distanz zwischen zwei nacheinander berrechneten Punkten geringer als 10 cm beziehungsweise wurden bereits 100 Punkte für die Trajektorie berechnet, bricht die Trajektoriengenerierung ab.
Anschlie{\ss}end wird die fertige Trajektorie als Vektor komplett gepublisht.
\begin{lstlisting}[language=C++,caption={Publishen der gesamten Trajektorie},breaklines=true,basicstyle=\footnotesize]
    bool ToRobot::publish()
    {
        float dist = sqrt(pow(trajectory.points.front().x - trajectory.points.back().x, 2) + pow(trajectory.points.front().y - trajectory.points.back().y, 2));

        if(dist < 0.1 && trajectory.points.size() > 100)
        {
            c_trajecPub.publish(trajectory);
            trajectory.points.clear();

            return true;
        }

        return false;
    }

\end{lstlisting}
Beide Funktionen werden hierbei vom \texttt{Trajechandler} aufgerufen.\\[\baselineskip]
Mit diesen Parametern dauerte es circa 20 Sekunden bis alle Punkte der Trajektorie berechnet, die gesamte Trajektorie gepublisht und vom Roboter letztlich ausgeführt wurde.

\section{Erstellen eines individuellen ROS Message-Types}
\label{sec:msg-type}
 Um die Trajektorie in einem Array speichern zu können, in dem alle für die auf dem Mikrocontroller implementierten Regler wichtigen Informationen enthalten sind, musste ein neuer, individueller "`message type"' in ROS implementiert.
Da die Beschreibungen und Anleitungen hierzu im Internet etwas spärlich ausfallen, wird im Folgenden anhand des in der Projektarbeit verwendeten custom message Types "`c\_trajec.msg"' erklärt, wie ein neuer message Typ erstellt wird.
 Dies kann im Falle einer Erweiterung oder Änderung der ROS-Software von Bedeutung sein.
 \begin{enumerate}
    \item \textbf{Navigieren zum entsprechenden ROS Paket} \newline
    \bverb|roscd trajecgenerator|
    \item \textbf{Erstellen eines \texttt{msg} Ordners (sofern nicht bereits vorhanden)} \newline
    \bverb|mkdir msg|
    \item \textbf{Navigieren zum \texttt{msg} Ordner} \newline
    \bverb|cd msg| 
    \item \textbf{Definitionen des neuen Message Typs erstellen (entweder per Kommandozeile oder direkt in der Datei)} \newline
    \bverb|echo "float32 x" > msg/c_trajec.msg| \newline
    \bverb|echo "float32 y" > msg/c_trajec.msg| \newline
    \bverb|echo "float32 dx" > msg/c_trajec.msg| \newline
    \bverb|echo "float32 dy" > msg/c_trajec.msg| \newline
    \bverb|echo "float32 ddx" > msg/c_trajec.msg| \newline
    \bverb|echo "float32 ddy" > msg/c_trajec.msg| \newline
    \bverb|echo "uint64 timestamp" > msg/c_trajec.msg|
    \item \textbf{Überprüfen der Inhalte in der \texttt{c\_trajec.msg} Datei} \newline
    \bverb|cat c_trajec.msg| 
    \item \textbf{Editieren der \texttt{package.xml} Datei} \newline
    In der Datei \texttt{package.xml} sicherstellen, dass die Inhalte \newline
    \bverb|<build_depend>message_generation</build_depend>| \newline
    \bverb|<run_depend>message_runtime</run_depend>| vorhanden und auskommentiert sind.
    \item \textbf{Editieren der \texttt{CmakeLists.txt} Datei} \newline
    In der Datei \texttt{CmakeLists.txt} bei \texttt{COMPONENTS} den Inhalt \texttt{message\_generation} hinzufügen, damit mindestens Folgendes zu sehen ist: \newline
    \bverb|find_package(catkin REQUIRED COMPONENTS| \newline
    \bverb|     roscpp| \newline
    \bverb|     rospy| \newline
    \bverb|     std_msgs| \newline
    \bverb|     message_generation| \newline
    \bverb|)|
    \item \textbf{Einstellen der Dependencies in der \texttt{CmakeLists.txt} Datei} \newline
    \bverb|catkin_package(| \newline
    \bverb|CATKIN_DEPENDS roscpp std_msgs| \newline
    \item \textbf{Hinzufügen der Message-Datei in der \texttt{CmakeLists.txt} Datei} \newline
    \bverb|add_message_files(| \newline
    \bverb|FILES| \newline 
    \bverb|c_trajec.msg| \newline
    \bverb|# c_trajec_vector.msg| \newline
    \bverb|)| \newline
    \item \textbf{Editieren der \texttt{CmakeLists.txt} Datei} \newline
    In der Datei \texttt{CmakeLists.txt} sicherstellen, dass die Inhalte \newline
    \bverb|generate_messages(| \newline
    \bverb|     DEPENDENCIES| \newline
    \bverb|     std_msgs| \newline
    \bverb|)| \newline vorhanden und auskommentiert sind.
    \item \textbf{Erneutes Bauen des Packages} \newline
    \bverb|roscd trajecgenerator| \newline
    \bverb|catkin_make|
\end{enumerate}





% !TEX root = ../Projektdokumentation.tex
\chapter{Marvelmind}\label{ch:marvelmind}
In diesem Kapitel wird beschrieben, wie der Aufbau seitens Marvelminds war.
Es wird dargestellt, welche Verbesserungen gegenüber der Vorgängergruppe ermittelt werden konnten.
An dieser Stelle soll auch auf die Dokumentation der Mastergruppe \cite{ReSiSchwa} hingewiesen werden.
Diese erläutern das grundlegende Konzept und die Hintergründe der Marvelmind-Beacons, sowie erste Einrichtungsschritte.
Wurde noch nicht mit dem Marvelmind Lokalisierungssystem gearbeitet, wird empfohlen das Kapitel 7 der Dokumentation \cite{ReSiSchwa} zu lesen und als Referenz zu verwenden.
Um Redundanzen zu vermeiden, werden auf diese Details in dieser Projektarbeit-Dokumentation nicht mehr eingegangen.\\
Im Folgenden werden die Einstellungen und der fortlaufende Lernprozess beschrieben, der zu wichtigen Erkenntnissen führte, die in der Nutzung des Marvelmind-Systems zu beachten sind.
Diese sind die Grundlage für die herausgearbeiteten Ergebnisse.\\
Zur weiteren Nutzung, Einrichtung und Weiterentwicklung durch eine spätere Gruppe sollte die nachfolgende Beschreibung ausreichend sein, um eine erfolgreiche Inbetriebnahme dés Marvelmind-Systems vornehmen zu können.


\section{Marvelmind-Aufbau}
Der Aufbau des Marvelmindsystems ist in dieser und den vorangegangenen Projektarbeiten als inverse Architektur realisiert.
Dies bedeutet, dass die stationären Beacons Ultraschallsignale aussenden und die mobilen Beacons diese empfangen.
Deshalb müssen die stationären Beacons jeweils unterschiedliche Frequenzen nutzen.\\[\baselineskip]
Zu Beginn unserer Projektarbeit wurden uns zwei stationäre Beacons (1x 20,2 kHz und 1x 32,7 kHz) sowie ein mobiler Beacon und ein dem Marvelmind-System dazugehöriges Modem übergeben.
Diese wurden wie in der Abbildung \ref{fig:marv_two_beacons} in einer Ebene angeordnet.
Hierzu wurden sie mit circa 6 m Abstand voneinander an einer Wand angebracht.
\begin{figure}[H]
    \centering
    \includegraphics[width=\textwidth]{figures/Marvelmind/zweiBeaconsAufbau}
    \caption{Schematischer Aufbau bei einer Nutzung von zwei stationären Beacons}
    \label{fig:marv_two_beacons}
\end{figure}
An dieser Stelle soll betont werden, dass bei der Platzierung der Beacons grundsätzlich darauf geachtet werden muss, dass sowohl die stationären als auch die mobilen Beacons zueinander jeweils Sichtkontakt haben müssen.
Somit werden störende Reflexionen der Ultraschallsignale vermieden, die zu fehlerhaften und ungenauen Lokalisationsdaten führen können.\\
Trotz eines solchen Aufbaus konnten mit zwei vorhandenen statiionären Beacons keine zuverlässigen Messdaten bestimmt werden.
Die Position des Roboters, an dem ein mobiler Beacon befestigt wurde, konnte nicht genau ermittelt werden.
Vielmehr gab es häufige "`Sprünge"' des Roboters auf der Karte im Marvelmind-Dashboard, da zeitweise die Verbindung zum mobilen Beacon scheinbar abgebrochen ist.\\[\baselineskip]
Aufgrund dessen wurde ein dritter stationärer Beacon bestellt, der mit einer Frequenz von 45 kHz betrieben wurde.
Dadurch erhoffte man sich mittels der Trilateration bessere, genauere und stabile Messdaten zu erhalten.\\
Für zukünftige Gruppen ist hier zu erwähnen, dass bei Nachbestellungen (Austausch, Reparatur oder Ergänzung eines stationären Beacons) explizit die gewünschte Ultraschallfrequenz anzugeben ist.
Nach Erfahrungsberichten scheinen Onlineshops gerne, trotz Angabe der Wunschfrequenz, standardmä{\ss}ig einen 20 kHz Beacon zu liefern.\\
Nachdem der dritte Beacon geliefert wurde, ergänzten wir diesen zum bereits existierenden Aufbau.
So platzierten wir den neuen Beacon an einer weiteren Wand, während die zwei vorherigen Beacons aufgebaut in einer Ebene verblieben.
Somit ergab sich circa eine "`L-förmige"' Anordnung.
Diese ist schematisch in Abbildung \ref{fig:marv_three_beacons_L} zu sehen.
\begin{figure}[H]
    \centering
    \includegraphics[width=\textwidth]{figures/Marvelmind/Skizze_Beacons_L}
    \caption{Schematische "`L-förmige"' Anordnung der stationären Beacons $\rightarrow$ nicht empfehlenswert}
    \label{fig:marv_three_beacons_L}
\end{figure}

Trotz des Hinzufügens eines dritten Beacons war in der "`L-förmigen"' Anordnung allerdings weiterhin keine wirkliche Verbesserung in der Lokalsierung des Roboters und in der Stabilität der Marvelmind-Messdaten zu erkennen.
Daher probierten wir eine unterschiedliche Anordnung, die letztlich zu deutlich besseren Messergebnissen führte.\\
So wurden die drei stationären Beacons nun jeweils an einer unterschiedlichen Wand angeordnet.
Dadurch bildeten die Beacons ungefähr Eckpunkte eines Dreiecks ab, was in Abbildungen \ref{fig:marv_three_beacons_Skizze} und \ref{fig:marv_three_beacons} sichtbar wird.
Hierbei waren die Eckpunkte ungefähr fünf Meter voneinander entfernt.
\begin{figure}[H]
    \centering
    \includegraphics[width=\textwidth]{figures/Marvelmind/Skizze_Beacons_3}
    \caption{Verbesserter schematischer Aufbau bei einer Nutzung von drei stationären Beacons}
    \label{fig:marv_three_beacons_Skizze}
\end{figure}

\begin{figure}[H]
    \centering
    \includegraphics[width=\textwidth]{figures/Marvelmind/dreiBeaconsAufbau}
    \caption{Verbesserter schematischer Aufbau bei einer Nutzung von drei stationären Beacons}
    \label{fig:marv_three_beacons}
\end{figure}
Nachdem die Karte im Dashboard entsprechend des Hardware-Aufbaus geändert wurde, konnte eindeutig eine Besserung der Lokalisierung festgestellt werden: 
Der mobile Beacon auf dem Roboter wurde genauer lokalisiert, stabilere Messdaten ermittelt und nun passierten auch deutlich weniger häufige Verbindungsabbrüche, bei denen der Roboter in der Lokalisierungskarte augenscheinlich "`Sprünge"' vornahm.
Auffällig war jetzt auch au{\ss}erdem, dass wenn doch eine völlig falsche Position des Roboters angezeigt wurde, diese relativ schnell korrigiert wurden, sodass der Roboter nach einigen wenigen Augenblicken wieder korrekt in der Karte zu verordnen war.\\[\baselineskip]


%
Gleichzeitig konnte festgestellt werden, dass auch die räumliche Gegebenheit die Qualität der Marvelmind Ergebnisse beeinflusst.
In den Gebäuden der Hochschule in der Wassertorstra{\ss}e gibt es einige Räume, die keine geschlossene Deckenabhängung haben.
Dies hat sich tatsächlich als Vorteil erwiesen, da im ersten Versuch ein Raum verwendet wurde, bei dem die Decke völlig geschlossen war.
Unter diesen Bedingungen war zu beobachten, dass sich die Messdaten im Vergleich zu einem Raum mit offener Decke als schlechter erwiesen.
Daher vermuten wir, dass durch die geschlossene Decke und der somit zusätzlichen Reflexionsfläche das Marvelmindsystem Schwierigkeiten hatte die Laufzeiten korrekt zu berechnen.
Dadurch war auch mit drei Beacons keine deutliche Verbesserung unter diesen Umständen in der Lokalisierung zu beobachten.\\
Für die Zukunft ist als Projekt-Arbeitsstätte, wenn möglich, ein leerer Seminarraum zu empfehlen.
Dieser sollte circa eine freie Bodenfläche mit den Dimensionen von 7 x 7 Metern besitzen.
Wichtig ist, dass keine unnötigen Reflexionen, wie zum Beispiel durch Tische oder Ähnliches, geschaffen werden.
Daher sind beispielsweise das Regelungstechnik-Labor oder ein Büro ungeeignet, da die Raum- und Bodenfläche schlichtweg zu gering ist.\\
Genauso ist es wichtig, dass die stationären Beacons jederzeit eine direkte Sichtverbindung haben können.\\[\baselineskip]

\section{Marvelmind-Dashboard}
Marvelmind bietet in den verschiedenen Betriebssystem-Umgebungen jeweils ein \texttt{Marvelmind-Dashboard} an, das eine grafische Übersicht der Marvelmind-Beacons und zudem Einstellungsmöglichkeiten bietet.
Allerdings ist nach Erfahrungsberichten eine Installation unter Windows zu empfehlen. Diese funktionierte tendenziell eher reibungslos als unter Linux.\\
Für die grundsätzliche Trajektorienplanung und die entsprechende Ausführung der Trajektorien ist die Software eigentlich nicht notwendig.
Vielmehr sind die durch Marvelmind bereitgestellten Lokalisierungsdaten essentiell.
Allerdings ist ein grafischer Überblick definitiv von Vorteil, da es dadurch optisch leicht festzustellen ist, ob die Lokalisierung fehlerhaft ist, somit den Reglern falsche Positionsdaten weitergegeben werden und letzlich die Trajektorie nicht korrekt ausgeführt wird.\\[\baselineskip]
\subsection{(Erst-)Einrichtung des Marvelmind-Systems}
Die erstmalige Treiberinstallation der Beacons ist durch uns schon vorgenommen worden.
Werden Beacons allerdings ausgetauscht beziehungsweise neue hinzugefügt, darf nicht vergessen, dass vor der Nutzung noch die Ersteinrichtung durchzuführen ist. 
Die Inbetriebnahme wird im Detail im Kapitel 7.5 der Dokumentation \cite{ReSiSchwa} beschrieben.
\subsection{Einstellungen im Marvelmind-Dashboard}
Hier werden im nachfolgenden Teil die wichtigsten Einstellungen, die im Dashboard vorgenommen werden, zusammenfassend dargestellt.
Da in der Projektarbeit die inverse Architektur als Marvelmind-Systemstruktur verwendet wird, können die stationären Beacons nicht automatisch ihre Positionen in der Karte anzeigen.
Daher muss dies manuell im Dashboard eingestellt werden.
Am unteren Rand des Dashboards werden alle 250 möglichen Beacons aufgelistet.
Das System erkennt nicht automatisch, welche Beacons mit der entsprechenden Adresse aktiv sind und verwendet werden sollen.
Damit die genutzten Beacons aktiviert werden, müssen diese "`aufgeweckt"' werden.
Dies geschieht, indem man in der Auflistung der gesamt möglichen Beacons jeweils den Beacon mit der korrekten Adresse anklickt.\\[\baselineskip]
In dieser Projektarbeit wurden folgende Beacons mit den entsprechenden Adressen verwendet:
\begin{table}[H]
    \begin{tabular}{|l|c|c|l|}
    \hline
    \textbf{Marvelmind-Bestandteil} & \textbf{Frequenz} & \textbf{Adresse} & \textbf{Beschreibung}                 \\ \hline
    Super Beacon HW 4.9             & 20,2 kHz          & 44               & stationärer Beacon                    \\
    Super Beacon HW 4.9             & 32,7 kHz          & 77               & stationärer Beacon                    \\
    Super Beacon HW 4.9             & 49 kHz            & xx               & stationärer Beacon                    \\
    Beacon-Mini-RX                  & /                 & 58               & mobiler Beacon (Hedgehog) auf Roboter \\
    Beacon-Mini-RX                  & /                 & 70               & mobiler Beacon (Hedgehog) auf Roboter \\
    Modem HW 4.9                    & /                 & 01               & Modem                                 \\ \hline
    \end{tabular}
    \end{table}
Dementsprechend müssen diese Beacons mit den jeweiligen Geräteadressen im unteren Feld des Marvelmind-Dashboards ausgewählt werden.
Nachdem sie aufgeweckt wurden, sollten die Beacons in der Dashboard-Oberfläche nun auftauchen.\\
Um die Einstellungen zu bearbeiten, muss die Schaltfläche \texttt{unfreeze submap} betätigt werden, da somit die Karte in den Bearbeitungsmodus übergeht.\\
Mit einem Rechtsklick auf die Beacons im unteren Feld des Dashboards können nun weitere Einstellungen vorgenommen werden.
Bei einer Ersteinrichtung müssen in diesem Schritt zum Beispiel die stationären Beacons noch jeweils einer Submap hinzugefügt.
Alle Beacons sollten dann der gleichen Submap hinzugefügt werden.
Da momentan drei stationäre Beacons verwendet werden, reicht hierzu eine Submap aus, da maximal vier Stück pro Submap verwendet werden können.\\
Nach dem Markieren des Beacons werden zudem alle Parameter aufgelistet.
So kann die Position der Beacons festgelegt werden, indem die Parameter in der Tabelle eingetragen werden.
Hier wird die relative Position der Beacons zueinander eingetragen.
Genauso ist es möglich die absolute Position der Beacons zum Koordinatensystem festzulegen.
Dies kann über "`Manual setup coordinates"' nach einem Rechtsklick auf den entsprechenden Beacon geändert werden.
So können die X, Y und Z-Koordinate im Koordinatensystem, welches im Dashboard angezeigt wird, eingetragen werden.
Diese Angaben erfolgen in Metern und beziehen sich als Anhaltspunkt jeweils auf die Mitte des Beacons.\\[\baselineskip]
Au{\ss}erdem müssen als Grundeinstellungen folgende Parameter konfiguriert werden:
\begin{itemize}
    \item "`Radio frequency band"' $\rightarrow$ 868 MHz
    \item "`Parameters of radio $\rightarrow$  Radio profile"' $\rightarrow$ 153 Kbps
    \item "`Device address"' $\rightarrow$ Festlegen der einzelnen Geräteadressen 
\end{itemize}
Nachdem alle Einstellungen vorgenommen wurden, kann der Bearbeitungsmodus verlassen werden.
Hierzu muss die Schaltfläche \texttt{freeze submap} betätigt werden.

Mit all diesen Einstellungen konnten beispielsweise folgende zwei Formen mit Marvelmind aufgezeichnet werden.
\begin{figure}[H]
    \centering
    \includegraphics[width=\textwidth]{figures/Marvelmind/Screenshot_liegendeAcht}
    \caption{Screenshot eines erfolgreich durchgeführten Catmull-Rom Splines im Marvelmind Dashboard}
    \label{fig:marv_screenshot8}
\end{figure}
%
%\vspace*{2cm}
%
\begin{figure}[H]
    \centering
    \includegraphics[width=\textwidth]{figures/Marvelmind/Screenshot_Kreis}
    \caption{Screenshot einer erfolgreich durchgeführten Kreis-Trajektorie im Marvelmind Dashboard}
    \label{fig:marv_screenshotCircle}
\end{figure}
\section{Probleme in der Nutzung von Marvelmind}
Obwohl während dieser Projektarbeit einige sehr wichtige Erkenntnisse gewonnen wurden, die die Genauigkeit und Stabilität der Lokalisierungsdaten des Marvelmind-Systems verbesserten, bleiben für einige Punkte weiterhin Raum für Verbesserungen.
Selbst wenn die Lokalisierung mit der ergänzten Hardware und dem verbesserten Aufbau deutlich optimiert wurden, kann Stand jetzt keine hundertprozentige Zuverlässigkeit gewährleistet werden.
In einigen Momenten bleibt weiterhin das Problem bestehen, dass Marvelmind ungenaue Lokalisationsdaten ermittelt, wenngleich das System diese meist zeitnah korrigiert.
Allerdings können zu diesem Zeitpunkt keine weiteren expliziten Lösungsansätze für die Marvelmind-Hardware selbst vorgeschlagen werden, da diese von unserer Projektgruppe letztlich bereits durchgeführt wurden.
Allerdings besteht die Hoffnung und die Zielsetzung mit der Erweiterung des Kalman-Filters der zeitweisen (Un-)Genauigkeit der Positionsdaten entgegenzuwirken.\\[\baselineskip]
Einige dieser Erkenntnisse konnten auch durch den Dialog mit Marvelmind als direkter Kontakt gewonnen werden.
Falls somit zukünftig Probleme mit dem Marvelmind-System auftreten, kann daher eine Anfrage an Marvelmind in Erwägung gezogen werden, da nach einem Beitrag in deren Forum zeitnah auch eine hilfreiche Rückmeldung kam.



% !TEX root = ../Projektdokumentation.tex
\chapter{Docker}\label{ch:docker}
Zur Vereinfachung des Arbeitsflusses und Ordnung einzelner Softwarekomponenten wurde \texttt{Docker} und \texttt{Docker Compose} verwendet.\\
Im folgenden Kapitel wird nun die Funktion und Vorgehensweise während der Nutzung von \texttt{Docker} erklärt.
\vspace{12cm}
\clearpage
\section{Docker}




\subsection{Einführung und Funktion von Docker}
Docker ist eine offene Plattform für die Entwicklung, Bereitstellung und Ausführung von Anwendungen.
Dabei verpackt Docker die Software in standardisierte Einheiten, sogenannte Container.
Diese enthalten alles, was zum Ausführen der Software erforderlich ist, wie Bibliotheken, Systemtools, Code und Laufzeit.\footnote{weitere Infos unter \url{https://docs.docker.com/get-started/overview/}}\\
Docker führt also Container aus.
Dabei fungiert ein Container als eine Art von Virtueller Maschine.
Allerdings ist hier der Aufbau relativ leicht, da dieser nicht virtualisiert ist und auf dem selben Kernel läuft.
Gleichzeitig ist der Container trotzdem abgeschottet, was dem Prinzip des Sandboxings
\footnote{Sandboxing ist eine Softwareverwaltungsstrategie, die Anwendungen von wichtigen Systemressourcen und anderen Programmen isoliert
Sie bietet eine zusätzliche Sicherheitsebene, die verhindert, dass sich Malware oder schädliche Anwendungen negativ auf Ihr System auswirken.\\Nähere Infos unter: \url{https://techterms.com/definition/sandboxing}} ähnelt.\\[\baselineskip]
Dabei erstellt man Container von einem Abbild.
Diese kann man selbst erstellen und nennt man Container-Image.
Der große Vorteil ist, dass man Container letztlich wie Rezepte zum Beschreiben verwendet, die ein leichtes Installieren der Software garantieren.\\
Eine Erweiterung stellt Docker Compose dar.
Dies erlaubt uns einen kompletten Stack zu beschreiben, also ganze Ansammlungen von Containern (und nicht nur einzelne Container) zusammen zu starten oder mit denen zu kommunizieren.\\[\baselineskip]
Bei einem Start des Projekts muss nur das repository gecloned werden und ein Befehl gestartet werden.
Diese Art zu Arbeiten ermöglicht einen schnellen Workflow.
\section{Installation von Docker-Desktop}
\begin{enumerate}
  \item Stellen Sie zunächst sicher, dass Windows auf dem neuesten Stand ist
  \begin{itemize}
    \item Geben Sie in der Windows-Suche "Windows Update" ein und wählen Sie \texttt{Windows Update-Einstellung}.
    \item Sie sollten ein grünes Häkchen sehen und "Sie sind auf dem neuesten Stand" . Falls nicht, klicken Sie auf "Nach Updates suchen". Sie müssen diesen Vorgang so lange wiederholen, bis Sie keine Updates mehr zu installieren haben.
  \end{itemize}
  \item Installation von \href{https://docs.microsoft.com/en-us/windows/wsl/install}{WSL2}
  \begin{itemize}
    \item Geben Sie in der Windows-Suche "powershell" ein, klicken Sie mit der rechten 
    Maustaste auf \texttt{Windows PowerShell} und dann auf \texttt{Als Administrator ausführen}.
    \item Klicken Sie auf "Ja", um PowerShell zu erlauben, Änderungen an Ihrem Gerät vorzunehmen.
    \item Führen Sie im Windows PowerShell-Fenster den Befehl "wsl --install -d Ubuntu" aus.
    \item Aktivieren Sie anschließend die Plattform für virtuelle Maschinen. Im Windows PowerShell ausführen. 
    (kopieren und einfügen) "dism.exe /online /enable-feature /featurename:VirtualMachinePlatform /all /norestart".
    \item Laden Sie das \href{https://wslstorestorage.blob.core.windows.net/wslblob/wsl_update_x64.msi}{WSL2 Linux-Kernel-Update-Paket für x64-Maschinen} herunter und installieren Sie es.
    \item Windows neu starten. 
    \item Geben Sie nochmal in der Windows-Suche "powershell" ein, klicken Sie mit der rechten 
    Maustaste auf \texttt{Windows PowerShell} und dann auf \texttt{Als Administrator ausführen}.
    \item Führen Sie im Windows PowerShell-Fenster den Befehl "wsl --set-default-version 2" aus.
    \item Als nächstes installieren Sie eine Linux-Distribution aus dem \href{https://aka.ms/wslstore}{Microsoft Store}. Ich empfehle \href{https://www.microsoft.com/store/productId/9MTTCL66CPXJ}{Ubuntu 20.04.4 LTS}
    .(Das Herunterladen und Installieren wird einige Minuten dauern)
    \item Sie werden aufgefordert, einen Linux-Benutzer einzurichten. Am besten Verwenden Sie dafür denselben Benutzernamen, den Sie für Windows verwenden.
    \item Sie können nun Linux-Befehle im Ubuntu-Terminalfenster ausführen. Ich empfehle, das Ubuntu-Symbol an die Taskleiste zu heften.
  \end{itemize}
  \item Jetzt können Sie \href{https://desktop.docker.com/win/main/amd64/Docker%20Desktop%20Installer.exe}{Docker Desktop} für Windows installieren
  \begin{itemize}
  \item Führen Sie das Installationsprogramm aus und starten Sie anschließend Windows neu.
  \item Melden Sie sich bei Windows an und starten Sie Docker-Desktop.Lassen sie Docker die Einrichtung abschließen, dies kann je nach Rechner einige Minuten dauern.
\end{itemize}
\end{enumerate}
\newpage
\section{Unsere Verwendung von Docker}
\subsection{Docker-Compose}
\label{dockercompose}
Sämtliche Software, die in unserem Projekt ausgeführt wird, läuft in Docker-Containern. Im folgenden werden die \texttt{Docker Compose}-Befehle, welche wir in unserem 
\textit{dokcer-compose.yml} Datei verwenden auführlich erklärt. 
\lstdefinelanguage{docker-compose-2}{
  keywords={version, volumes,networks,services},
  keywordstyle=\color{blue}\bfseries,
  keywords=[2]{image, environment, ports,networks,command,depends_on,context, container_name, ports, links, build},
  keywordstyle=[2]\color{olive}\bfseries,
  identifierstyle=\color{black},
  sensitive=false,
  comment=[l]{\#},
  commentstyle=\color{purple}\ttfamily,
  stringstyle=\color{red}\ttfamily,
  morestring=[b]',
  morestring=[b]"
}
\begin{lstlisting}[language=docker-compose-2,caption={docker-compose.yml}\label{buildBefehle},breaklines=true,basicstyle=\footnotesize]
version: '3.9' #Compose file format wird definiert, hierfuer wird Docker Engine 19.03.0 und hoeher verwendet
networks: #Mit diesen Befehlen erzeugen wir ein Netzwerk names rosnet
  rosnet:

services: #Unter services werden die einzelnen Containern definiert

  master: #Hier wird ein Container mit dem Namen "master" angelegt
    image: ros:noetic-robot # Als image wird hier "ros:noetic-robot" aus Docker-Hub verwendet 
    command:
      - roscore #Mit command wird der Befehl "roscore" in der Kommandozeile ausgefuehrt
    networks:
      - rosnet #networks definiert das Netzwerk des masters 

#Da, die Einstellungen und Befehle von den letzten zwei Container sich aehneln, wird es nur fuer ein Container erklaert.  
  rosbridge: #Hier wird ein Container mit dem Namen "rosbridge" angelegt
    build: 
      context: ./rosbridge #Als image, bzw build wird hier zu einem Dockerfile navigiert, welche unter dem Verzeichnis "./rosbridge" befindet
    environment:
      - "ROS_HOSTNAME=rosbridge"
      - "ROS_MASTER_URI=http://master:11311" #Unter environment werden Environmentvariablen festgelegt, in dem Fall wie der HOSTNAME des Containers lautet und wo sich der master im Netzwerk befindet
    networks:
      - rosnet 
    depends_on: 
      - master #Der Container faehrt erst hoch, wenn der master bereits laeuft und schaltet sich vorm master aus 
    ports:
      - 9090:9090 #Hier werden Ports in diesem Form (HOST-PORT:CONTAINER-PORT) nach aussen freigegeben 

  rosrobotbridge:
    build:
      context: ./rosrobotbridge
    environment:
      - "ROS_HOSTNAME=rosrobotbridge"
      - "ROS_MASTER_URI=http://master:11311"
    networks:
      - rosnet
    depends_on:
      - master
    ports:
      - 2888:2888
\end{lstlisting}
\newpage
\subsection{Dockerfile}
Im Unterkapitel \hyperref[dockercompose]{Docker-Compose} werden in einem \textit{dokcer-compose.yml} Datei mehrere \hyperref[buildBefehle]{build-Befehle} aufgerufen, welche zu dem eigentlichen \textit{Dockerfile} nagivieren und diesen als 
\newline Hauptimage bauen. Im folgenden werden die Befehle aus dem \textit{Dockerfile} und deren \newline Funktionen abhand eines Beispiels aus unserem Projekt erklärt und geschildert. 
\lstdefinelanguage{Dockerfile}{
  keywords={FROM, RUN,COPY,CMD},
  keywordstyle=\color{blue},
  identifierstyle=\color{orange},
  sensitive=false,
  comment=[l]{\#},
  commentstyle=\color{purple}\ttfamily,
  stringstyle=\color{red}\ttfamily,
  morestring=[b]',
  morestring=[b]"
}
\begin{lstlisting}[language=Dockerfile,caption={Dockerfile (rosbridge)},breaklines=true,basicstyle=\footnotesize]
  FROM ros:noetic-robot #Hier wird ein bereits vorhandenes Image aus dem Dockerhub aufgerufen und als Basis verwendet
  RUN apt-get -y update #Mit dem RUN-Befehl werden Befehle in der Kommandozeile ausgefuehrt
  RUN apt-get -y upgrade # -y sorgt dafuer, dass wenn es nach dem ausfuehren von dem Befehl eine JA-NEIN-Frage erscheinen soll, diese automatisch mit JA bzw YES beantwortet wird
  RUN apt-get -y install ros-noetic-rosbridge-server 
  COPY ./start.sh start.sh #COPY-Befehl kopiert die Datei start.h auf unserem Image. COPY SOURCE-Verzeichnis DESTINATTION-Verzeichnis
  RUN chmod +x start.sh
  CMD ["./start.sh"] #Die Datei start.h wird ausgefuehrt
\end{lstlisting}
\chapter{Web-App}\label{ch:webapp}
Die Web-App ist eine Browserapplikation, die manuelle Echtzeitsteuerung der Roboter und deren LED-Farben, unabhängig von deren Anzahl ermöglicht. Die Steuerung erfolgt
über eine ROS-Schnittstelle "'rosbridge"' genannt, mit der sich die Web-App über ein Javascriptbibliothek namens "'\href{http://wiki.ros.org/roslibjs}{roslibjs}"' mit dem \\Rosserver verbinden kann, 
um Daten einzulesen oder zuzusenden.
\\ \colorbox{yellow}{\textbf{\textcolor{red}{Wichtiger Hinweis! }}}\\
\textbf{\textcolor{red}{Die Web-App verbindet sich nur mit der "'rosbridge"', wenn man die IP-Adresse des Hostsrechners manuel in "'rosserver.js"' ändert. Stellen Sie sicher, dass dort die richtige IP-Adresse des Hostsrechners steht. Die Portnummer bleibt unverändert.}}
\newpage
\section{React}

React ist eine Javascript Bibliothek zum Entwickeln und Erstellen von Benutzeroberflächen.
React ist ein Opensource Projekt, welches damals von Facebook, heute Meta genannt, entwickelt wurde.
Um in React Web-Applikationen zu erstellen, braucht man Grundkenntnisse in CSS, HTML und Javascript.\\
Die wichtigste Eigenschaft von React ist, dass die Zustände der Applikation und der Benutzeroberfläche synchronisiert agieren, das heißt, wenn Änderungen am Sourcecode vorgenommen werden,
 verändert sich auch die Benutzeroberfläche.
React ist Komponentenbasiert, ein Reactapp besteht daher aus vielen kleinen React-Komponenten, welche das Programmieren und die Wiederverwendbarkeit von Objekten erleichtern.
\lstdefinelanguage{npm}{
  keywords={npm},
  keywordstyle=\color{blue}\bfseries,
  keywords=[2]{install, start},
  keywordstyle=[2]\color{olive}\bfseries,
  identifierstyle=\color{black},
  sensitive=false,
  comment=[l]{\#},
  commentstyle=\color{purple}\ttfamily,
  stringstyle=\color{red}\ttfamily,
  morestring=[b]',
  morestring=[b]"
}
\section{Einrichten der Web-App}
\begin{enumerate}
    \item Vorbereiten der Umgebung 
    \begin{itemize}
        \item Zuallererst muss ein geeigneter Editor auf dem Rechner installiert sein. Empfohlen wird \href{https://code.visualstudio.com}{Visual Studio Code}, welches ebenfalls zum Entwickeln der Web-App verwendet wurde.
        \item Installation von \href{https://nodejs.org/en/}{Node.js} und npm\footnote[1]{npm ist ein Packet Manager, welcher bei der Installation von Node.js mitgeliefert wird. Durch npm wird das Installieren und 
        Aktualisieren von Drittbibliotheken für Entwickler mit kurzen \href{https://docs.npmjs.com/cli/v6/commands}{Befehlen} erleichtert.}. Bitte installieren Sie hierzu die empfohlene Version und nicht die neueste, da es sein kann, dass 
        manche Befehle nicht mehr verfügbar oder überarbeitet wurden. 
    \end{itemize}
    \item Die \href{https://git.efi.th-nuernberg.de/gitea/sammarimo78617/Webapp-th.git}{Git-Repository} von der Web-App in einem Projekt-Ordner klonen. 
    \item Den Ordner "'Webapp-th"' in Visual Studio Code öffnen.
    \item Ein \href{https://thomaskrause.github.io/nlp-mit-python/01-python-starten/index.html#:~:text=Um%20ein%20neues%20Terminal%20in,”%20oder%20“sh”).}{neues Terminal} in VS-Code öffnen und folgende Befehle eingeben.
 
    \begin{lstlisting}[language=npm,caption={Befehle zum Starten der Web-App},breaklines=true,basicstyle=\footnotesize]
        npm install #Installiert alle Packete, die sich in package.json befinden, welche fuer die Web-App essential sind. Dies kann je nach Internetverbindung eine gewisse Zeit in Anspruch nehmen. 
        npm start #Startet die Web-App. Im Anschluss erscheint die Web-App in dem eingestellten Standardbrowser. 
    \end{lstlisting}
    \item Wenn alle Schritte erfolgreich abgeschlossen sind, erscheint folgendes im Terminalfenster.
    \begin{figure}[H]
        \centering
        \includegraphics[width=0.5\textwidth]{figures/Web-App/Web-App-Erfolgreich.png}
        \caption{Erfolgreicher Web-App-Start}
        \label{fig:erfolgreich-Web-App-Start}
    \end{figure}
Die Web-App wird nun Lokal auf dem Rechner und in unserem Netzwerk, wie es in der Abbildung \ref{fig:erfolgreich-Web-App-Start} zu sehen ist, gehostet. Man kann daher 
die Web-App mit anderen Endgeräten, welche sich im unserem Netzwerk befinden, steuern. 
\end{enumerate}
\section{Bedienung der Web-App-Oberfläche}
Die Web-App besteht aus einer Hauptschaltfläche, die sich je nach Funktion ändert. Die Änderung der Hauptschaltfläche erfolgt mit einem Rechtsklick auf die Navigationsbuttons, welche in der Sidebar vertikal untereinander platziert sind. Die Sidebar kann mit dem Switch-Button
auf- und zugeklappt werden. Oberhalb der Hauptschaltfläche ist die Topbar, welche das Logo der Hochschule und den Switch-Button beinhaltet. 
\begin{figure}[H]
    \centering
    \includegraphics[width=0.8\textwidth]{figures/Web-App/Web-App-Fläche.jpeg}
    \caption{Web-App Oberfläche}
    \label{fig:Web-App-Oberflaeche}
\end{figure}
Die Funktionsnamen der einzelnen Navigationsbuttons werden eingeblendet, wenn man mit der Maus über die Buttons hovert. 
\subsection{Home-Button}
Nach dem Bedienen vom Home-Button ändert sich die Hauptschaltfläche und es wird folgendes wie in der Abbildung \ref{fig:Web-App-Home} angezeigt. Mit dem Button "'Update State"' werden die Daten erneuert und die Anzahl der 
verbundenen Roboter aktualisiert. 
\begin{figure}[H]
    \centering
    \includegraphics[width=0.5\textwidth]{figures/Web-App/Home.png}
    \caption{Web-App/Home}
    \label{fig:Web-App-Home}
\end{figure}
\subsection{LED-Button}
Die Hauptschaltfläche ändert sich folgendermaßen wie in der Abbildung \ref{fig:Web-App-LED}, wenn man den LED-Button auswählt.\label{Dropdown-Menu} Oben links ist ein Dropdown-Menu, welche den jeweiligen Verbindungszustand
oder die \\ Roboternummer erfasst und anzeigen lässt. Diese wurde so eingestellt, dass jedesmal nach Aufruf der Menuleiste, die Verbindungszustände und die Roboternummern erneut erfasst werden, damit der 
Benutzer keine falschen Daten senden kann und ebenfalls sofort erfassen kann, wo das Problem liegt. Falls Verbindungsprobleme angezeigt werden, werden mögliche Problembehebungsmaßnahmen dem Benutzer
in der Menüleiste vorgeschlagen. Im besten Fall funktioniert alles und die verbundenen \\Roboternummern erscheinen in der Menüleiste. Nun kann der Benutzer einen Roboter selektieren, welcher nach dem Selektieren 
in der Dropdown-Menu angezeigt wird. Die ausgewählte Roboternummer wird zwischengespeichert und man kann entweder die LED-Farben mit einem RGB-Generator einstellen und mit dem Set-Color-Button freigeben oder vorprogrammierte Abläufe abspielen lassen. 
\begin{figure}[H]
    \centering
    \includegraphics[width=0.5\textwidth]{figures/Web-App/LED.png}
    \caption{Web-App/LED}
    \label{fig:Web-App-LED}
\end{figure}
\newpage
\subsection{Controller-Button}
Die Hauptschaltfläche vom Controller sieht wie folgt aus.(Siehe Abbildung \ref{fig:Web-App-Controller}). Das Dropdown-Menu funktioniert ähnlich wie bei der \hyperref[Dropdown-Menu]{LED-Hauptschaltfläche} außer, dass die selektierte 
Roboternummer rechts davon angezeigt wird. Unten ist ein Schieberegler, mit dem man die Geschwindigkeit einstellen kann\footnote[1]{Die maximale Geschwindigkeit muss manuell im Code in der Datei "'Controller.js"' geändert werden, da die gepushte Geschwindigkeit ein prozentualer Anteil der Maximalgeschwindigkeit ist. Dieser Prozentualanteil kann mit dem Schieberegler eingestellt werden.}
Der Benutzer hat die Wahl, den Roboter mit einem Joystick oder mit Controller-Buttons zu bedienen. Man kann zwischen den Bedieneroberflächen hin- und herwechseln, wenn man auf den jeweiligen Button drückt.  
\begin{figure}[H]
    \centering
    \includegraphics[width=0.5\textwidth]{figures/Web-App/Controller.png}
    \caption{Web-App/Controller}
    \label{fig:Web-App-Controller}
\end{figure}
\newpage
\section{Web-App Ausblick}
Die Hauptschaltfläche von der Trajektorie ist leer, da während der Entwicklung der Web-App große
Realisierungsprobleme entstanden sind. Die Idee bestand darin, eine Real-Time-Map mit den aktuellen
Roboterpositionen zu entwickeln, indem sich die Roboter auf der Karte, auf den von den Sensoren erfasste
Position, platzieren. Nach einigen Überlegungen und Versuchen konnte festgestellt werden, dass die Idee
so nicht realisierbar ist. Die Sensoren erfassen schließlich nur absolute Positionsdaten, um so eine Karte zu
realisieren, müsste die Fläche der Karte jedes Mal gleich bleiben und die Bildschirmgröße der Web-App
ebenfalls. Da diese zwei Faktoren je nach Umgebung und Bedieneroberfläche unterschiedlich sind, sollten die
Roboterpositionsdaten relativ zu der Bildschirmweite des Endgerätes, auf dem sich die Web-App abspielt, auf einer Karte dargestellt werden.
Ebenso sollten die Roboter als Buttons mit deren Roboternummer als Erkennungsmerkmal für eine leichtere Auswahl und Bedienung dargestellt werden. 
Im Code sollen die Roboter als ein Array mit Buttons realisiert werden, welches alle $5ms$ deren Position auf der Karte und deren Verbindung zum Server aktualisieren sollte. 
Dabei wird das Array alle $5ms$ geleert und mit den erfassten Roboter neu gefüllt. Dies wurde in den Anfangsphasen als die optimale Realisierungsvariante angesehen. Nach mehreren Testversuchen 
konnte man feststellen, dass die Web-App nach jedem Aufruf der Karte abgestürzt ist. Die Warnungen, die in der Web-App-Console angezeigt wurden, wiesen auf zu viele Renders bzw. Performance pro Sekunde hin. 
Auch eine Vergrößerung der Aktualisierungszeit ändert das Ausgangsproblem nicht. Nach einer langen Recherche ist eine neue Realisierungsvariante mit der Hilfe von "'\texttt{ros2djs}"', ein Javascriptbibliothek, für sinnvoll und möglich gehalten worden.
Dazu gibt es leider genauso wie "'\texttt{roslibjs}"' sehr wenige Dokumentationsunterlagen und Beispiele zu den \\Funktionen. Allein das Einbinden und die Installation von "'\texttt{ros2djs}"' mit dem Packetmanager "'\texttt{npm}"'
 war nicht möglich.  Auch ein manuelles Einbinden und das Kontaktieren der Entwickler waren aussichtslos. Daher wurde die Realisierung der Trajektoriehauptschaltfläche nach Absprache mit den Teamkollegen abgebrochen.
\chapter{Fazit \& Ausblick}\label{ch:fazit}
\section{Fazit}
\section{Ausblick}


% Literatur ------------------------------------------------------------------
\clearpage
\renewcommand{\refname}{Literaturverzeichnis}
\bibliography{Bibliographie}
\bibliographystyle{Allgemein/natdin} % DIN-Stil des Literaturverzeichnisses
% !TEX root = Projektdokumentation.tex
\clearpage
\chapter{Eidesstattliche Erklärung}



Wir, \autorName, versichern hiermit, dass wir die \textbf{\betreff} mit dem
Thema
\begin{quote}
\textit{\kompletterTitel}
\end{quote}
selbständig verfasst und keine anderen als die angegebenen Quellen und Hilfsmittel benutzt haben,
wobei wir alle wörtlichen und sinngemä{\ss}en Zitate als solche gekennzeichnet habe. Die Arbeit
wurde bisher keiner anderen Prüfungsbehörde vorgelegt und auch nicht veröffentlicht.\\[6ex]

\abgabeOrt, den \abgabeTermin

\vspace{1cm}


\rule[-0.2cm]{14.5cm}{0.5pt}

\textsc{\autorName}


% Anhang ---------------------------------------------------------------------
\clearpage
\appendix
\pagenumbering{roman}


\end{document}